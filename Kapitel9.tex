\section{Fourierreihe und Fouriertransformation}
\subsection{Fourierreihe}
Im Folgenden sind wir an periodischen Funktionen $f(x)$ interessiert, mit
Periode $L$, d.h.
\begin{align*}
f(x + L) = f(x)
\end{align*}
Insbesondere dürfen die Funktionen auch komplexwertig sein, und sollen die {\em
Dirichlet Bedingung} erfüllen.
\begin{enumerate}
  \item $|f(x)|$ ist integrierbar
  \item $f(x)$ ist stückweise stetig
  \item $f(x)$ hat endlich viele Extrema
  \item $f(x) = \lim \limits_{\varepsilon\to0}
  \frac{1}{2}\left[f(x+\varepsilon) + f(x-\varepsilon)\right]$
\end{enumerate}
Ein spezielles Set von solchen Funktionen ist
\begin{align*}
f_n(x) = e^{i\ k_n\ x}\qquad k_n = \frac{2\pi n}{L},\quad n\in\Z.
\end{align*}
Insbesondere gilt
\begin{align*}
\frac{1}{L} \int \limits_{0}^{L} \dx f_n(x) f_m^*(x) = \frac{1}{L} \int
\limits_{0}^{L} \dx e^{i\ \frac{2\pi}{L}\ x(n-m)} = \delta_{n,m}.
\end{align*}
\begin{Definition}[Fourierkoeffizienten]
Die Fourierkoeffizienten einer Funktion $f(x)$ sind definiert als
\begin{align*}
\hat{f}(n) = \frac{1}{L} \int \limits_{0}^{L} e^{-i\ k_n\ x} f(x) \equiv c_n.
\end{align*}
\end{Definition}
\begin{Definition}[Fourierreihe]
Die Fourierreihe hat somit die Form
\begin{align*}
\sum \limits_{n = -\infty}^{\infty} \hat{f}(n) e^{i\ k_n\ x},
\end{align*}
und es gilt, dass diese Reihe konvergiert und die Funktion identisch zu $f(x)$
ist, d.h. wir können $f(x)$ darstellen als
\begin{align*}
&f(x) = \sum \limits_{n = -\infty}^{\infty} \hat{f}(n) e^{i\ k_n\ x}\quad k_n
= \frac{2\pi n}{L}\\
&\hat{f}(n) = \frac{1}{L} \int \limits_{0}^{L} \dx e^{-i\ k_n\ x} f(x).
\end{align*}
Somit können wir jede Funktion in {\em elementare Schwingungen} zerlegen. 
\end{Definition}
\begin{Bemerkung}
$e^{ikx} = \cos(kx) + i\sin(kx)$ und somit haben wir die Funktion in $\sin$ und
$\cos$ zerlegt.
\end{Bemerkung}
\begin{Bemerkung}
Alternative Form der Fourierreihe
\begin{align*}
f(x) &= \sum \limits_{n=-\infty}^{\infty} \hat{f}(n)e^{i k_n x} = \sum
\limits_{n = -\infty}^{\infty} \hat{f}(n) \cos(k_n x) + i\hat{f}(n) \sin(k_n
x)\\
&= \underbrace{\hat{f}(0)}_{\frac{a_0}{2}} + \sum \limits_{n=1}^{\infty}
\underbrace{\left(\hat{f}(n) + \hat{f}(-n) \right)}_{a_n} \cos(k_n x) + i\sum
\limits_{n=1}^{\infty} \underbrace{\left( \hat{f}(n) - \hat{f}(-n)
\right)}_{b_n} \sin(k_n x) \\
&= \frac{a_0}{2} + \sum\limits_{n=1}^{\infty} a_n
\cos(k_n x) + \sum \limits_{n=1}^{\infty} b_n \sin(k_n x)
\end{align*}
\begin{align*}
\text{mit } & a_n = \frac{2}{L} \int \limits_{0}^{L} \dx \cos(k_n x)f(x)\\
&b_n = \frac{2}{L} \int \limits_{0}^{L} \dx \sin(k_n x)f(x)
\end{align*}
\end{Bemerkung}
\begin{Bemerkung}
\par
\begin{itemize}
  \item Falls $f(x)$ reell ist, so gilt $\hat{f}(n) = \hat{f}^*(-n)$ $(a_n,
  b_n \in \R)$
  \item Falls $f(x)$ symmetrisch ist, d.h. $f(x) = f(-x)$, so gilt $\hat{f}(n)
  = \hat{f}(-n)$ $(b_n = 0)$
  \item Falls $f(x)$ symmetrisch und reell ist, so ist $\hat{f}(n)$ symmetrisch
  und reell. $(a_n \in \R, b_n = 0)$
\end{itemize}
\end{Bemerkung}
Die Idee für den Beweis für den obigen Satz hat die Form
\begin{align*}
F_N(x) &= \sum \limits_{n=-N}^{N} \hat{f}(n) e^{i k_n x} = \frac{1}{L} \int
\limits_{0}^{L} \dy \sum \limits_{n=-N}^{N} e^{i k_n (x-y)} f(y) = \int
\limits_{0}^{L} \dy D_N (x-y) f(x)
\end{align*}
mit
\begin{align*}
D_N(x-y) \equiv
\frac{1}{L}\frac{\sin\left(\frac{2\pi(N+\frac{1}{2})(x-y)}{L}\right)}{\sin\left(\frac{\pi(x-y)}{L}\right)}
= e^{-i N \frac{2\pi}{L}(x-y)}\frac{1-
\left(e^{i\frac{2\pi}{L}(x-y)}\right)^{2N+1}}{1-e^{i\frac{2\pi}{L}(x-y)}}
\end{align*}
Die Funktion $D_N(x)$ gleicht aber der Funktionenreihe, die gegen eine
$\delta$-Funktion konvergiert, d.h.
\begin{align*}
\int \limits_{0}^{L} \dy D(x-y) = 1.
\end{align*}
Formal gilt daher
\begin{align*}
D_N(x-y) \underset{N\to\infty}{\rightarrow} \sum \limits_{j=-\infty}^{\infty}
\delta\left(x-y + jL\right),
\end{align*}
und somit
\begin{align*}
F_N(x) \overset{N\to\infty}{\rightarrow} \int \limits_{0}^{L} \dy \sum
\limits_{j=-\infty}^{\infty} \delta\left(x-y + jL\right)f(y) = f(x)
\end{align*}
\begin{Bemerkung}
Die Relation
\begin{align*}
\sum \limits_{n=-\infty}^{\infty} e^{i k_n x} = \sum
\limits_{j=-\infty}^{\infty} \delta\left(x + jL\right)
\end{align*}
wird in der Physik oft verwendet.
\end{Bemerkung}
\begin{Bemerkung}
Für die Fourierkoeffizienten gilt die Gleichung
\begin{align*}
\frac{1}{L} \int \limits_{0}^{L} \left|f(x)\right|^2 = \sum
\limits_{n=-\infty}^{\infty} \left| \hat{f}(n) \right|^2 \text{ : Satz von
Parseval}
\end{align*}
\end{Bemerkung}