\section{Differentialrechnung}

Ableiten ist der Prozess in dem bestimmt wird, wie sich eine Funktion $f(x)$
ändert unter einer Variation ihres Arguments $x$.


\begin{Beispiel}[Geschwindigkeit eines Massenpunktes]
Bei gleichförmiger Bewegung gilt
\begin{align*}
x(t) = v\cdot t + x_0
\end{align*}
mit der Geschwindigkeit
\begin{align*}
v = \frac{x(t+\Delta t)-x(t)}{\Delta t} \equiv \frac{\Delta
x}{\Delta t}.
\end{align*}

\begin{center}
\psset{unit=1cm}
\begin{pspicture}(-1,-1)(10,6)
 \psline[linewidth=0.5pt,arrowsize=4pt]{->}(-1,0)(10,0)
 \psline[linewidth=0.5pt,arrowsize=4pt]{->}(0,-1)(0,6)
 \psline[linewidth=0.5pt,linecolor=gray,linestyle=dashed](0,4)(10,4)
 \psline[linewidth=0.5pt,linecolor=gray,linestyle=dashed](0,5)(10,5)
 \psplot[linewidth=1.2pt,algebraic=true]{-1}{10}{(1/6)*x+(4-1/3)}
 \psline[linewidth=0.5pt,arrows=-*](2,0)(2,4)
 \psline[linewidth=0.5pt,arrows=-*](8,0)(8,5)
 \rput(-0.36,-0.26){$0$}
 \rput(2,-0.26){$t$}
 \rput(8,-0.26){$t+\Delta t$}
 \rput(9.7,-0.26){$x$}
 \rput(-0.36,4.5){$\Delta x$}
 \rput(-0.36,5.7){$y$}
 \rput(9,5.5){$x(t)$}
\end{pspicture}
\end{center}
\end{Beispiel}

Die Ableitung bestimmt somit das Prinzip, mit dem wir die
Geschwindigkeit eines Massenpunktes auf beliebige Wege $x(t)$ verallgemeinern
können.

Auf einem kleinen Bereich $\Delta x = x_1 - x_0$ ändert sich die Funktion
$f(x)$ um den kleinen Wert
\begin{align*}
\Delta f = f(x_1) - f(x_0).
\end{align*}

\begin{center}
\psset{unit=1cm}
\begin{pspicture}(-1,-1)(10,6)
 \psline[linewidth=0.5pt,arrowsize=4pt]{->}(-1,0)(10,0)
 \psline[linewidth=0.5pt,arrowsize=4pt]{->}(0,-1)(0,6)
 \psline[linewidth=0.5pt,linecolor=gray,linestyle=dashed](0,3.44)(10,3.44)
 \psline[linewidth=0.5pt,linecolor=gray,linestyle=dashed](0,4.56)(10,4.56)
 \psplot[linewidth=1.2pt,algebraic=true]{0}{10}{0.04*x^2+2}
 \psplot[linewidth=1.2pt,algebraic=true]{0}{10}{0.48*(x-6)+3.44}

 \psline[linewidth=0.5pt,arrows=-*](6,0)(6,3.44)
 \psline[linewidth=0.5pt,arrows=-*](8,0)(8,4.56)
 \rput(-0.36,-0.26){$0$}
 \rput(6,-0.26){$x_0$}
 \rput(7,-0.26){$\Delta x$}
 \rput(8,-0.26){$x_1$}
 \rput(9.7,-0.26){$x$}
 \rput(-0.36,4){$\Delta f$}
 \rput(-0.36,5.7){$y$}
 \rput(8.6,5.5){$f(x)$}
\end{pspicture}
\end{center}

Für kleine $\Delta x$ beschreibt somit der Quotient
\begin{align*}
\frac{\Delta f}{\Delta x} = \frac{f(x_1) - f(x_0)}{x_1-x_0}
\end{align*}
das Verhalten der Funktion $f(x)$ im Punkt $x_0$.\\

\begin{Definition}[Ableitung]
Die formale Definition der Ableitung folgt mittels des Grenzwerts.
\begin{equation}
\frac{d}{dx}f(x) \equiv f'(x) = \lim \limits_{\Delta x \to 0}
\frac{f(x+\Delta x) - f(x)}{\Delta x}
\end{equation}
\end{Definition}

Eine Funktion $f(x)$ heißt differenzierbar, wenn dieser Grenzwert existiert. Da
$f'(x)$ wieder Funktion der Variable $x$ ist, können wir höhere Ableitungen
bilden mittels
\begin{equation}
\frac{d^n}{dx^n}f(x) \equiv f^{(n)}(x) = \lim \limits_{\Delta x \to 0}
\frac{f^{(n-1)}(x+\Delta x) - f^{(n-1)}(x)}{\Delta x}
\end{equation}

\begin{Beispiel}[Ableitungen]
\begin{itemize}
\item $\frac{d}{dx}(x^n) = nx^{n-1}$\\
\begin{info}
\begin{align*}
\Delta f &= (x+\Delta x)^n - x^n = \sum \limits_{k}{\binom{n}{k}} x^{n-k}\Delta
x^k -x^n  \\
&= {\binom{n}{k}} \Delta x\,x^{n-1} + \sigma(\Delta x)
\end{align*}
\end{info}
\item $\frac{d}{dx}e^{ax} = ae^{ax}$
\item $\frac{d}{dx}\ln x = \frac{1}{x}$
\item $\frac{d}{dx}\sin x = \cos x$
\item $\frac{d}{dx}\cos x = -\sin x$
\item nicht differenzierbar in $x = 0$
\begin{align*}
f(x) &= |x|
\end{align*}
\begin{align*}
\frac{d}{dx}|x| & = \sgn(x) = \begin{cases}
1 & x > 0\\0 & x = 0\\-1 &  x < 0
\end{cases}
\end{align*}

\begin{center}
\psset{unit=1cm}
\begin{pspicture}(-6,-1)(6,6)
 \psline[linewidth=0.5pt,arrowsize=4pt]{->}(-6,0)(6,0)
 \psline[linewidth=0.5pt,arrowsize=4pt]{->}(0,-1)(0,6)
 \psplot[linewidth=1.2pt,algebraic=true]{-5}{0}{-x}
 \psplot[linewidth=1.2pt,algebraic=true]{0}{5}{x}

 \rput(-0.36,-0.26){$0$}
 \rput(5.7,-0.26){$x$}
 \rput(-0.36,5.7){$y$}
 \end{pspicture}
\end{center}

\begin{align*}
\frac{d^2}{dx^2}|x| &= 2\delta(x) \text{ Dirac $\delta$-Funktion}
\end{align*}
\end{itemize}

\end{Beispiel}


\begin{Definition}[Taylor Reihe]
 Die Ableitungen beschreiben das Verhalten um einen bestimmten Punkt. Daher
können wir eine differenzierbare Funktion in einer kleinen Umgebung 
als {\em Taylor Reihe} approximieren.
\begin{equation}
f(x) = f(x_0) + f'(x_0)(x-x_0) + \frac{f''(x_0)}{2!}(x-x_0)^2 +
\sigma((x-x_0)^3)
\end{equation}
\end{Definition}

\par{\bf Achtung}
Die Approximation kann nicht immer durch Mitnahme höherer Terme beliebig genau
gemacht werden.

\par
Die erste Ableitung beschreibt die Steigung der Kurve während die zweite
Ableitung die Krümmung einer Kurve ergibt. Bei der Bewegung $x(t)$ eines Massenpunktes beschreibt die erste Ableitung die
Geschwindigkeit
\begin{align*}
v = \frac{d}{dt}x(t) \equiv \dot{x}(t),
\end{align*} während die zweite
Ableitung die Beschleunigung ergibt
\begin{align*}
&a = \frac{d^2}{dt^2}x(t) \equiv \ddot{x}(t).\\
&\Rightarrow\;  \text{Newton'sche Gesetz: } m \ddot{x}(t) = F(x,t)
\end{align*}

\subsection{Differentationsregeln}
Die Ableitung ist eine lineare Operation, d.h.
\begin{align*}
&(a\cdot f(x))' = af'(x)\\
&(f(x)+g(x))' = f'(x) + g'(x)\nonumber
\end{align*}
\par{\bf Produktregel}
\begin{equation}
(f(x)\cdot g(x))' = f'(x)g(x) + f(x)g'(x)
\end{equation}
\begin{info}
\begin{align*}
&f(x+\Delta x)g(x+\Delta x) \\
&= (f(x) + f'(x)\Delta x+\sigma(\Delta x^2))(g(x)+g'(x)\Delta x+\sigma(\Delta
x^2)) \\ 
&= f(x)\cdot g(x) + \Delta x(\underbrace{f'(x)g(x)+f(x)g'(x)+\sigma(\Delta
x^2)}_{f(x)g(x)})
\end{align*}
\end{info}
\par{\bf Kettenregel}
Betrachte die Funktion $f(g(x))$. Für die
Ableitung erhalten wir
\begin{equation}
\frac{d}{dx}f(g(x)) = \left( \frac{d}{dg}f(g) \right) \frac{d}{dx}g(x)
\end{equation}
\begin{info}
\begin{align*}
\begin{split}
f(g(x+\Delta x)) \simeq f(g(x)+\underbrace{\Delta xg'(x)}_{\Delta g} \simeq
f(g(x))+\underbrace{f'(g(x))\cdot g'(x)}_{f(g(x))'}\Delta x \\+ \sigma(\Delta
x^2)
\end{split}
\end{align*}
\end{info}
\par{\bf Umkehrfunktion}
\begin{align*}
y &= f(x) \Rightarrow x = f^{-1}(y) \nonumber\\
\frac{d}{dy}f^{-1}(y) &= \frac{1}{\left(\frac{d}{dx}f(x)\right)} =
\frac{1}{f'(f^{-1}(y))}
\end{align*}