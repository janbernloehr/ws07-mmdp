\section{Komplexe Zahlen} %erzeugt einen Abschnitt mit Nummer

Die Gleichung $z²=-1$ hat keine Lösung in den reellen Zahlen. Man kann jedoch
die reellen Zahlen $\mathbb{R}$ erweitern, sodass diese Gleichung ebenfalls
zwei Lösungen besitzt.\\
Dazu führt man die {\em imaginäre Einheit} ein
\begin{equation}
	i \text{ mit der Definition } i^2=-1
\end{equation}
Die beiden Lösungen obiger Gleichung nehmen die Form an
\begin{equation*}
	z = \pm i
\end{equation*}
Eine Zahl in diesem erweiterten Raum kann somit einen Realteil und einen
Imaginärteil besitzen und heißt komplexe Zahl.
\begin{equation}
	z = \underbrace{x}_{\text{Realteil}} + \underbrace{iy}_{\text{Imaginärteil}}
	x,y
	\in
	\mathbb{R}
\end{equation}
Das Rechnen mit komplexen Zahlen verhält sich gleich wie mit reellen Zahlen:
\begin{eqnarray}
(x+iy)+(u+iv) & = & (x+u)+i(y+v)\\
(x+iy)\cdot(u+iv) & = & xu-yv+i(xv+uy)\nonumber
\end{eqnarray}

Eine komplexe Zahl kann in der komplexen Ebene dargestellt werden.
\begin{align}
\operatorname{Re} z &= x = r \cos(\varphi)&& \text{ : Real-Teil}\\
\operatorname{Im} z &= y = r \sin(\varphi)&& \text{ :
Imaginär-Teil}\nonumber\\
r &= \sqrt{x^2+y^2} \equiv |z|&& \text{ : Betrag von z}\nonumber\\
\varphi &= \arctan \frac{y}{x} \equiv \arg(z)&& \text{ : Argument von
z}\nonumber
\end{align}
\begin{center}
\psset{unit=1cm}
\begin{pspicture}(-1,-1)(10,6)
 \psline[linewidth=0.5pt,arrowsize=4pt]{->}(-1,0)(10,0)
 \psline[linewidth=0.5pt,arrowsize=4pt]{->}(0,-1)(0,6)
 
 \psline[linewidth=0.5pt,arrowsize=4pt,linestyle=dashed](0,4)(6,4)
 \psline[linewidth=0.5pt,arrowsize=4pt,linestyle=dashed](6,0)(6,4)
 
 \psline[linewidth=0.5pt,arrowsize=4pt]{->}(0,0)(6,4)
 
 \psarc[linewidth=.5pt](A){1}{0}{33}
 
 \rput(-0.2,-0.3){$0$}
 \rput(6,-0.3){$x$}
 \rput(-0.2,4){$y$}
 \rput(6.2,4.2){$z$}
 
 \rput(3.2,2.4){$r$}
 
 \rput(0.7,0.2){$\varphi$}
 
 \rput(9.7,-0.26){$\Re z$}
 \rput(-0.36,5.7){$\Im z$}
 
\end{pspicture}
\end{center}

\begin{Bemerkung}
\begin{align}
z_1 \cdot z_2 &=
r_1(\cos\varphi_1+i\sin\varphi_1)
r_2(\cos\varphi_2+i\sin\varphi_2) \nonumber \\ &= r_1 r_2
(\cos\varphi_1 \cos\varphi_2 - \sin\varphi_1 \sin\varphi_2 + i(\cos\varphi_1 \sin\varphi_2 + \sin\varphi_1
\cos\varphi_2)) \nonumber \\ &= r_1 r_2 [\cos(\varphi_1 +
\varphi_2) + i\sin(\varphi_1 + \varphi_2)]
\end{align}
d.h. Multiplikation zweier komplexer Zahlen multipliziert ihren Betrag und
addiert ihr Argument.
\begin{center}
\psset{unit=1cm}
\begin{pspicture}(-6,-1)(6,6)
 \psline[linewidth=0.5pt,arrowsize=4pt]{->}(-6,0)(6,0)
 \psline[linewidth=0.5pt,arrowsize=4pt]{->}(0,-1)(0,6)
 %\psplot[linewidth=1.2pt,algebraic=true]{-5}{0}{-x}
 %\psplot[linewidth=1.2pt,algebraic=true]{0}{5}{x}
 
 \psline[linewidth=1.2pt,arrowsize=4pt]{->}(0,0)(3,2)
 \psline[linewidth=1.2pt,arrowsize=4pt]{->}(0,0)(-1,3)
 \psline[linewidth=1.2pt,arrowsize=4pt]{->}(0,0)(-5,3.89)

\psarc[linewidth=.5pt](A){3}{0}{33.7}
\psarc[linewidth=.5pt](A){2}{0}{108.4}
\psarc[linewidth=.5pt](A){5}{0}{142.1}
 \rput(-1.1,2.5){$r_2$}
 \rput(2.5,2){$r_1$}
 \rput(3.2,2.13){$z_1$}
 \rput(-1.1,3.3){$z_2$}
 \rput(-5.1,4.1){$z_1\cdot z_2$}
 \rput(0.3,2.2){$\varphi_2$}
 \rput(3.2,0.2){$\varphi_1$}
 \rput(-3,3){$\varphi_1+\varphi_2$}
 \rput(-5,3.3){$r_1+r_2$}
 \rput(5.7,-0.26){$\Re z$}
 \rput(-0.36,5.7){$\Im z$}
 \end{pspicture}
\end{center}
\end{Bemerkung}

\subsection{Komplex Konjugierte}
\begin{Definition}[Komplexe Konjugation]
Es gibt eine spezielle Funktion, das {\em Komplex Konjugierte}, das jeder
komplexen Zahl eine neue komplexe Zahl zuordnet mittels
\begin{align*}
z = x+iy \Rightarrow \bar{z} = x-iy.
\end{align*}
\end{Definition}

Es gelten die Relationen
\begin{align}
z+\bar{z} &= 2\operatorname{Re}z\\
z-\bar{z} &= 2\operatorname{Im}z \nonumber\\
z\cdot\bar{z} &= |z|^2 \nonumber
\end{align}
Es gelten die Rechenregeln
\begin{align}
(\overline{z_1 + z_2}) &= \overline{z_1}+\overline{z_2} \\
(\overline{z_1 \cdot z_2}) &= \overline{z_1} \cdot \overline{z_2} \nonumber\\
(\overline{z_1 + iz_2}) &= \overline{z_1}-i\overline{z_2} \nonumber
\end{align}
\begin{Bemerkung}
Die Division von komplexen Zahlen erfolgt am einfachsten mit folgendem Trick
\begin{align*}
\frac{z_1}{z_2} = \frac{x + iy}{u + iv} = \frac{z_1 \cdot \bar{z_2}}{|z_2|^2} =
\frac {(x + iy)(u + iv)}{u^2 + v^2} = \frac{xu + yv + i(xv+yu)}{u^2+v^2}
\end{align*}
\end{Bemerkung}

\subsection{Funktionen komplexer Variablen}

\begin{Beispiel}
\begin{align*}
f(z) = z^2 = (x + iy)^2  = x^2 - y^2 + 2ixy
\end{align*}
\end{Beispiel}
Die Ableitung von komplexen Funktionen folgt analog zur Ableitung reeller
Funktionen
\begin{equation}
\frac{d}{dz}f(z)=\lim \limits_{\delta z \to 0} \frac{f(z+\Delta z) -
f(z)}{\Delta z}
\end{equation}
\begin{Bemerkung}
Falls eine komplexe Funktion differenzierbar in einer kleinen Umgebung ist, so
heißt sie {\em analytisch}.
\end{Bemerkung}
\begin{Beispiel}
\begin{align*}
&\frac{d}{dz} z^n = n \cdot z^{n-1}\\
&\operatorname{Re}z,\operatorname{Im}z,\bar{z}, |z|^2 \text{ sind nicht
differenzierbar}
\end{align*}
\end{Beispiel}

\subsection{Exponentialfunktion}
\begin{Definition}[Exponentialfunktion]
Die exponential Funktion $\exp(z)$ ist definiert mittels der Potenzreihe
\begin{equation}
\exp(z) = \sum \limits_{n = 0}^\infty {1 + z +
\frac{z^2}{2} + \frac{z^3}{3!} + \frac{z^4}{4!} + \ldots}
\end{equation}
\end{Definition}

\begin{Bemerkung}
Die Potenzreihe konvergiert für alle $z \in \mathbb{C}$
\begin{align*}
\exp(1) &\equiv e && \text{ Eulersche Zahl}\\
\exp(x) &\equiv e^x && \text{ für } x \in \mathbb{R}\\
\Rightarrow \exp(z) &\equiv e^z && \text{ Erweiterung von } e^z \text{ auf
komplexe Zahlen}
\end{align*}
\end{Bemerkung}

Die Exponentailfunktion hat folgende Eigenschaften
\begin{align*}
\frac{d}{dz} \exp(z) = \exp(z)
\end{align*}
\begin{info}
\begin{align*}
\frac{d}{dz} \exp(z) &= \frac{d}{dz} \left(\sum \limits_{n = 0}^{\infty}
\frac{1}{n!}z^n \right) 
= \sum \limits_{n=0}^{\infty} \frac{1}{n!} \frac{d}{dz} z^n
= \sum \limits_{n=1}^{\infty} \frac{1}{(n-1)!} z^{n-1} \\
& = \sum \limits_{n=0}^{\infty} \frac{1}{n!} z^{n} = \exp(z)
\end{align*}
\end{info}

\begin{align*}
\exp(z_1) \cdot \exp(z_2) = \exp(z_1 + z_2)
\end{align*}
\begin{info}
\begin{align*}
\begin{split}
\exp(z_1) \cdot \exp(z_2) &= \left( \sum \limits_{n=0}^\infty \frac{1}{n!}
z_1^n \right) \left( \sum \limits_{m=0}^\infty \frac{1}{m!}z_2^m \right) 
= \sum \limits_{n=0 \atop m=0}^\infty \frac{1}{n!m!} z_1^n \cdot z_2^m \\
&= \sum \limits_{l=0}^\infty \frac{1}{l!} (z_1+z_2)^l = \exp(z_1 + z_2)
\end{split}
\end{align*}
\begin{center}
\psset{unit=1cm}
\begin{pspicture}(-1,-1)(10,6)
 \psline[linewidth=0.5pt,arrowsize=4pt]{->}(-1,0)(10,0)
 \psline[linewidth=0.5pt,arrowsize=4pt]{->}(0,-1)(0,6)
 
\psdots(1,1)(2,1)(3,1)(4,1)(5,1)(6,1)(7,1)(8,1)(9,1)
\psdots(1,2)(2,2)(3,2)(4,2)(5,2)(6,2)(7,2)(8,2)(9,2)
\psdots(1,3)(2,3)(3,3)(4,3)(5,3)(6,3)(7,3)(8,3)(9,3)
\psdots(1,4)(2,4)(3,4)(4,4)(5,4)(6,4)(7,4)(8,4)(9,4)
\psdots(1,5)(2,5)(3,5)(4,5)(5,5)(6,5)(7,5)(8,5)(9,5)
 
 \rput(9.7,-0.26){$n$}
 \rput(-0.36,5.7){$m$}
 \rput{-45}(0,1.62){\psframe[framearc=0.5](0,0)(2.3,0.5)}
 \rput(0.5,0.5){$k$}
\end{pspicture}
\end{center}
\end{info}

Von besonderem Interesse ist der Wert von $e^z$ für ein imaginäres $z =
i\varphi$:
\begin{align*}
& e^{i\varphi} \cdot \overline{e^{i\varphi}} = |e^{i\varphi}|^2 = e^0 =
e^{i\varphi} e^{-i\varphi} = 1 \\
\Rightarrow\; & e^{i\varphi} \text{ liegt auf dem
Einheitskreis}
\end{align*}
\begin{center}
\psset{unit=1cm}
\begin{pspicture}(-6,-1)(6,6)
 \psline[linewidth=0.5pt,arrowsize=4pt]{->}(-6,0)(6,0)
 \psline[linewidth=0.5pt,arrowsize=4pt]{->}(0,-1)(0,6)
 %\psplot[linewidth=1.2pt,algebraic=true]{-5}{0}{-x}
 %\psplot[linewidth=1.2pt,algebraic=true]{0}{5}{x}
 
 \psline[linewidth=1.2pt,arrowsize=4pt]{->}(0,0)(3,3)

\psarc[linewidth=.5pt](A){4,23}{0}{180}

\rput(0.5,0.2){$\varphi$}
\rput(3.4,3.4){$\exp z$}
\rput(-0.2,4.6){$1$}
 \rput(5.7,-0.26){$\Re z$}
 \rput(-0.36,5.7){$\Im z$}
 \end{pspicture}
\end{center}

Aus der Definition von $\exp(i\varphi)$ folgt
\begin{align}
e^{i\varphi} = \sum \limits_{n=0}^\infty \frac{(i\varphi)^n}{n!} =
\underbrace{\sum \limits_{k=0}^\infty \frac{(-1)^k}{(2k)!}
\varphi^{2n}}_{\cos{\varphi}} + i \underbrace{\sum
\limits_{l=0}^\infty \frac{(-1)^l}{(2l+1)!} \varphi^{2\ln+1}}_{\sin{\varphi}}.
\end{align}
Daraus ergeben sich die Relationen
\begin{align*}
e^{i\varphi} &= \cos(\varphi) + i\sin(\varphi)\\
\cos(\varphi) &= \frac{e^{i\varphi}+e^{-i\varphi}}{2}\\
\sin(\varphi) &= \frac{e^{i\varphi}-e^{-i\varphi}}{2i} 
\end{align*}

\begin{Bemerkung}
Es gilt die {\em Euler'sche Gleichung}
\begin{align}
e^{i\pi}+1 = 0.
\end{align}
\end{Bemerkung}

Jede komplexe Zahl kann somit geschrieben werden als
\begin{align*}
z = r \cdot e^{i\varphi}.
\end{align*}

\begin{center}
\psset{unit=1cm}
\begin{pspicture}(-1,-1)(10,6)
 \psline[linewidth=0.5pt,arrowsize=4pt]{->}(-1,0)(10,0)
 \psline[linewidth=0.5pt,arrowsize=4pt]{->}(0,-1)(0,6)
 
 \psline[linewidth=0.5pt,arrowsize=4pt,linestyle=dashed](0,4)(6,4)
 \psline[linewidth=0.5pt,arrowsize=4pt,linestyle=dashed](6,0)(6,4)
 
 \psline[linewidth=0.5pt,arrowsize=4pt]{->}(0,0)(6,4)
 
 \psarc[linewidth=.5pt](A){1}{0}{33}
 
 \rput(-0.2,-0.3){$0$}
 \rput(6,-0.3){$x$}
 \rput(-0.2,4){$y$}
 \rput(6.2,4.2){$z$}
 
 \rput(3.2,2.4){$r$}
 
 \rput(0.7,0.2){$\varphi$}
 
 \rput(9.7,-0.26){$\Re z$}
 \rput(-0.36,5.7){$\Im z$}
 
\end{pspicture}
\end{center}
\begin{Beispiel}
\begin{align*}
z^n &= (re^{i\varphi})^n = r^ne^{in\varphi}\\
(e^{i\varphi})^2  &= e^{i2\varphi} = \cos2\varphi + i\sin2\varphi \\
 &= (\cos\varphi + i\sin\varphi)^2  =\underbrace{\cos^2\varphi -
 \sin^2\varphi}_{\cos{2\varphi}} +
 i\,\underbrace{2\sin\varphi\cos\varphi}_{i\sin{2\varphi}} \\
 &= \cos(2\varphi) + i\sin(2\varphi)
\end{align*}
\end{Beispiel}

\begin{Bemerkung}
$\cos{(z)}$ und $\sin{(z)}$ sind nun ebenfalls für komplexe Zahlen definiert. 
Insbesondere gilt
\begin{align*}
\cos ix & = \frac{e^{i\cdot ix}+e^{-i\cdot ix}}{2} = \frac{e^{-x}+e^x}{2} =
\cosh x \\
\sin ix & = \frac{e^{-x}-e^x}{2i} = i\frac{e^x-e^{-x}}{2} = i\sinh x
\end{align*}
\end{Bemerkung}

\begin{Beispiel}
\begin{align*}
\cosh 2x = \cos 2ix = \cos^2 ix - \sin^2 ix = \cosh^2 x + \sinh^2 x
\end{align*}
\end{Beispiel}

\subsection {Logarithmus}
\begin{Definition}[Logarithmus Funktion]
Der Logarithmus ist definiert als Umkehrfunktion von $\exp(z)$:
\begin{align*}
z = e^w \Leftrightarrow w = \ln z
\end{align*} 
\end{Definition}

Schreiben wir $z = re^{i\varphi} = re^{i(\varphi + k\cdot2\pi)}$ für $k \in
\mathbb{Z}$, so gilt
\begin{align*}
\ln(z) = \ln\left( re^{i(\varphi + k\cdot2\pi)} \right) = \ln(r) +
\ln\left( e^{i(\varphi + k\cdot2\pi)} \right) = \ln(r) + i(\varphi +
k\cdot2\pi).
\end{align*}
Somit ist der Logarithmus nicht mehr eindeutig.
\par
Die Exponentialfunktion bildet den Bereich $x \in (-\infty, \infty)$ und $y \in
[0,2\pi)$ bereits auf die gesamte komplexe Ebene ab. Daher wissen wir nicht
mehr, von welchem $k$ wir gestartet sind. Der Logarithmus hat daher einen
Schnitt in der komplexen Ebene, in dem die Funktion nicht stetig ist. Die Lage
des Schnittes kann im Prinzip frei gewählt werden.
\par
Mit Hilfe des Logarithmus' lassen sich jetzt auch beliebige Wurzeln und Potenzen
definieren.
\begin{equation*}
f(z) = z^v = \exp(v \cdot \ln z)
\end{equation*}
Für $v = \frac{1}{2}$ erhalten wir die Winkelfunktion
\begin{equation*}
f(z) = \sqrt{z}.
\end{equation*}
Die Vieldeutigkeit des Logarithmus wird auf die Winkelfunktion vererbt
\begin{equation*}
\sqrt{z} = \sqrt{re^{i\varphi}} = \sqrt{r}\;e^{i\frac{\varphi}{2}+ki\pi} \qquad
k = 0,1
\end{equation*}

Die Vieldeutigkeit kann elegant interpretiert werden durch das Einführen von
{\em Riemannschen Blättern}. Man interpretiert die z-Ebene bestehend aus 2
Riemannschen Blättern: Das erste Blatt wird mittels $\sqrt{z}$ auf die obere
Halbebene in der w-Ebene abgebildet. Das zweite Riemannsche Blatt wird auf die
untere Halbebene abgebildet.
\begin{center}
\psset{unit=1cm}
\begin{pspicture}(-6,-4)(6,4)

 \psframe[fillstyle=solid,fillcolor=lightgray,linestyle=none](-5,-2)(-1,2)
 \psframe[fillstyle=solid,fillcolor=lightgray,linestyle=none](1,0)(5,2)
 
 \psline[linewidth=0.5pt,arrowsize=4pt]{->}(-5.5,0)(-0.5,0)
 \psline[linewidth=0.5pt,arrowsize=4pt]{->}(-3,-2.5)(-3,2.5)
 
 \psline[linewidth=0.5pt,arrowsize=4pt]{->}(0.5,0)(5.5,0)
 \psline[linewidth=0.5pt,arrowsize=4pt]{->}(3,-2.5)(3,2.5)
 
 \psarc{->}(0,0){3}{60}{120}
 \psarc{->}(0,0){3}{240}{300}
 
 \rput(0,3.4){$w^2$}
 \rput(0,-2.6){$\sqrt{w}$}
 
 \rput(-0.8,-0.26){$\Re z$}
 \rput(5.2,-0.26){$\Re z$}
 \rput(-3.4,2.3){$\Im z$}
 \rput(2.6,2.3){$\Im z$}
 
\end{pspicture}
\end{center}
Die beiden Riemannschen Blätter heißen Riemannsche Flächen. Die Wurzel Funktion
ist dann analytisch von der Riemannschen Fläche nach $\mathbb{C}$ (außer
$z=0$). Der Schnitt ist die Kreuzungslinie der beiden Blätter.

\begin{Bemerkung}
Für den $\ln(z)$ ist die Riemannsche Fläche eine Spirale.
\end{Bemerkung}
\begin{Beispiel}
Lösungen von $z^n = 1$ haben die Form $$z = e^{i\frac{2\pi}{n}}$$
\end{Beispiel}