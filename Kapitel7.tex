\section{Vektoranalysis}
Wir sind interessiert an Funktionen von $\R^m\to\R^n$.
\begin{itemize}
 \item $f(x)$: reelle Funktion
 \item $f(\vec{r})$: skalares Feld $\phi(\vec{r}), E(\vec{r})$
 \item $\vec{f}(x)$: Raumkurve, $\vec{x}(t), \vec{v}(t)$
 \item $\vec{f}(\vec{r})$: Vektorfeld, $\vec{F}(\vec{r}), \vec{E}(\vec{r})$
\end{itemize}
Ableiten eines Vektors:
\begin{align*}
 \vec{x}(t) &= \begin{pmatrix}x_1(t)\\x_2(t)\\ \vdots\\x_n(t)\end{pmatrix} =
 \sum \limits_{i=1}^{n} x_i(t) \vec{e}_i\\ \Rightarrow \frac{d}{dt}\vec{x}(t) &= \begin{pmatrix} \dot{x}_1(t) \\  \dot{x}_2(t) \\ \vdots \\ \dot{x}_n(t) \end{pmatrix} = \sum \limits_{i=1}^{n} \dot{x}_i(t) \vec{e}_i
\end{align*}
\begin{align*}
 \vec{A}(\vec{r}) &= \begin{pmatrix}A_1(\vec{r})\\A_2(\vec{r})\\
 \vdots\\A_n(\vec{r})\end{pmatrix} = \sum \limits_{i=1}^{n}
 A_i(\vec{r})\vec{e}_i\\ \Rightarrow  \frac{\partial}{\partial x}\vec{A}(\vec{r}) &= \begin{pmatrix}\partial_x A_1(\vec{r})\\ \partial_x A_2(\vec{r})\\ \partial_x A_3(\vec{r}) \end{pmatrix} = \sum \limits_{i=1}^{n} \frac{\partial}{\partial x} A_i(\vec{r}) \vec{e_i}
\end{align*}

\begin{Bemerkung}
 Mittels Kettenregel ermitteln wir das Verhalten von verschiedenen Punkten
 \begin{itemize}
  \item
  $\frac{d}{du}(\phi \vec{a}) = \phi \frac{d\vec{a}}{du} + \left(\frac{d\phi}{du}\right)\vec{a}$
  \item
  $\frac{d}{du}(\vec{a}\cdot\vec{b}) = \vec{a} \frac{d\vec{b}}{du} + \left(\frac{d\vec{a}}{du}\right)\vec{b}$
  \item
  $\frac{d}{du}(\vec{a}\times\vec{b}) = 
   \left(\frac{d\vec{a}}{du}\right)\times\vec{b} +  \vec{a} \times \left(
   \frac{d\vec{b}}{du}\right)$
 \end{itemize}
\end{Bemerkung}
\begin{Beispiel}{\bf  Kugelkoordinaten}
\begin{align*}
 \vec{r}(r,\varphi,\vartheta) = \begin{pmatrix}r\cos(\varphi)\sin(\vartheta)\\r\sin(\varphi)\sin(\vartheta)\\r\cos(\vartheta)\end{pmatrix}
\end{align*}
\begin{align*}
 \Rightarrow \vec{e}_r &= \frac{d\vec{r}}{dr} = \begin{pmatrix}\cos(\varphi)\sin(\vartheta)\\\sin(\varphi)\sin(\vartheta)\\\cos(\vartheta)\end{pmatrix}\\
 \vec{e}_\varphi &= \frac{d\vec{r}}{d\varphi} = \begin{pmatrix}-r\sin(\varphi)\sin(\vartheta)\\r\cos(\varphi)\sin(\vartheta)\\0\end{pmatrix}\\
 \vec{e}_\vartheta &= \frac{d\vec{r}}{d\vartheta} = \begin{pmatrix}r\cos(\varphi)\cos(\vartheta)\\r\sin(\varphi)\cos(\vartheta)\\-r\sin(\vartheta)\end{pmatrix}
\end{align*}

\begin{align*}
\Rightarrow &|\vec{e}_\varphi| = |\vec{e}_\vartheta| = r\\
&|\vec{e}_r| = 1\\
&\vec{e}_r\cdot\vec{e}_\varphi = \vec{e}_r\cdot\vec{e}_\vartheta =
\vec{e}_\varphi\cdot\vec{e}_\vartheta = 0
\end{align*}
\end{Beispiel}

\subsection{Linien- und Oberflächenintegrale}

Ein Linienintegral ist gegeben durch Ausdrücke von der Form
\begin{align*}
I = \int \limits_{C} \dvecr\vec{A}(x,y,z),
\end{align*}
wobei $C$ eine Kurve im Raum darstellt. Die Kurve ist parametrisiert durch
einen Parameter $t$.
\begin{align*}
&[t_0,t_1] \to \R^3\qquad t \mapsto \vec{r}(t) \in C
\end{align*}
mit $\vec{r}(t_0) = \vec{r}_A,\; \vec{r}(t_1) = \vec{r}_B$. Die Berechnung des
Linienintegrals erfolgt dann mittels
\begin{align*}
\dvecr&= \frac{d}{dr}\vec{r}(t) \cdot \dt\\
\Rightarrow I &= \int \limits_{t_0}^{t_1} \dt \left[ \frac{d}{dt}\vec{r}(t)
\right]\cdot \vec{A}(\vec{r}(t))
\end{align*}

\begin{Beispiel}
Die Arbeit entlang eines Weges $\vec{r}(t)$ im Kraftfeld $\vec{F}(\vec{r})$
\begin{align*}
W = \int \limits_{C} \dvecr \vec{F}(\vec{r})
\end{align*}
\end{Beispiel}

\begin{Bemerkung}
Das Integral ist unabhängig von der Parametrisierung der Kurve $C$:
\begin{align*}
t = f(s) \Rightarrow \vec{r}(f(s)) \equiv \vec{\hat{r}}(s)
\end{align*}
\begin{align*}
\Rightarrow I &= \int \limits_{C} \dvecr \vec{A}(\vec{r}) = \int
\limits_{t_0}^{t_1} \dt \left[ \frac{d\vec{r}}{dt}\right]
\vec{A}(\vec{r}(t))\\
& \overset{dt = f'(s)ds,t = f(s)}{=} \int \limits_{s_0}^{s_1} \ds \underbrace{f'
\left[ \frac{d\vec{r}}{dt} \right]}_{\frac{d\vec{r}}{dt}} \vec{A}(\vec{r}(t))
\end{align*}
\end{Bemerkung}

\begin{Beispiel}
\begin{align*}
\vec{F} &= \frac{1}{x^2+y^2}\begin{pmatrix}-y \\ x\end{pmatrix}\\
\vec{r}(\varphi) &= R\begin{pmatrix}\cos(\varphi) \\ \sin(\varphi)
\end{pmatrix}\\
\frac{d}{d\varphi}\vec{r} &= R \begin{pmatrix}-\sin(\varphi) \\ \cos(\varphi)
\end{pmatrix}
\end{align*}
\begin{align*}
\Rightarrow I &= \int \limits_{0}^{2\pi} \dphi R \begin{pmatrix}-\sin(\varphi) \\ \cos(\varphi)
\end{pmatrix} \begin{pmatrix}-\sin(\varphi) \\ \cos(\varphi)
\end{pmatrix} \frac{R}{(\cos^2(\varphi)+\sin^2(\varphi))R^2}\\
&= \int \limits_{0}^{2\pi} \dphi = 2\pi
\end{align*}
\end{Beispiel}

\begin{Bemerkung}{\bf Weitere Linienintegrale}
\begin{align*}
\int \limits_{C} \dvecr f(\vec{r}) = \int \limits_{t_0}^{t_1} \dt
f(\vec{r}(t)) \frac{d\vec{r}}{dt} \text{ : Vektorgröße}
\end{align*}
{\bf Bogenlänge}
\begin{align*}
&\int \limits_{C} \ds = \int \limits_{t_0}^{t_1} \dt
\left|\frac{d\vec{r}(t)}{dt}\right|\\ \Rightarrow  &\int \limits_{C} \ds
\phi(\vec{r})
\text{ : Skalare Größe}\\ &\int \limits_{C} \ds \vec{F}(\vec{r}) \text{ : Vektorgröße}
\end{align*}
\end{Bemerkung}

\begin{Beispiel}[Länge einer Geraden]
\begin{align*}
&\vec{r}(t) = \begin{pmatrix}t\\ \sin(\varphi) t\end{pmatrix}\qquad 0\le
t\le1\\ &\int \limits_{C} \ds = \int \limits_{0}^{1} \dt \sqrt{1+\sin(\varphi)} =
\sqrt{1+\sin^2(\varphi)}
\end{align*}
\end{Beispiel}
\begin{Beispiel}[Umfang eines Kreises]
\begin{align*}
 &\vec{r}(\varphi) =
R\begin{pmatrix}\cos(\varphi)\\ \sin(\varphi) \end{pmatrix}\\
 &L = \int \limits_{C} \ds = \int \limits_{0}^{2\pi} \dphi
 R\sqrt{\sin^2(\varphi) + \cos^2(\varphi)} = R\cdot2\pi
 \end{align*}
\end{Beispiel}

Eine Fläche im $\R^3$ können wir darstellen mit Hilfe von zwei Parametern.
\begin{align*}
\vec{r}(u,v) = \begin{pmatrix}x(u,v)\\y(u,v)\\z(u,v)\end{pmatrix}
\end{align*}

\begin{Beispiel}
Eine Ebene aufgespannt durch $\vec{a}$ und $\vec{b}$ durch den Punkt $\vec{r}_0$
hat die Form
\begin{align*}
\vec{r}(u,v) = \vec{r}_0 + u\vec{a} + v\vec{b}.
\end{align*}
Eine Kugel mit dem Radius $R$
\begin{align*}
\vec{r}(\varphi, \vartheta) =
\begin{pmatrix}\cos(\varphi)\sin(\vartheta)\\\sin(\varphi)\sin(\vartheta)\\\cos(\vartheta)\end{pmatrix}.
\end{align*}
\end{Beispiel}

Die Vektoren
\begin{align*}
&\vec{e}_u = \frac{\partial}{\partial u} \vec{r}(u,v)\\
&\vec{e}_v = \frac{\partial}{\partial v} \vec{r}(u,v)\\
\end{align*}
beschreiben die Tangentenvektoren an die Linien auf der Fläche mit $v=const$
und $u=const$.\\
Das Flächendifferential $dA$ in einem Punkt hat somit die Form
\begin{align*}
dA = |\vec{e}_u(u,v) \times \vec{e}_v(u,v)| \overset{du\ dv}{=}
\left|\frac{d\vec{r}}{du}(u,v) \times \frac{d\vec{r}}{dv}(u,v)\right|du\ dv
\end{align*}
Somit ist der Flächeninhalt einer Fläche im $\R^3$ gegeben durch
\begin{align*}
F_B = \int \int \limits_{B} \du \dv \left|\frac{d}{du}\vec{r}(u,v) \times
\frac{d}{dv}\vec{r}(u,v)\right|
\end{align*}
\begin{Beispiel}{Darstellung der Fläche}
\begin{align*}
\vec{r}(u,v) = \begin{pmatrix}1\\0\\0\end{pmatrix} + u
\begin{pmatrix}-1\\1\\0\end{pmatrix} + v \begin{pmatrix}-1\\0\\1\end{pmatrix}
\end{align*}
\begin{align*}
\Rightarrow & \vec{e}_u = \begin{pmatrix}-1\\1\\0\end{pmatrix}\\
& \vec{e}_v = \begin{pmatrix}-1\\0\\1\end{pmatrix}\\
\Rightarrow & \left|\frac{d\vec{r}}{du}(u,v) \times
\frac{d\vec{r}}{dv}(u,v)\right| =
\left|\begin{pmatrix}1\\1\\1\end{pmatrix}\right| =
\sqrt{3}
\end{align*}
\begin{align*}
F_B = \int \limits_{0}^{1} \dv \int \limits_{0}^{1-v} \du \sqrt{3} = \int
\limits_{0}^{1} \dv \sqrt{3}(1-v) = \frac{\sqrt{3}}{2}
\end{align*}
\end{Beispiel}
Bei einem Oberflächenintegral wird jetzt jeder Punkt auf der Oberfläche mit
einer Funktion gewichtet.
\begin{align*}
\int \limits_{S} \ds \phi(\vec{r}) = \int \limits_{B} \du \dv
\phi(\vec{r}(u,v)) \left|\frac{d\vec{r}}{du}(u,v) \times
\frac{d\vec{r}}{dv}(u,v)\right| \text{ : Skalares Integral}
\end{align*}
Es ist jedoch auch möglich vektorielle Oberflächenintegrale zu definieren.
Insbesondere ist das Oberflächenelement
\begin{align*}
d\vec{A} = \frac{d\vec{r}}{du} \times
\frac{d\vec{r}}{dv} du dv = \vec{n} dA
\end{align*}
eine Vektorgröße, wobei der Einheitsvektor $\vec{n}$ senkrecht auf der
Oberfläche steht.
\begin{align*}
\vec{n} = \frac{\frac{d\vec{r}}{du} \times
\frac{d\vec{r}}{dv}}{\left| \frac{d\vec{r}}{du} \times
\frac{d\vec{r}}{dv} \right|}
\end{align*}
Die Richtung von $\vec{n}$ hängt von der Orientierung der Oberfläche ab. Bei
geschlossenen Flächen, wie z.B. einer Kugel, wird die Orientierung normalerweise
so gewählt, dass $\vec{n}$ nach außen zeigt.

\begin{Beispiel}[Kugel]
\begin{align*}
&\vec{r}(\vartheta,\varphi) =
\begin{pmatrix}\cos(\varphi)\sin(\vartheta) \\
\sin(\varphi)\sin(\vartheta) \\ \cos(\vartheta)\end{pmatrix}\\
&\vec{e}_\varphi = \frac{\partial \vec{r}}{\partial \varphi} =
\begin{pmatrix}-\sin(\varphi)\sin(\vartheta) \\ \cos(\varphi)\sin(\vartheta) \\
0\end{pmatrix}\\
&\vec{e}_\vartheta = \frac{\partial \vec{r}}{\partial \vartheta} =
\begin{pmatrix}\cos(\varphi)\cos(\vartheta) \\ \sin(\varphi)\cos(\vartheta) \\
-\sin(\vartheta)\end{pmatrix}\\
\Rightarrow & \vec{e}_\vartheta \times \vec{e}_\varphi =
\begin{pmatrix}\cos(\varphi)\sin^2(\vartheta) \\ \sin(\varphi)\sin^2(\vartheta)
\\ \cos(\vartheta)\sin(\vartheta)\end{pmatrix} = \sin(\vartheta) \begin{pmatrix}\cos(\varphi)\sin(\vartheta) \\ \sin(\varphi)\sin(\vartheta)
\\ \sin(\vartheta)\end{pmatrix}\\
&\left| \vec{e}_\vartheta \times \vec{e}_\varphi \right| = |\sin(\vartheta)|\\
\Rightarrow & \vec{n} = \begin{pmatrix}\cos(\varphi)\sin(\vartheta) \\
\sin(\varphi)\sin(\vartheta) \\ \cos(\vartheta)\end{pmatrix} =
\vec{r}(\vartheta, \varphi)
\end{align*}
\end{Beispiel}

Wir finden jetzt die weiteren Oberflächenintegrale
\begin{align*}
&\int \limits_{S} \dvecs \vec{F}(\vec{r}) = \int \limits_{B} \du\dv
\vec{F}(u,v) \cdot \left( \frac{\partial \vec{r}}{\partial u} \times
\frac{\partial \vec{r}}{\partial v}\right) \text{ : skalare Größe}\\
&\int \limits_{S} \dvecs \phi(\vec{r}) = \int \limits_{B} \du\dv
\phi(u,v) \left( \frac{\partial \vec{r}}{\partial u} \times
\frac{\partial \vec{r}}{\partial v}\right) \text{ : Vektorgröße}\\
&\int \limits_{S} \dvecs \times \vec{F}(\vec{r}) = \int \limits_{B} \du\dv
 \left( \frac{\partial \vec{r}}{\partial u} \times
\frac{\partial \vec{r}}{\partial v}\right) \times \vec{F}(u,v) \text{ :
Vektorgröße}\\
\end{align*}

\begin{Beispiel}
\begin{itemize}
\item  Der Tragflügel eines Flugzeugs erzeugt durch seine Form und den Luftstrom
verschiedene Drücke auf der Unter- und Oberseite. Bezeichnen wir mit
$p(\vec{r})$ das erzeugte Druckfeld, so ist der Auftrieb des Flugzeugs gegeben
durch
\begin{align*}
\vec{F} = \int \limits_{S} \dvecs  p(\vec{r}) \text{ : wobei $S$ die
Oberfläche der Tragflächen ist.}
\end{align*}
\item Eine Flüssigkeit fließt mit einem Geschwindigkeitsfeld
$\vec{v}(\vec{r})$. D.h. jedem Punkt $\vec{r}$ ordnen wir die lokale
Geschwindigkeit der Flüssigkeit zu. Zudem hat sie eine Massendichte
$\rho(\vec{r})$.\\
Die lokale Masse, die pro Zeiteinheit durch die Oberfläche fließt, ist
gegeben durch
\begin{align*}
M = \int \limits_{S} \dvecs \cdot \vec{v}(\vec{r})\rho(\vec{r}).
\end{align*}
\end{itemize}
\end{Beispiel}

\subsection{Ableitungsoperatoren}
\begin{Definition}[Gradient]
Ein Skalarfeld $\phi(\vec{r})$ ordnet jedem Punkt $\vec{r}$ im Raum eine reelle
Zahl zu. Der Gradient des Skalarfeldes ist definiert als
\begin{align*}
\nabla \phi(\vec{r}) = \frac{\partial \phi}{\partial x}\vec{e}_x +
\frac{\partial \phi}{\partial y}\vec{e}_y + \frac{\partial \phi}{\partial
z}\vec{e}_z.
\end{align*}
Somit ist $\nabla \phi(\vec{r})$ ein Vektorfeld. In seinen Komponenten hat es
die Form (in der natürlichen Basis)
\begin{align*}
\nabla \phi(\vec{r}) = \begin{pmatrix}\partial_x\phi \\
\partial_y\phi \\ \partial_z\phi\end{pmatrix}.
\end{align*}
\end{Definition}

Das Verhalten des Skalarfeldes von einem Punkt $\vec{r}_0$ lässt sich mit dem
Gradienten sehr einfach beschreiben.
\begin{align*}
&\vec{r}(s) = \vec{r}_0 + s\vec{a}\\
\Rightarrow & \frac{\partial \phi(\vec{r}(s))}{ds} = \vec{a} \cdot \nabla
\phi(\vec{r}_0) \text{ : Richtungsableitung}
\end{align*}
Der Gradient zeigt somit in die Richtung der stärksten Zunahme des
Skalarfeldes.\\
Die Gleichung $\phi(\vec{r}) = const$ bestimmt eine Fläche im $\R^3$, wobei der
Gradient immer senkrecht auf dieser Fläche steht.\\

\begin{Definition}[Ableitungsoperator]
Der Name {\em Ableitungsoperator} kommt von der Eigenschaft, dass jedem
Skalarfeld ein Vektorfeld zugeordnet ist. Daher schreibt man auch oft
\begin{align*}
\nabla = \partial_x \vec{e}_x + \partial_y \vec{e}_y + \partial_z \vec{e}_z, 
\text{ : Nabla}
\end{align*}
wobei für jedes Skalarfeld gilt
\begin{align*}
\nabla \phi(\vec{r}) = (\partial_x \vec{e}_x + \partial_y \vec{e}_y +
\partial_z \vec{e}_z)\phi(\vec{r}) = \partial_x \phi \vec{e}_x + \partial_y \phi
\vec{e}_y + \partial_z \phi \vec{e}_z
\end{align*}
\end{Definition}


\begin{Beispiel}
  \begin{align*}
  &\phi(x,y,z) = xyz\qquad \nabla\phi =
  \begin{pmatrix}yz\\xz\\yx\end{pmatrix}\\
   &\phi(x,y,z) = MmG \frac{1}{\sqrt{x^2+y^2+z^2}} \text{ :
 Gravitationspotential}\\
 \Rightarrow & F_g = -\nabla \phi = MmG
 \frac{1}{\sqrt{x^2+y^2+z^2}^{\frac{3}{2}}}\begin{pmatrix}x\\y\\z\end{pmatrix}
 \text{ : Gravitationskraft}
 \end{align*}

\end{Beispiel}

\begin{Definition}[Divergenz]
Betrachte ein Vektorfeld $\vec{A}(\vec{r})$, das jedem Punkt $\vec{r}$ einen
Vektor zuordnet.\\
Die {\em Divergenz} von $\vec{A} =
\begin{pmatrix}A_x\\A_y\\A_z\end{pmatrix}$ ist definiert als
\begin{align*}
\div \vec{A} \equiv \nabla \cdot \vec{A} = \frac{\partial A_x}{\partial x} +
\frac{\partial A_y}{\partial y} + \frac{\partial A_z}{\partial z}.
\end{align*}
\end{Definition}
Die Divergenz eines Vektorfeldes beschreibt physikalisch Quellterme.

\begin{itemize}
  \item Die Divergenz eines $\vec{E}$-Feldes verschwindet, wenn keine Ladungen
  vorhanden sind: $\div \vec{E} = 0$.
  \item Für das Geschwindigkeitsfeld $\vec{v}(\vec{r})$ und die Massendichte
  $\rho(\vec{r})$ einer fließenden Flüssigkeit gilt $\div \left[
  \rho(\vec{r})\vec{v}(\vec{r}) \right] = 0$, wenn keine Quelle/Abfluss
  vorhanden ist.
\end{itemize}

\begin{Beispiel}
\begin{align*}
&\vec{A}(x,y,z) = \begin{pmatrix}x\\y\\z\end{pmatrix} \Rightarrow \div \vec{A}
= 3\\
&\vec{A}(x,y,z) = \begin{pmatrix}y\\z\\x\end{pmatrix} \Rightarrow \div \vec{A}
= 0\\
\end{align*}
\end{Beispiel}

\begin{Definition}[Laplace Operator]
Die Kombination von Gradient und Divergenz ergibt den {\em Laplace Operator}
für ein Skalarfeld $\phi(\vec{r})$.
\begin{align*}
\Delta \phi(\vec{r}) = \div \nabla \phi(\vec{r}) = \partial^2_x\phi +
\partial^2_y\phi + \partial^2_z\phi
\end{align*}
\end{Definition}

\begin{Beispiel}
Gravitationspotential $\phi(\vec{r}) = MmG \frac{1}{\sqrt{x^2+y^2+z^2}}$
\begin{align*}
\Rightarrow \Delta \phi(\vec{r}) = 0 \text{ für } \vec{r} \neq 0
\end{align*}
Da aber bei $\vec{r} = 0$ eine Quelle/Masse sitzt, die ein Gravitationsfeld
erzeugt, sollte also $\div \nabla \phi(\vec{r})$ bei $\vec{r} = 0$ nicht
verschwinden. In der Tat gilt:
\begin{align*}
\Delta \phi(\vec{r}) = 4\pi MmG
\underbrace{\delta(x)\delta(y)\delta(z)}_{\delta(\vec{r})},
\end{align*}
mit der bekannten $\delta$-Funktion.
\end{Beispiel}

\begin{Definition}[Rotation]
Die {\em Rotation} ist definiert für ein Vektorfeld $\vec{A}(\vec{r})$ mittels
\begin{align*}
\rot \vec{A}(\vec{r}) &= \nabla \times \vec{A}(\vec{r}) \\
&=
\vec{e}_x\left(\partial_y A_z - \partial_z A_y\right) +
\vec{e}_y\left(\partial_z A_x - \partial_x A_z\right) +
\vec{e}_z\left(\partial_x A_y - \partial_y A_x\right) \\
&=
\begin{pmatrix}\partial_y A_z - \partial_z A_y \\ \partial_z A_x -
\partial_x A_z \\ \partial_x A_y - \partial_y A_x\end{pmatrix}
\end{align*}
und ist somit wieder ein Vektorfeld.
\end{Definition}

Die physikalische Interpretation ist, dass die Rotation Wirbel/Drehungen
beschreibt:
\par
Betrachte das Kraftfeld $\vec{F}(x,y,z) =
\begin{pmatrix}0\\x\\0\end{pmatrix}$.\\ 
Ein Objekt das wir in einem solchen Feld platzieren, beginnt sich um seine
eigenen Achse zu drehen.
\begin{align*}
\rot \vec{F} = \begin{pmatrix}0\\0\\ \partial_x F_y\end{pmatrix} =  
\begin{pmatrix}0\\0\\1\end{pmatrix} \text{ : Drehachse des Objekts}
\end{align*}



\begin{Beispiel}
{\bf Maxwell Gleichung Elektrodynamik}
 \begin{align*}
  &\vec{E} : \text{ elektrisches Feld}, && \vec{B} : \text{ magnetisches Feld}\\
  &\rho : \text{ Ladungsdichte}, &&j : \text{ Ladungsstrom}\\
  &\div \vec{E} = 4\pi\rho(\vec{r}) && \rot \vec{E} + \frac{1}{L}\partial_t
  \vec{B} = 0\\
   &\div \vec{B} = 4\pi\rho(\vec{r}) && \rot \vec{B} + \frac{1}{L}\partial_t
   \vec{E} = j(\vec{r}
 \end{align*}
{\bf Quantenmechanik des Wasserstoffatoms}
\begin{align*}
 i\hbar\partial_t \psi(\vec{r},t) = \frac{\hbar^2}{2m}\Delta\psi(\vec{r},t) + \frac{l^2}{|\vec{r}|}\psi(\vec{r},t)
\end{align*}
{\bf Navier-Stokes}
\begin{align*}
 \rho\left(\frac{d\vec{v}}{dt} +\vec{v}\cdot\nabla\cdot\vec{v}\right) = -\nabla p + \mu \Delta\vec{v}
\end{align*}
\end{Beispiel}

\begin{Bemerkung}
Für ein Skalarfeld $\phi(\vec{r})$ gilt
\begin{align*}
\rot \nabla \phi(\vec{r}) = 0
\end{align*}
\begin{info}
\begin{align*}
\rot \begin{pmatrix}\partial_x \phi \\ \partial_y \phi \\
\partial_z \phi\end{pmatrix} = \begin{pmatrix}\partial_y\partial_z \phi -
\partial_z\partial_y \phi \\ \partial_z\partial_x \phi - \partial_x\partial_z
\phi \\ \partial_x\partial_y \phi - \partial_y\partial_x \phi\end{pmatrix} = \begin{pmatrix}0\\0\\0\end{pmatrix}
\end{align*}
\end{info}
\par
Für ein Vektorfeld $\vec{A}(\vec{r})$ gilt
\begin{align*}
\div \rot \vec{A}(\vec{r}) = 0
\end{align*}
\end{Bemerkung}

Zudem sind folgende Relationen einfach zu beweisen
\begin{align*}
&\nabla\cdot(\phi\vec{A}) = (\nabla\phi)\cdot\vec{A} + \nabla\cdot\vec{A}\\
&\nabla\times(\phi\vec{A}) = (\nabla\phi)\times\vec{A} +
\phi(\nabla\times\vec{A})\\
&\nabla\cdot(\vec{A}\times\vec{B}) = \vec{B}\cdot(\nabla\times\vec{A}) -
\vec{A}\cdot(\nabla\times\vec{B})\\
&\nabla\times(\vec{A}\times\vec{B}) = \vec{A}(\nabla\cdot\vec{B}) -
\vec{B}(\nabla\cdot\vec{A}) + (\vec{B}\cdot\nabla)\vec{A} -
(\vec{A}\cdot\nabla)\vec{B}\\
&\nabla\times(\nabla\times\vec{A}) = \nabla(\nabla\cdot\vec{A}) - \Delta\vec{A}
\end{align*}

\subsection{Gauß'scher Integralsatz}
Die Integralsätze stellen einen Zusammenhang her zwischen den
Ableitungsoperatoren und den Oberflächenintegralen. Der Gauß'sche Integralsatz
besagt
\begin{align*}
\int \limits_{V} \dvecr \div \vec{A}(\vec{r}) = \int \limits_{S=\partial V} \dvecs \cdot
\vec{A}(\vec{r}) = \int \limits_{S = \partial V} \ds
\vec{n}\cdot\vec{A}(\vec{r}),
\end{align*}
wobei $\partial V$ die Oberfläche des Volumens $V$ beschreibt und $\vec{n}$
senkrecht auf der Oberfläche steht und nach außen zeigt.
\par
\begin{info}
Für einen Quader mit dem Volumen $V$ gilt
\begin{align*}
&\int \limits_{V} \dV \div \vec{A} \\
=&\int
\limits_{0}^{a}\dx\int\limits_{0}^{b}\dy\int\limits_{0}^{c}\dz
\left[\partial_xA_x + \partial_yA_y + \partial_zA_z\right]\\
=&\int\limits_{0}^{b}\dy\int\limits_{0}^{c}\dz A_x(a,y,z) -
\int\limits_{0}^{b}\dy\int\limits_{0}^{c}\dz A_x(0,y,z) \\&+
\int\limits_{0}^{a}\dx\int\limits_{0}^{c}\dz A_y(x,b,z) -
\int\limits_{0}^{a}\dx\int\limits_{0}^{c}\dz A_y(x,0,z) \\&+
\int\limits_{0}^{a}\dx\int\limits_{0}^{b}\dy A_z(x,y,c) - 
\int\limits_{0}^{a}\dx\int\limits_{0}^{b}\dy A_z(x,y,0) \\=& \int
\limits_{\partial V = S} \dvecs \cdot\vec{A}
\end{align*}
Für ein allgemeines Volumen folgt der Satz durch Zerlegen des Volumens in
kleine Quader und mittels einem Grenzwert.
\end{info}

\begin{Bemerkung}
Falls das umschlossene Gebiet Löcher aufweist, so müssen diese Löcher bei der
Bestimmung des Randes berücksichtigt werden.
\end{Bemerkung}

\begin{Beispiel}
Betrachte das Vektorfeld
\begin{align*}
\vec{A} = \vec{r} = \begin{pmatrix}x\\y\\z\end{pmatrix},\quad \Rightarrow \div
\vec{A} = 3
\end{align*}
mit dem Volumen $V$ einer Kugel mit Radius $R$.
\begin{align*}
\Rightarrow & \int \limits_{V} \dvecr  \div \vec{A} = 3\frac{4\pi}{3} R^3
= 4\pi R^3\\
& \int \limits_{S = \partial V} \ds \vec{n}\cdot\vec{r} = \int \limits_{0}^{2\pi}\dphi \int
\limits_{0}^{\pi}\dtheta \sin(\theta) \cdot R^3 = 4\pi R^3
\end{align*}
{\em Kontinuitätsgleichung} Die Änderung der Teilchenzahl von einer Flüssigkeit
im Volumen $V$ hat die Form
\begin{align*}
\partial_t N_v(t) = \partial_t \int \limits_{V} \dvecr \varphi(\vec{r})t
= \int \limits_{V} \dvecr \partial_t \varphi(\vec{r}, t), \qquad
\varphi(\vec{r}, t) \text{ : Teilchendichte}
\end{align*}
Teilchenfluss aus dem Volumen: $\int \limits_{\partial V = S}\dvecr \cdot
\underbrace{\varphi(\vec{r},t)\vec{v}(\vec{r},t)}_{j(\vec{r},t) : \text{
Teilchenstrom}}$
\begin{align*}
\Rightarrow \partial_t N_v(t) &= \int \limits_{V} \dvecr
\partial_t\varphi(\vec{r},t) = -\int \limits_{\partial V = S}\dvecs
\vec{j}(\vec{r},t) \\
&= -\int \limits_{V}\dvecr \div j(\vec{r},t) : \text{
gilt für alle Volumen } V
\end{align*}
\begin{align*}
\Rightarrow \partial_t \varphi(\vec{r},t) + \div j(\vec{r},t) = 0
\end{align*}
\end{Beispiel}

\subsection{Integralsatz von Stokes}
Der {\em Satz von Stokes} verbindet Oberflächenintegrale mit Linienintegralen
entlang der Begrenzungslinie der Oberfläche. Für ein Vektorfeld
$\vec{A}(\vec{r})$ gilt
\begin{align*}
\int \limits_{S}\dvecs \cdot\left(\rot \vec{A}(\vec{r})\right) = \int
\limits_{C = \partial S} \dvecr \cdot \vec{A}(\vec{r}).
\end{align*}
Dabei bildet der Normalenvektor $\vec{n}$ auf der Oberfläche mit der
Umlaufrichtung der Linie $C$ eine rechtshändige Schraube.

\begin{Bemerkung}
Die Fläche soll orientierbar sein, der Satz ist nicht anwendbar auf ein
Möbiusband.
\end{Bemerkung}
\par
Für ein Vektorfeld mit $\rot \vec{A} = 0$ gilt somit, dass alle geschlossenen
Linienintegrale verschwinden.
\begin{align*}
\int \limits_{C} \dvecr \vec{A}(\vec{r}) = 0
\end{align*}
$\Rightarrow$ man kann jedoch nun zeigen, dass somit ein Skalarfeld $\phi$
existiert mit
\begin{align*}
\vec{A}(\vec{r}) = \nabla\phi(\vec{r}).
\end{align*}


\subsection{Anwendung: Coulomb Potential}
Eine homogen gefüllte Kugel mit Masse $M$ erzeugt ein Gravitationspotential
\begin{align*}
\phi(\vec{r}) = \frac{\alpha}{|r|} \qquad & : \alpha = MmG\\
& : |r| > R \text{ Radius der Kugel}\\
& : M = \frac{4\pi}{3}R^3\rho 
\end{align*}
Mittels dem Gradienten erzeugt dies das Gravitationsfeld
\begin{align*}
\vec{F}(\vec{r}) = -\nabla\phi(\vec{r}) = -\alpha \frac{\vec{r}}{|r|^3}.
\end{align*}
Weiter gilt, dass $\div \vec{F}(\vec{r}) = 0$ für $\vec{r}\neq0$. Es bleibt
somit die Frage, was bei $\vec{r} = 0$ passiert. Allerdings macht das
Resultat keinen Sinn für $|r| < R$, da sich innerhalb der Kugel das Potential
verändert. Der Satz von Gauß besagt wiederum für $S$ eine Kugel mit Radius $> R$
\begin{align*}
\int \limits_{S}\dvecs \cdot\vec{F}(\vec{r}) &= \int
\limits_{0}^{2\pi}\dphi \int \limits_{0}^{\pi}\dtheta \sin(\vartheta) \alpha \cdot
\frac{\vec{r}}{|r|}\cdot\frac{\vec{r}}{|r|^3}\cdot|r|^2 \\
&= \int
\limits_{0}^{2\pi}\dphi \int\limits_{0}^{\pi}\dtheta \sin(\vartheta) \alpha =
4\pi\alpha : \text{ unabhängig von } L\\
&\overset{\text{Satz von Gauß}}{=} \int \limits_{V}\dvecr
\div\vec{F}(\vec{r}),
\end{align*}
also muss die $\div \vec{F}$ innerhalb des
massiven Körpers gerade so sein, dass sich die Konstante $4\pi\alpha$ ergibt.
\par
Der Körper hat eine homogene Massendichte und daher lässt sich das Potential
schreiben
\begin{align*}
\phi(\vec{r}) = \int \limits_{V}\dvecr \frac{3\alpha}{4\pi
R^3}\frac{1}{|r-r'|}
\end{align*}
Für $|r| > R$ ergibt dies $\phi(\vec{r}) = \frac{\alpha}{|r|}$, aber innerhalb
des Körpers wird das Potential zu
\begin{align*}
\phi(\vec{r}) =
\begin{cases}\frac{3}{2}\frac{\alpha}{R}-\frac{\alpha}{2}\frac{|r|^2}{R^3} &
|r| < R
\\
\frac{\alpha}{|r|} & |r| > R\end{cases}
\end{align*}
und das Gravitationsfeld wird zu
\begin{align*}
\vec{F}(\vec{r}) = -\nabla\phi(\vec{r}) = 
\begin{cases}\alpha\frac{\vec{r}}{|R^3|} & |r| < R \\
\alpha\frac{\vec{r}}{|r|^3} & |r| > R\end{cases}
\end{align*}
und die Divergenz
\begin{align*}
\div \vec{F}(\vec{r}) = \begin{cases}\frac{3\alpha}{R^3} & |r| < R \\
0 & |r| > R\end{cases}
\end{align*}
Der Satz von Gauß ist somit erfüllt für den realen Fall einer homogenen Kugel.
\par
Um jetzt das Verhalten einer reinen Punktladung zu untersuchen, lassen wir den
Radius der Kugel gegen Null gehen, behalten die Masse aber konstant.
\par
Im Limes $R\to0$ konvergiert $\div \vec{F}(\vec{r})$ aber gegen eine
$3$-dimensionale $\delta$-Funktion
\begin{align*}
\div \vec{F}(\vec{r}) \underset{R\to0}{\rightarrow} 4\pi\delta{\vec{r}}
\end{align*}
\begin{info}
$\int \dvecr \div\vec{F}(\vec{r}) = 4\pi$ unabhängig von R und für $R\to0$
ist alles Gewicht in einer kleinen Kugel um den Ursprung konzentriert.
\end{info}
\par
Somit gilt für das Potential einer punktförmigen Masse
\begin{align*}
&\phi(\vec{r}) = \frac{\alpha}{|\vec{r}|}\qquad \vec{F}(\vec{r}) =
\alpha\frac{\vec{r}}{|r|^3}\\
&\Delta\phi(\vec{r}) = -4\pi\alpha\delta(\vec{r}) 
\end{align*}
\begin{Bemerkung}
In der Elektrostatik hat das Potential eines geladenen Punktteilchens
(Elektron/Proton) ebenfalls das Potential
$\phi(\vec{r})\sim\frac{1}{|\vec{r}|}$. Daher der Name {\em Coulomb Potential}.
\end{Bemerkung}
