\documentclass[a4paper,12pt,halfparskip,fleqn]{scrreprt}

% Sonderzeichen
\usepackage[ngerman]{babel}
\usepackage[utf8x]{inputenc}

% Schriftart Minion Pro (auskommentieren, falls diese fehlt)
%\usepackage[fullfamily,minionint,mathlf]{MinionPro}

% Mathematik Pakete
\usepackage{amsmath,array,stmaryrd}

% Header, Footer, Farben
\usepackage{fancyhdr,color}

% pstricks zum Zeichnen
\usepackage{pstricks-add,pst-pdf}

% ntheorem für Theoremumgebungen
\usepackage[amsmath,thmmarks]{ntheorem}

% Seitenlayout
\pagestyle{fancy} %eigener Seitenstil

\fancyhf{} %alle Kopf- und Fußzeilenfelder bereinigen
\fancyhead[L]{\sf \textbf{\rightmark}} %Kopfzeile links
\fancyhead[C]{} %zentrierte Kopfzeile
\fancyhead[R]{\sf \textbf{}} %Kopfzeile rechts
\renewcommand{\headrulewidth}{0.4pt} %obere Trennlinie
\fancyfoot[L]{{\color{lightgray}\thepage \big|}} %Seitennummer

% Verwendete Operatoren
\DeclareMathOperator\sgn{sgn}
\DeclareMathOperator\arcsinh{arcsinh}
\renewcommand{\div}{\operatorname{div}}
\DeclareMathOperator\rot{rot}

% Verwendete Farben
\definecolor{darkblue}{rgb}{0,0,.5}
\definecolor{darkgreen}{rgb}{0,.5,0}
\definecolor{lightgray}{rgb}{0.7,0.7,0.7}

% Diese Farben zeigt einen Rand um jedes Bild an
% (zur Größenmessung)
%\definecolor{framecolor}{rgb}{0,0,0}
\definecolor{framecolor}{rgb}{1,1,1}

% Theoremumgebungen für Satz, Definition, Beispiel, Bemerkung usw.
\newenvironment{satz}[2]{{\bf\color{darkblue} Satz
#1} {\em #2}\par}{{\color{darkblue}\ensuremath{\rtimes}}}
\theoremstyle{plain}
\newtheorem{Satz}{Satz}[section] 

\theoremstyle{marginbreak}
\theorembodyfont{\upshape}
\theoremheaderfont{\normalfont\bfseries}
\theoremsymbol{\ensuremath{{\color{lightgray}\rtimes}}}
\newtheorem{Definition}[Satz]{Definition}

\theoremstyle{marginbreak}
\theoremseparator{\hfill}
\theoremsymbol{\ensuremath{{\color{lightgray}\triangleleft}}}
\newtheorem{Beispiel}[Satz]{Beispiel}

\theoremstyle{margin}
\theoremsymbol{\ensuremath{{\color{lightgray}\multimap}}}
\newtheorem{Bemerkung}[Satz]{Bemerkung}

\newenvironment{info}{\ensuremath{{\color{lightgray}\looparrowright}}}{\ensuremath{{\color{lightgray}\looparrowleft}}}

\renewcommand\thesubsubsection{\ensuremath{\triangleleft}}

% Kapitelnummer bei Formeln voranstellen
\renewcommand{\theequation}{\thechapter.\arabic{equation}}
\numberwithin{equation}{section}

% und quantoren sollen bitte nicht so gequetscht aussehen
\let\existsorig\exists 
\renewcommand{\exists}{\ \existsorig\ } 
\let\forallorig\forall 
\renewcommand{\forall}{\ \forallorig\ } 

% Komplexe Zahlen, Natürliche Zahlen etc. alle wollen sie ihre eigenen Symbole
% haben
\newcommand{\Co}{\mathbb{C}}  
\newcommand{\N}{\mathbb{N}}
\newcommand{\R}{\mathbb{R}}
\newcommand{\Z}{\mathbb{Z}}
\newcommand{\Q}{\mathbb{Q}}

% auch die Integrale wollen Raum
\newcommand{\dx}{\,dx\,}
\newcommand{\dy}{\,dy\,}
\newcommand{\dz}{\,dz\,}

\newcommand{\du}{\,du\,}
\newcommand{\dv}{\,dv\,}
\newcommand{\dw}{\,dw\,}

\newcommand{\dt}{\,dt\,}
\newcommand{\ds}{\,ds\,}

\newcommand{\dV}{\,dV\,}

\newcommand{\dvecv}{\,d\vec{v}\,}
\newcommand{\dvecs}{\,d\vec{s}\,}
\newcommand{\dvecr}{\,d\vec{r}\,}

\newcommand{\dr}{\,dr\,}
\newcommand{\dphi}{\,d\varphi\,}
\newcommand{\dtheta}{\,d\vartheta\,}

\begin{document}

\markright{\sf \textbf{Mathematische Methoden der
Physik}\hfill\textbf{\thesection}}

\renewcommand\thesection{\arabic{section}}

\title{Mathematische Methoden der Physik}
\author{Jan-Cornelius Molnar}
\date{Revision: \today}

% den Titel einfügen
\maketitle

% und los gehts!

\tableofcontents % erzeugt ein Inhaltsverzeichnis

\newpage
\section{Differentialrechnung}

Ableiten ist der Prozess in dem bestimmt wird, wie sich eine Funktion $f(x)$
ändert unter einer Variation ihres Arguments $x$.


\begin{Beispiel}[Geschwindigkeit eines Massenpunktes]
Bei gleichförmiger Bewegung gilt
\begin{align*}
x(t) = v\cdot t + x_0
\end{align*}
mit der Geschwindigkeit
\begin{align*}
v = \frac{x(t+\Delta t)-x(t)}{\Delta t} \equiv \frac{\Delta
x}{\Delta t}.
\end{align*}

\begin{center}
\psset{unit=1cm}
\begin{pspicture}(-1,-1)(10,6)
 \psline[linewidth=0.5pt,arrowsize=4pt]{->}(-1,0)(10,0)
 \psline[linewidth=0.5pt,arrowsize=4pt]{->}(0,-1)(0,6)
 \psline[linewidth=0.5pt,linecolor=gray,linestyle=dashed](0,4)(10,4)
 \psline[linewidth=0.5pt,linecolor=gray,linestyle=dashed](0,5)(10,5)
 \psplot[linewidth=1.2pt,algebraic=true]{-1}{10}{(1/6)*x+(4-1/3)}
 \psline[linewidth=0.5pt,arrows=-*](2,0)(2,4)
 \psline[linewidth=0.5pt,arrows=-*](8,0)(8,5)
 \rput(-0.36,-0.26){$0$}
 \rput(2,-0.26){$t$}
 \rput(8,-0.26){$t+\Delta t$}
 \rput(9.7,-0.26){$x$}
 \rput(-0.36,4.5){$\Delta x$}
 \rput(-0.36,5.7){$y$}
 \rput(9,5.5){$x(t)$}
\end{pspicture}
\end{center}
\end{Beispiel}

Die Ableitung bestimmt somit das Prinzip, mit dem wir die
Geschwindigkeit eines Massenpunktes auf beliebige Wege $x(t)$ verallgemeinern
können.

Auf einem kleinen Bereich $\Delta x = x_1 - x_0$ ändert sich die Funktion
$f(x)$ um den kleinen Wert
\begin{align*}
\Delta f = f(x_1) - f(x_0).
\end{align*}

\begin{center}
\psset{unit=1cm}
\begin{pspicture}(-1,-1)(10,6)
 \psline[linewidth=0.5pt,arrowsize=4pt]{->}(-1,0)(10,0)
 \psline[linewidth=0.5pt,arrowsize=4pt]{->}(0,-1)(0,6)
 \psline[linewidth=0.5pt,linecolor=gray,linestyle=dashed](0,3.44)(10,3.44)
 \psline[linewidth=0.5pt,linecolor=gray,linestyle=dashed](0,4.56)(10,4.56)
 \psplot[linewidth=1.2pt,algebraic=true]{0}{10}{0.04*x^2+2}
 \psplot[linewidth=1.2pt,algebraic=true]{0}{10}{0.48*(x-6)+3.44}

 \psline[linewidth=0.5pt,arrows=-*](6,0)(6,3.44)
 \psline[linewidth=0.5pt,arrows=-*](8,0)(8,4.56)
 \rput(-0.36,-0.26){$0$}
 \rput(6,-0.26){$x_0$}
 \rput(7,-0.26){$\Delta x$}
 \rput(8,-0.26){$x_1$}
 \rput(9.7,-0.26){$x$}
 \rput(-0.36,4){$\Delta f$}
 \rput(-0.36,5.7){$y$}
 \rput(8.6,5.5){$f(x)$}
\end{pspicture}
\end{center}

Für kleine $\Delta x$ beschreibt somit der Quotient
\begin{align*}
\frac{\Delta f}{\Delta x} = \frac{f(x_1) - f(x_0)}{x_1-x_0}
\end{align*}
das Verhalten der Funktion $f(x)$ im Punkt $x_0$.\\

\begin{Definition}[Ableitung]
Die formale Definition der Ableitung folgt mittels des Grenzwerts.
\begin{equation}
\frac{d}{dx}f(x) \equiv f'(x) = \lim \limits_{\Delta x \to 0}
\frac{f(x+\Delta x) - f(x)}{\Delta x}
\end{equation}
\end{Definition}

Eine Funktion $f(x)$ heißt differenzierbar, wenn dieser Grenzwert existiert. Da
$f'(x)$ wieder Funktion der Variable $x$ ist, können wir höhere Ableitungen
bilden mittels
\begin{equation}
\frac{d^n}{dx^n}f(x) \equiv f^{(n)}(x) = \lim \limits_{\Delta x \to 0}
\frac{f^{(n-1)}(x+\Delta x) - f^{(n-1)}(x)}{\Delta x}
\end{equation}

\begin{Beispiel}[Ableitungen]
\begin{itemize}
\item $\frac{d}{dx}(x^n) = nx^{n-1}$\\
\begin{info}
\begin{align*}
\Delta f &= (x+\Delta x)^n - x^n = \sum \limits_{k}{\binom{n}{k}} x^{n-k}\Delta
x^k -x^n  \\
&= {\binom{n}{k}} \Delta x\,x^{n-1} + \sigma(\Delta x)
\end{align*}
\end{info}
\item $\frac{d}{dx}e^{ax} = ae^{ax}$
\item $\frac{d}{dx}\ln x = \frac{1}{x}$
\item $\frac{d}{dx}\sin x = \cos x$
\item $\frac{d}{dx}\cos x = -\sin x$
\item nicht differenzierbar in $x = 0$
\begin{align*}
f(x) &= |x|
\end{align*}
\begin{align*}
\frac{d}{dx}|x| & = \sgn(x) = \begin{cases}
1 & x > 0\\0 & x = 0\\-1 &  x < 0
\end{cases}
\end{align*}

\begin{center}
\psset{unit=1cm}
\begin{pspicture}(-6,-1)(6,6)
 \psline[linewidth=0.5pt,arrowsize=4pt]{->}(-6,0)(6,0)
 \psline[linewidth=0.5pt,arrowsize=4pt]{->}(0,-1)(0,6)
 \psplot[linewidth=1.2pt,algebraic=true]{-5}{0}{-x}
 \psplot[linewidth=1.2pt,algebraic=true]{0}{5}{x}

 \rput(-0.36,-0.26){$0$}
 \rput(5.7,-0.26){$x$}
 \rput(-0.36,5.7){$y$}
 \end{pspicture}
\end{center}

\begin{align*}
\frac{d^2}{dx^2}|x| &= 2\delta(x) \text{ Dirac $\delta$-Funktion}
\end{align*}
\end{itemize}

\end{Beispiel}


\begin{Definition}[Taylor Reihe]
 Die Ableitungen beschreiben das Verhalten um einen bestimmten Punkt. Daher
können wir eine differenzierbare Funktion in einer kleinen Umgebung 
als {\em Taylor Reihe} approximieren.
\begin{equation}
f(x) = f(x_0) + f'(x_0)(x-x_0) + \frac{f''(x_0)}{2!}(x-x_0)^2 +
\sigma((x-x_0)^3)
\end{equation}
\end{Definition}

\par{\bf Achtung}
Die Approximation kann nicht immer durch Mitnahme höherer Terme beliebig genau
gemacht werden.

\par
Die erste Ableitung beschreibt die Steigung der Kurve während die zweite
Ableitung die Krümmung einer Kurve ergibt. Bei der Bewegung $x(t)$ eines Massenpunktes beschreibt die erste Ableitung die
Geschwindigkeit
\begin{align*}
v = \frac{d}{dt}x(t) \equiv \dot{x}(t),
\end{align*} während die zweite
Ableitung die Beschleunigung ergibt
\begin{align*}
&a = \frac{d^2}{dt^2}x(t) \equiv \ddot{x}(t).\\
&\Rightarrow\;  \text{Newton'sche Gesetz: } m \ddot{x}(t) = F(x,t)
\end{align*}

\subsection{Differentationsregeln}
Die Ableitung ist eine lineare Operation, d.h.
\begin{align*}
&(a\cdot f(x))' = af'(x)\\
&(f(x)+g(x))' = f'(x) + g'(x)\nonumber
\end{align*}
\par{\bf Produktregel}
\begin{equation}
(f(x)\cdot g(x))' = f'(x)g(x) + f(x)g'(x)
\end{equation}
\begin{info}
\begin{align*}
&f(x+\Delta x)g(x+\Delta x) \\
&= (f(x) + f'(x)\Delta x+\sigma(\Delta x^2))(g(x)+g'(x)\Delta x+\sigma(\Delta
x^2)) \\ 
&= f(x)\cdot g(x) + \Delta x(\underbrace{f'(x)g(x)+f(x)g'(x)+\sigma(\Delta
x^2)}_{f(x)g(x)})
\end{align*}
\end{info}
\par{\bf Kettenregel}
Betrachte die Funktion $f(g(x))$. Für die
Ableitung erhalten wir
\begin{equation}
\frac{d}{dx}f(g(x)) = \left( \frac{d}{dg}f(g) \right) \frac{d}{dx}g(x)
\end{equation}
\begin{info}
\begin{align*}
\begin{split}
f(g(x+\Delta x)) \simeq f(g(x)+\underbrace{\Delta xg'(x)}_{\Delta g} \simeq
f(g(x))+\underbrace{f'(g(x))\cdot g'(x)}_{f(g(x))'}\Delta x \\+ \sigma(\Delta
x^2)
\end{split}
\end{align*}
\end{info}
\par{\bf Umkehrfunktion}
\begin{align*}
y &= f(x) \Rightarrow x = f^{-1}(y) \nonumber\\
\frac{d}{dy}f^{-1}(y) &= \frac{1}{\left(\frac{d}{dx}f(x)\right)} =
\frac{1}{f'(f^{-1}(y))}
\end{align*}
\newpage
\section{Integration}
Das Integral $I = \int \limits_{a}^{b} f(x) \dx \equiv \int \limits_{a}^{b} dx f(x)$ kann als Fläche unter der Kurve $f(x)$ verstanden werden.

Die formale Definition folgt ebenfalls aus einem Grenzwertprozess. Dazu wird
das Intervall $a \le x \le b$ in eine große Anzahl von kleinen Intervallen
aufgeteilt
\begin{equation*}
 a = \xi_0 < \xi_1 < \xi_3 < ... < \xi_n = b
\end{equation*}
und dann folgende Summe geformt
\begin{equation}
 S = \sum \limits_{i=0}^{n} f(x_i)(\xi_i-\xi_{i-1}).
\end{equation}

\begin{center}
\psset{unit=1cm}
\begin{pspicture}(-1,-1)(10,6)
 \psline[linewidth=0.5pt,arrowsize=4pt]{->}(-1,0)(10,0)
 \psline[linewidth=0.5pt,arrowsize=4pt]{->}(0,-1)(0,6)

\psplot[linewidth=1.2pt,algebraic=true]{0}{10}{sin(2/3.41*x)+4} 
 \pscustom[fillstyle=solid,fillcolor=lightgray]{
\psplot[linewidth=1.2pt,algebraic=true]{2}{8}{sin(2/3.41*x)+4}
\psline[linestyle=none](8,3)(8,0)
\psline[linestyle=none](2,0)(2,4.922)
}
 
 \rput(-0.36,-0.26){$0$}
 \rput(2,-0.26){$a$}
 \rput(8,-0.26){$b$}
 \rput(9.7,-0.26){$x$}
 \rput(-0.36,5.7){$y$}
 \rput(8.5,4){$f(x)$}
\end{pspicture}
\end{center}

Die Positionen $x_i$ sind beliebig im Intervall $\xi_{i+1} \le x_i \le \xi_i$.
Das (Riemann'sche) Integral erhält man nun im Limes, wenn man die Länge der
Intervalle $\xi_{i-1} \le x \le \xi_i$ gegen Null streben lässt.

\begin{center}
\psset{unit=1cm}
\begin{pspicture}(-1,-1)(10,6)
 \psline[linewidth=0.5pt,arrowsize=4pt]{->}(-1,0)(10,0)
 \psline[linewidth=0.5pt,arrowsize=4pt]{->}(0,-1)(0,6)

\psplot[linewidth=1.2pt,algebraic=true]{0}{10}{sin(2/3.41*x)+4} 

\psframe[fillcolor=lightgray](2,0)(3,5)
\psframe[fillcolor=lightgray](3,0)(4,4.89)
\psframe[fillcolor=lightgray](4,0)(5,4.48)
\psframe[fillcolor=lightgray](5,0)(6,3.92)
\psframe[fillcolor=lightgray](6,0)(7,3.38)
\psframe[fillcolor=lightgray](7,0)(8,3.05)

\psline[linewidth=0.5pt,arrows=-*,linecolor=gray,linestyle=dashed](2.5,-0.6)(2.5,5)
\psline[linewidth=0.5pt,arrows=-*,linecolor=gray,linestyle=dashed](3.5,-0.6)(3.5,4.89)
\psline[linewidth=0.5pt,arrows=-*,linecolor=gray,linestyle=dashed](4.5,-0.6)(4.5,4.48)
\psline[linewidth=0.5pt,arrows=-*,linecolor=gray,linestyle=dashed](5.5,-0.6)(5.5,3.92)
\psline[linewidth=0.5pt,arrows=-*,linecolor=gray,linestyle=dashed](6.5,-0.6)(6.5,3.38)
\psline[linewidth=0.5pt,arrows=-*,linecolor=gray,linestyle=dashed](7.5,-0.6)(7.5,3.05)

 
 \rput(-0.36,-0.26){$0$}
 \rput(1.8,-0.26){$a=\xi_1$}
 \rput(3,-0.26){$\xi_2$}
 \rput(4,-0.26){$\xi_3$}
 \rput(5,-0.26){$\xi_4$}
 \rput(6,-0.26){$\xi_5$}
 \rput(7,-0.26){$\xi_6$}
 \rput(8.2,-0.26){$b=\xi_7$}
 \rput(9.7,-0.26){$x$}
 
 \rput(2.5,-0.8){$x_1$}
 \rput(3.5,-0.8){$x_2$}
 \rput(4.5,-0.8){$x_3$}
 \rput(5.5,-0.8){$x_4$}
 \rput(6.5,-0.8){$x_5$}
 \rput(7.5,-0.8){$x_6$}
 
 \rput(-0.36,5.7){$y$}
 \rput(8.5,4){$f(x)$}
\end{pspicture}
\end{center}

Eine Funktion $f(x)$ heißt integrierbar, wenn dieser Grenzwert existiert. Desweiteren muss er eindeutig sein, d.h., unabhängig von der Wahl von $\xi_i$ und $x_i$.

\begin{Beispiel}
\begin{equation*}
\int \limits_{0}^{b} \dx x = \frac{1}{2}b^2
\end{equation*}
Unterteile das Intervall $0 \le x \le b$ mit $\Delta x = \frac{b}{n}$. Wähle
$x_i = \xi_i$, so gilt für die Summe
\begin{align*}
 S &= \sum \limits_{k=1}^{n} \Delta x \cdot k \cdot \Delta x = b^2 \frac{1}{n^2} \sum \limits_{k=1}^{n} k \\ &= b^2 \frac{1}{n^2} \frac{n(n+1)}{2} = \frac{b^2}{2} \frac{n+1}{n}
\end{align*}
Den Grenzwert $\Delta x \to c$ erhalten wir somit für $n \to \infty$ und das Integral wird zu
\begin{align*}
 I = \int \limits_{a}^{b} \dx x = \lim \limits_{n\to\infty} \frac{b^2}{2} \frac{n+1}{n} = \frac{b^2}{2}
\end{align*}

\end{Beispiel}

\begin{Bemerkung}
Jedes Integral kann geschrieben werden als
\begin{align}
\int \limits_{a}^{b} f(x) = \lim \limits_{n\to \infty} \sum \limits_{i = 1}^{n}
f(x_i)\Delta x,\quad \text{ mit } \Delta x = \frac{b-a}{n} f(x_i)\Delta x.
\end{align}
\end{Bemerkung}
Das Integral hat folgende Eigenschaften
\begin{itemize}
 \item $\int \limits_{a}^{b} \dx 0 = 0$
 \item $\int \limits_{a}^{a} \dx f(x) = 0$
 \item $\int \limits_{a}^{c} \dx f(x) = \int \limits_{a}^{b} \dx f(x) + \int \limits_{b}^{c} \dx f(x)$
 \item $\int \limits_{a}^{b} \dx [f(x) + g(x)] = \int \limits_{a}^{b} \dx f(x) = \int \limits_{a}^{b} \dx g(x)$
 \item $\int \limits_{a}^{\infty} \dx f(x) = \lim \limits_{b\to\infty} \int \limits_{a}^{b} \dx f(x)$
\end{itemize}
Falls $a < b$, so definiert man
\begin{align}
 \int \limits_{a}^{b} \dx f(x) = -\int \limits_{b}^{a} \dx f(x)
\end{align}


\subsection{Stammfunktion}
Ersetzt man die obere Integrationsgrenze $b$ durch $x$, so definiert das Integral eine neue Funktion
\begin{equation}
 F(x) = \int \limits_{a}^{x} \du f(u)
\end{equation}
Diese Funktion lässt sich jetzt differenzieren
\begin{align*}
 F(x+\Delta x) &= \int \limits_{a}^{x+\Delta x} \du f(u) = \int \limits_{a}^{x}
 \du f(u) + \int \limits_{x}^{x+\Delta x} \du f(u) \\ &= F(x) + \int
 \limits_{x}^{x+\Delta x} \du f(u)
\end{align*}
Somit folgt die Ableitung
\begin{align*}
 \frac{d}{dx}F(x) &= \lim \limits_{\Delta x \to 0} \frac{F(x + \Delta x) - F(x)}{\Delta x}\\
 &= \lim \limits_{\Delta x \to 0} \frac{1}{\Delta x} \int
\limits_{x}^{x+\Delta x} \du f(u)\\
 &= \lim \limits_{\Delta x \to 0} \frac{1}{\Delta x} \left[ \Delta x f(x) + \sigma(\Delta x^2)\right] = f(x)
\end{align*}

\begin{center}
\psset{unit=1cm}
\begin{pspicture}(-1,-1)(10,6)
 \psline[linewidth=0.5pt,arrowsize=4pt]{->}(-1,0)(10,0)
 \psline[linewidth=0.5pt,arrowsize=4pt]{->}(0,-1)(0,6)

\psplot[linewidth=1.2pt,algebraic=true]{0}{10}{sin(2/3.41*x)+4}
 \psline[linewidth=0.5pt](0,4.2)(5,4.2)
\psline[linewidth=0.5pt,arrows=-*,linecolor=gray,linestyle=dashed](4.5,0)(4.5,4.48)
\psline[linewidth=0.5pt,arrows=-*,linecolor=gray,linestyle=dashed](5.5,0)(5.5,3.92)
 
 \rput(-0.36,-0.26){$0$}
 \rput(5,-0.26){$\Delta x$}
 \rput(9.7,-0.26){$x$}
 \rput(-0.36,5.7){$y$}
 \rput(6.5,4.4){$I \approx \Delta x f(x)$}
\end{pspicture}
\end{center}


\begin{Definition}[Stammfunktion]
 Jede Funktion $F(x)$ mit
\begin{equation}
\frac{d}{dx} F(x) = f(x)
\end{equation}
heißt {\em Stammfunktion} von $f(x)$ oder {\em unbestimmtes Integral}.
\end{Definition}

Alle Stammfunktionen unterscheiden sich nur durch eine Konstante: $G(x) = F(x) + c$
ist ebenfalls eine Stammfunktion zu $f(x)$.\\
Man schreibt daher für das unbestimmte Integral
\begin{equation}
F(x) = \int \dx f(x)
\end{equation}
Das {\em bestimmte Integral} folgt somit aus der Stammfunktion
\begin{equation}
\int \limits_{a}^{b} \dx f(x) = F(b) - F(a) \equiv F |_{a}^{b}
\end{equation}
Es folgen sogleich eine Reihe elementarer Integrale
\begin{itemize}
  \item $\int \dx x^n = \frac{x^{n+1}}{n+1}$
  \item $\int \dx e^{ax} = \frac{1}{a}e^{ax}$
  \item $\int \dx \frac{a}{x} = a\cdot\ln{|x|} $
  \item $\int \dx \cos{bx} = \frac{1}{b}\sin{bx} $
\end{itemize}

\subsection{Integrationsregeln}

\par{\bf Substitution}
Es sei $y = g(x)$ so gilt
\begin{equation}
\int \dy f(y) = \int \dx g'(x)f(g(x))
\end{equation}

\begin{Bemerkung}
Schreibe $\frac{d}{dx}y = g'(x) \Rightarrow dy = g'(x)dx$
\begin{equation*}
\int \limits_{a}^{b} \dy f(y) = \int \limits_{g^{-1}(a)}^{g^{-1}(b)} \dx
g'(x)f(g(x))
\end{equation*}
\end{Bemerkung}

\par{\bf Partielle Integration}
\begin{equation}
\int \dx f'(x)g(x) = f(x)g(x) - \int \dx f(x)g'(x)
\end{equation}

\begin{Beispiel}
\begin{align*}
\int \dx \frac{1}{\sqrt{1-x^2}} &\stackrel{x = \sin{y}}= \int \dy \cos{y}
\frac{1}{\sqrt{1-\sin^2{y}}} = \int \dy \cos{y} \frac{1}{\cos{y}}\\
 &= y = \arcsin{x}
\end{align*}
\begin{align*}
\int \dx \ln{x} &= \int \dx \underset{f'(x)}{1} \cdot \ln{x} = x\ln{x} -
\int \dx x \cdot \frac{1}{x} \\
&= x\ln{x} - x
\end{align*}
\end{Beispiel}
\newpage
\section{Vektoren}

\begin{Definition}[Skalar]
 Physikalische Größen, die nur von ihrem Betrag abhängen
\begin{itemize}
 \item T: Temperatur
\item p: Druck
\item m: Masse
\item $\alpha = \frac{e^2}{\hbar c} \approx \frac{1}{137}$ : Feinstrukturkonstante
\end{itemize}
\end{Definition}

\begin{Definition}[Vektor]
 Physikalische Größen, die durch ihren Betrag und ihre Richtung bestimmt sind
\begin{itemize}
 \item $\vec v$: Geschwindigkeit
\item $\vec F$: Kraft
\item $\vec E$: elektrisches Feld
\end{itemize}
\end{Definition}

Einen Vektor stellen wir durch einen Pfeil im Raum dar mit seiner Länge bestimmt durch seinen Betrag.
\par{\bf Addition}
\begin{align*}
 \vec a + \vec b = \vec b + \vec a & : \text{ kommutativ}\\
 \vec a + (\vec b + \vec c) = (\vec a + \vec b) + \vec c & : \text{ assoziativ}
\end{align*} 
\par{\bf Skalare Multiplikation} $\lambda \cdot \vec a$:
\begin{align*}
 (\lambda + \mu) \vec a = \lambda \vec a + \mu \vec a\\
\lambda (\vec a + \vec b) = \lambda \vec a + \lambda \vec b
\end{align*} 

\begin{center}
\psset{unit=1cm}
\begin{pspicture}(-6,-1)(6,6)
 \psline[linewidth=0.5pt,arrowsize=4pt]{->}(-5,1)(-1,2)
 \psline[linewidth=0.5pt,arrowsize=4pt]{->}(-5,1)(0,5)
 \psline[linewidth=0.5pt,arrowsize=4pt,linestyle=dashed](-1,2)(0,5)
 \psline[linewidth=0.5pt,arrowsize=4pt]{->}(-5,1)(-4,4)
 \psline[linewidth=0.5pt,arrowsize=4pt,linestyle=dashed](-4,4)(0,5)
 
 \rput(-5,2.5){$\vec{a}$}
 \rput(-3,1){$\vec{b}$}
 \rput(-2,5){$\vec{a}$}
 \rput(0,4){$\vec{b}$}
 \rput(-2.8,3.5){$\vec{a}+\vec{b}$}
 
 \rput(-3,0){Addition}
 
 \psline[linewidth=0.5pt,arrowsize=4pt]{->}(1,1)(3,3)
 \psline[linewidth=0.5pt,arrowsize=4pt,linestyle=dashed]{->}(3,3)(5,5)
 
 \rput(1.5,2.5){$\vec{a}$}
 \rput(3.5,4.5){$2\vec{a}$}
 
 \rput(3,0){Skalare Multiplikation}
 
\end{pspicture}
\end{center}
\subsection{Basisvektoren}
Für 3 beliebige Vektoren, die nicht in einer Ebene liegen, können wir jeden Vektor darstellen als
\begin{align}
 \vec a = a_1\vec e_1 + a_2 \vec e_2 + a_3 \vec e_3
\end{align}

Die drei Vektoren $\vec e_1, \vec e_2, \vec e_3$ formen eine Basis; die Skalare $a_1, a_2, a_3$ heißen Komponenten des Vektors $\vec a$ zu dieser Basis.

\begin{Bemerkung}
 Meistens werden die Basisvektoren orthogonal aufeinander gewählt aber es ist nicht notwendig.
\end{Bemerkung}
Im Allgemeinen gilt
\begin{enumerate}
 \item Die Anzahl Vektoren in der Basis ist bestimmt durch die Dimension des Raumes.
\item Die Vektoren in der Basis sind linear unabhängig, d.h.,
\begin{align*}
 c_1\vec e_1 + c_2\vec e_2 + ... + c_n\vec e_n \neq 0
\end{align*}
für alle $c_i$, außer $c_i = 0$.
\end{enumerate}
Im 3-dimensionalen kartesischen Koordinatensystem $(x,y,z)$ haben wir die natürliche Basis der Einheitsvektoren $\vec e_x, \vec e_y, \vec e_z$ entlang der $x,y,z$ Achsen.

\begin{center}
\psset{unit=1cm}
\begin{pspicture}(-6,-4)(6,6)
 \psline[linewidth=0.5pt,arrowsize=4pt,linestyle=dashed]{->}(0,0)(0,5)
 \psline[linewidth=0.5pt,arrowsize=4pt,linestyle=dashed]{->}(0,0)(5,0)
 \psline[linewidth=0.5pt,arrowsize=4pt,linestyle=dashed]{->}(0,0)(-2.5,-2.5)
 \psline[linewidth=0.5pt,arrowsize=4pt]{->}(0,0)(0,2)
 \psline[linewidth=0.5pt,arrowsize=4pt]{->}(0,0)(2,0)
 \psline[linewidth=0.5pt,arrowsize=4pt]{->}(0,0)(-1,-1)
 
 \psline[linewidth=0.5pt,arrowsize=4pt,linestyle=dashed](0,0)(2,-2)
 \psline[linewidth=0.5pt,arrowsize=4pt,linestyle=dashed](-2,-2)(2,-2)
 \psline[linewidth=0.5pt,arrowsize=4pt,linestyle=dashed](2,-2)(4,0)
 \psline[linewidth=0.5pt,arrowsize=4pt,linestyle=dashed](2,-2)(2,2)
 \psline[linewidth=0.5pt,arrowsize=4pt,linestyle=dashed](0,4)(2,2)
 \psline[linewidth=0.5pt,arrowsize=4pt]{->}(0,0)(2,2)
 
 \rput(-1.2,-0.7){$\vec e_x$} %-1,-1
 \rput(-2.4,-1.9){$a_x$} % -2,-2
 
 \rput(1.7,0.3){$\vec e_y$} %-1,-1
 \rput(4,0.3){$a_y$} % -2,-2
 
 \rput(-0.3,1.9){$\vec e_z$} %-1,-1
 \rput(-0.3,4){$a_z$} % -2,-2
 
 \rput(2.2,2.2){$\vec a$}
 
 
 \rput(-2.7,-2.5){$x$}
 \rput(5,0.3){$y$}
 \rput(-0.3,5){$z$}
  
\end{pspicture}
\end{center}

Für eine gewählte Basis können wir somit jeden Vektor in seinen Komponenten schreiben
\begin{align}
 \vec a = \begin{pmatrix}a_x\\a_y\\a_z\end{pmatrix} \equiv \begin{pmatrix}a_1\\a_2\\a_3\end{pmatrix}
\end{align}
Die Addition und skalare Multiplikation hat in Komponentenschreibweise die Form:
\begin{itemize}
 \item $\vec a + \vec b = \begin{pmatrix}a_1\\a_2\\a_3\end{pmatrix} + \begin{pmatrix}b_1\\b_2\\b_3\end{pmatrix} = \begin{pmatrix}a_1+b_1\\a_2+b_2\\a_3+b_3\end{pmatrix}$
\item $\lambda  \vec a = \lambda \begin{pmatrix}a_1\\a_2\\a_3\end{pmatrix} = \begin{pmatrix}\lambda a_1\\ \lambda a_2\\ \lambda a_3\end{pmatrix}$
\end{itemize}

\subsection{Skalar Produkt und Betrag}
\begin{Definition}[Skalarprodukt]
Das {\em Skalarprodukt} im dreidimensionalen kartesischen Raum ist definiert durch
\begin{align}
 \vec a \cdot \vec b = a_x b_x + a_y b_y + a_z b_z \equiv\,<\vec a|\vec b>.
\end{align}
\end{Definition}
\begin{Definition}[Betrag]
Der {\em Betrag} eines Vektors ist bestimmt durch das Skalarprodukt mit sich selbst.
\begin{align}
 a = |\vec a| = \sqrt{\vec a \cdot \vec a} = \sqrt{a_x^2 + a_y^2 + a_z^2}
\end{align}
\end{Definition}

Für die natürliche Basis $\vec e_x, \vec e_y, \vec e_z$ folgt somit
\begin{align*}
 \vec e_x \cdot \vec e_y = \vec e_x \cdot \vec e_z = \vec e_y \cdot \vec e_z =
 0 & \text{\qquad orthogonal}\\ |\vec e_x| = |\vec e_y| = |\vec e_z| = 1 &
 \text{\qquad normiert}
\end{align*}

\begin{Bemerkung}
Die Basis heißt orthonormiert.
\end{Bemerkung}

Das Skalarprodukt kann auch geschrieben werden als
\begin{align*}
 \vec a \cdot \vec b = |\vec a||\vec b|\cos{\vartheta}
\end{align*}
mit dem Winkel $\vartheta$ zwischen den beiden Vekoren $\vec a$ und $\vec b$.

\begin{center}
\psset{unit=1cm}
\begin{pspicture}(-1,-1)(6,6)
 \psline[linewidth=0.5pt,arrowsize=4pt]{->}(0,0)(5,-.5)
 \psline[linewidth=0.5pt,arrowsize=4pt]{->}(0,0)(3,4)
 
 \psarc[linewidth=.5pt](A){1}{-5}{52}
 
 \rput(3.5,0){$\vec b$}
 \rput(2,3.5){$\vec a$}  
 \rput(0.5,0.2){$\vartheta$}
\end{pspicture}
\end{center}

\subsection{Vektorprodukt}
Das Vektorprodukt wird geschrieben als
\begin{align*}
 \vec c = \vec a \times \vec b
\end{align*}
wobei $\vec c$ orthogonal auf $\vec a$ und $\vec b$ steht, mit dem Betrag
\begin{align*}
 |\vec c| = |\vec a||\vec b|\sin{\vartheta}, \text{\qquad der Fläche des
 Parallelogramms.}
\end{align*}

\begin{center}
\psset{unit=1cm}
\begin{pspicture}(-1,-2)(3,3)
 \psline[linewidth=0.5pt,arrowsize=4pt]{->}(0,0)(2,-1)
 \psline[linewidth=0.5pt,arrowsize=4pt]{->}(0,0)(2,1)
 \psline[linewidth=0.5pt,arrowsize=4pt]{->}(0,0)(0,2)
 
 \psline[linewidth=0.5pt,arrowsize=4pt,linestyle=dashed](2,1)(4,0)
 \psline[linewidth=0.5pt,arrowsize=4pt,linestyle=dashed](2,-1)(4,0)
 
 \psarc[linewidth=.5pt](A){0.9}{-26}{26}
 
 \rput(1.2,-1){$\vec b$}
 \rput(1.2,1){$\vec a$}
 \rput(-0.4,1.2){$\vec c$}  
 \rput(0.6,0){$\vartheta$}
\end{pspicture}
\end{center}

Zudem bilden $\vec a, \vec b, \vec c$ ein rechthändiges Dreibein.\\
Es gelten die Rechenregeln
\begin{align}
          &(\vec a + \vec b) \times \vec c = \vec a \times \vec c + \vec b
          \times \vec c, \nonumber\\ 
          &\vec b \times \vec a = - \vec a \times \vec b.
\end{align}

In den Koordinaten der natürlichen Basis $\vec e_x, \vec e_y, \vec e_z$ hat das Vektorprodukt die Form
\begin{align}
 \vec a \times \vec b = \begin{pmatrix}a_x\\a_y\\a_z\end{pmatrix} \times
 \begin{pmatrix}b_x\\b_y\\b_z\end{pmatrix} = \begin{pmatrix}a_y b_z - a_z b_y\\a_z b_x - a_x b_z\\ a_x b_y - a_y b_x\end{pmatrix} = \vec c.
\end{align}

\begin{Beispiel}
Bewegung eines Massenpunktes: $\vec r(t) =
\begin{pmatrix}x(t)\\y(t)\\z(t)\end{pmatrix}$\\ Geschwindigkeit: $\vec v(t) = \frac{d}{dt} \vec r(t) =
\begin{pmatrix}\dot{x}(t)\\\dot{y}(t)\\\dot{z}(t)\end{pmatrix}$\\
Drehimpuls: $\vec L = m(\vec r \times \vec v)$\\
Lorentz-Kraft: $\vec F = q(\underbrace{\vec E}_{\text{el. Feld}} + \vec v
\times \underbrace{\vec B}_{\text{Magnetfeld}})$\\
Beschleunigung: $\vec a(t) = \frac{d}{dt} \vec v(t) = \frac{d^2}{dt^2}\vec r(t)
= \begin{pmatrix}\ddot{x}(t)\\\ddot{y}(t)\\\ddot{z}(t)\end{pmatrix}$
\end{Beispiel}
\newpage
\section{Komplexe Zahlen} %erzeugt einen Abschnitt mit Nummer

Die Gleichung $z²=-1$ hat keine Lösung in den reellen Zahlen. Man kann jedoch
die reellen Zahlen $\mathbb{R}$ erweitern, sodass diese Gleichung ebenfalls
zwei Lösungen besitzt.\\
Dazu führt man die {\em imaginäre Einheit} ein
\begin{equation}
	i \text{ mit der Definition } i^2=-1
\end{equation}
Die beiden Lösungen obiger Gleichung nehmen die Form an
\begin{equation*}
	z = \pm i
\end{equation*}
Eine Zahl in diesem erweiterten Raum kann somit einen Realteil und einen
Imaginärteil besitzen und heißt komplexe Zahl.
\begin{equation}
	z = \underbrace{x}_{\text{Realteil}} + \underbrace{iy}_{\text{Imaginärteil}}
	x,y
	\in
	\mathbb{R}
\end{equation}
Das Rechnen mit komplexen Zahlen verhält sich gleich wie mit reellen Zahlen:
\begin{eqnarray}
(x+iy)+(u+iv) & = & (x+u)+i(y+v)\\
(x+iy)\cdot(u+iv) & = & xu-yv+i(xv+uy)\nonumber
\end{eqnarray}

Eine komplexe Zahl kann in der komplexen Ebene dargestellt werden.
\begin{align}
\operatorname{Re} z &= x = r \cos(\varphi)&& \text{ : Real-Teil}\\
\operatorname{Im} z &= y = r \sin(\varphi)&& \text{ :
Imaginär-Teil}\nonumber\\
r &= \sqrt{x^2+y^2} \equiv |z|&& \text{ : Betrag von z}\nonumber\\
\varphi &= \arctan \frac{y}{x} \equiv \arg(z)&& \text{ : Argument von
z}\nonumber
\end{align}
\begin{center}
\psset{unit=1cm}
\begin{pspicture}(-1,-1)(10,6)
 \psline[linewidth=0.5pt,arrowsize=4pt]{->}(-1,0)(10,0)
 \psline[linewidth=0.5pt,arrowsize=4pt]{->}(0,-1)(0,6)
 
 \psline[linewidth=0.5pt,arrowsize=4pt,linestyle=dashed](0,4)(6,4)
 \psline[linewidth=0.5pt,arrowsize=4pt,linestyle=dashed](6,0)(6,4)
 
 \psline[linewidth=0.5pt,arrowsize=4pt]{->}(0,0)(6,4)
 
 \psarc[linewidth=.5pt](A){1}{0}{33}
 
 \rput(-0.2,-0.3){$0$}
 \rput(6,-0.3){$x$}
 \rput(-0.2,4){$y$}
 \rput(6.2,4.2){$z$}
 
 \rput(3.2,2.4){$r$}
 
 \rput(0.7,0.2){$\varphi$}
 
 \rput(9.7,-0.26){$\Re z$}
 \rput(-0.36,5.7){$\Im z$}
 
\end{pspicture}
\end{center}

\begin{Bemerkung}
\begin{align}
z_1 \cdot z_2 &=
r_1(\cos\varphi_1+i\sin\varphi_1)
r_2(\cos\varphi_2+i\sin\varphi_2) \nonumber \\ &= r_1 r_2
(\cos\varphi_1 \cos\varphi_2 - \sin\varphi_1 \sin\varphi_2 + i(\cos\varphi_1 \sin\varphi_2 + \sin\varphi_1
\cos\varphi_2)) \nonumber \\ &= r_1 r_2 [\cos(\varphi_1 +
\varphi_2) + i\sin(\varphi_1 + \varphi_2)]
\end{align}
d.h. Multiplikation zweier komplexer Zahlen multipliziert ihren Betrag und
addiert ihr Argument.
\begin{center}
\psset{unit=1cm}
\begin{pspicture}(-6,-1)(6,6)
 \psline[linewidth=0.5pt,arrowsize=4pt]{->}(-6,0)(6,0)
 \psline[linewidth=0.5pt,arrowsize=4pt]{->}(0,-1)(0,6)
 %\psplot[linewidth=1.2pt,algebraic=true]{-5}{0}{-x}
 %\psplot[linewidth=1.2pt,algebraic=true]{0}{5}{x}
 
 \psline[linewidth=1.2pt,arrowsize=4pt]{->}(0,0)(3,2)
 \psline[linewidth=1.2pt,arrowsize=4pt]{->}(0,0)(-1,3)
 \psline[linewidth=1.2pt,arrowsize=4pt]{->}(0,0)(-5,3.89)

\psarc[linewidth=.5pt](A){3}{0}{33.7}
\psarc[linewidth=.5pt](A){2}{0}{108.4}
\psarc[linewidth=.5pt](A){5}{0}{142.1}
 \rput(-1.1,2.5){$r_2$}
 \rput(2.5,2){$r_1$}
 \rput(3.2,2.13){$z_1$}
 \rput(-1.1,3.3){$z_2$}
 \rput(-5.1,4.1){$z_1\cdot z_2$}
 \rput(0.3,2.2){$\varphi_2$}
 \rput(3.2,0.2){$\varphi_1$}
 \rput(-3,3){$\varphi_1+\varphi_2$}
 \rput(-5,3.3){$r_1+r_2$}
 \rput(5.7,-0.26){$\Re z$}
 \rput(-0.36,5.7){$\Im z$}
 \end{pspicture}
\end{center}
\end{Bemerkung}

\subsection{Komplex Konjugierte}
\begin{Definition}[Komplexe Konjugation]
Es gibt eine spezielle Funktion, das {\em Komplex Konjugierte}, das jeder
komplexen Zahl eine neue komplexe Zahl zuordnet mittels
\begin{align*}
z = x+iy \Rightarrow \bar{z} = x-iy.
\end{align*}
\end{Definition}

Es gelten die Relationen
\begin{align}
z+\bar{z} &= 2\operatorname{Re}z\\
z-\bar{z} &= 2\operatorname{Im}z \nonumber\\
z\cdot\bar{z} &= |z|^2 \nonumber
\end{align}
Es gelten die Rechenregeln
\begin{align}
(\overline{z_1 + z_2}) &= \overline{z_1}+\overline{z_2} \\
(\overline{z_1 \cdot z_2}) &= \overline{z_1} \cdot \overline{z_2} \nonumber\\
(\overline{z_1 + iz_2}) &= \overline{z_1}-i\overline{z_2} \nonumber
\end{align}
\begin{Bemerkung}
Die Division von komplexen Zahlen erfolgt am einfachsten mit folgendem Trick
\begin{align*}
\frac{z_1}{z_2} = \frac{x + iy}{u + iv} = \frac{z_1 \cdot \bar{z_2}}{|z_2|^2} =
\frac {(x + iy)(u + iv)}{u^2 + v^2} = \frac{xu + yv + i(xv+yu)}{u^2+v^2}
\end{align*}
\end{Bemerkung}

\subsection{Funktionen komplexer Variablen}

\begin{Beispiel}
\begin{align*}
f(z) = z^2 = (x + iy)^2  = x^2 - y^2 + 2ixy
\end{align*}
\end{Beispiel}
Die Ableitung von komplexen Funktionen folgt analog zur Ableitung reeller
Funktionen
\begin{equation}
\frac{d}{dz}f(z)=\lim \limits_{\delta z \to 0} \frac{f(z+\Delta z) -
f(z)}{\Delta z}
\end{equation}
\begin{Bemerkung}
Falls eine komplexe Funktion differenzierbar in einer kleinen Umgebung ist, so
heißt sie {\em analytisch}.
\end{Bemerkung}
\begin{Beispiel}
\begin{align*}
&\frac{d}{dz} z^n = n \cdot z^{n-1}\\
&\operatorname{Re}z,\operatorname{Im}z,\bar{z}, |z|^2 \text{ sind nicht
differenzierbar}
\end{align*}
\end{Beispiel}

\subsection{Exponentialfunktion}
\begin{Definition}[Exponentialfunktion]
Die exponential Funktion $\exp(z)$ ist definiert mittels der Potenzreihe
\begin{equation}
\exp(z) = \sum \limits_{n = 0}^\infty {1 + z +
\frac{z^2}{2} + \frac{z^3}{3!} + \frac{z^4}{4!} + \ldots}
\end{equation}
\end{Definition}

\begin{Bemerkung}
Die Potenzreihe konvergiert für alle $z \in \mathbb{C}$
\begin{align*}
\exp(1) &\equiv e && \text{ Eulersche Zahl}\\
\exp(x) &\equiv e^x && \text{ für } x \in \mathbb{R}\\
\Rightarrow \exp(z) &\equiv e^z && \text{ Erweiterung von } e^z \text{ auf
komplexe Zahlen}
\end{align*}
\end{Bemerkung}

Die Exponentailfunktion hat folgende Eigenschaften
\begin{align*}
\frac{d}{dz} \exp(z) = \exp(z)
\end{align*}
\begin{info}
\begin{align*}
\frac{d}{dz} \exp(z) &= \frac{d}{dz} \left(\sum \limits_{n = 0}^{\infty}
\frac{1}{n!}z^n \right) 
= \sum \limits_{n=0}^{\infty} \frac{1}{n!} \frac{d}{dz} z^n
= \sum \limits_{n=1}^{\infty} \frac{1}{(n-1)!} z^{n-1} \\
& = \sum \limits_{n=0}^{\infty} \frac{1}{n!} z^{n} = \exp(z)
\end{align*}
\end{info}

\begin{align*}
\exp(z_1) \cdot \exp(z_2) = \exp(z_1 + z_2)
\end{align*}
\begin{info}
\begin{align*}
\begin{split}
\exp(z_1) \cdot \exp(z_2) &= \left( \sum \limits_{n=0}^\infty \frac{1}{n!}
z_1^n \right) \left( \sum \limits_{m=0}^\infty \frac{1}{m!}z_2^m \right) 
= \sum \limits_{n=0 \atop m=0}^\infty \frac{1}{n!m!} z_1^n \cdot z_2^m \\
&= \sum \limits_{l=0}^\infty \frac{1}{l!} (z_1+z_2)^l = \exp(z_1 + z_2)
\end{split}
\end{align*}
\begin{center}
\psset{unit=1cm}
\begin{pspicture}(-1,-1)(10,6)
 \psline[linewidth=0.5pt,arrowsize=4pt]{->}(-1,0)(10,0)
 \psline[linewidth=0.5pt,arrowsize=4pt]{->}(0,-1)(0,6)
 
\psdots(1,1)(2,1)(3,1)(4,1)(5,1)(6,1)(7,1)(8,1)(9,1)
\psdots(1,2)(2,2)(3,2)(4,2)(5,2)(6,2)(7,2)(8,2)(9,2)
\psdots(1,3)(2,3)(3,3)(4,3)(5,3)(6,3)(7,3)(8,3)(9,3)
\psdots(1,4)(2,4)(3,4)(4,4)(5,4)(6,4)(7,4)(8,4)(9,4)
\psdots(1,5)(2,5)(3,5)(4,5)(5,5)(6,5)(7,5)(8,5)(9,5)
 
 \rput(9.7,-0.26){$n$}
 \rput(-0.36,5.7){$m$}
 \rput{-45}(0,1.62){\psframe[framearc=0.5](0,0)(2.3,0.5)}
 \rput(0.5,0.5){$k$}
\end{pspicture}
\end{center}
\end{info}

Von besonderem Interesse ist der Wert von $e^z$ für ein imaginäres $z =
i\varphi$:
\begin{align*}
& e^{i\varphi} \cdot \overline{e^{i\varphi}} = |e^{i\varphi}|^2 = e^0 =
e^{i\varphi} e^{-i\varphi} = 1 \\
\Rightarrow\; & e^{i\varphi} \text{ liegt auf dem
Einheitskreis}
\end{align*}
\begin{center}
\psset{unit=1cm}
\begin{pspicture}(-6,-1)(6,6)
 \psline[linewidth=0.5pt,arrowsize=4pt]{->}(-6,0)(6,0)
 \psline[linewidth=0.5pt,arrowsize=4pt]{->}(0,-1)(0,6)
 %\psplot[linewidth=1.2pt,algebraic=true]{-5}{0}{-x}
 %\psplot[linewidth=1.2pt,algebraic=true]{0}{5}{x}
 
 \psline[linewidth=1.2pt,arrowsize=4pt]{->}(0,0)(3,3)

\psarc[linewidth=.5pt](A){4,23}{0}{180}

\rput(0.5,0.2){$\varphi$}
\rput(3.4,3.4){$\exp z$}
\rput(-0.2,4.6){$1$}
 \rput(5.7,-0.26){$\Re z$}
 \rput(-0.36,5.7){$\Im z$}
 \end{pspicture}
\end{center}

Aus der Definition von $\exp(i\varphi)$ folgt
\begin{align}
e^{i\varphi} = \sum \limits_{n=0}^\infty \frac{(i\varphi)^n}{n!} =
\underbrace{\sum \limits_{k=0}^\infty \frac{(-1)^k}{(2k)!}
\varphi^{2n}}_{\cos{\varphi}} + i \underbrace{\sum
\limits_{l=0}^\infty \frac{(-1)^l}{(2l+1)!} \varphi^{2\ln+1}}_{\sin{\varphi}}.
\end{align}
Daraus ergeben sich die Relationen
\begin{align*}
e^{i\varphi} &= \cos(\varphi) + i\sin(\varphi)\\
\cos(\varphi) &= \frac{e^{i\varphi}+e^{-i\varphi}}{2}\\
\sin(\varphi) &= \frac{e^{i\varphi}-e^{-i\varphi}}{2i} 
\end{align*}

\begin{Bemerkung}
Es gilt die {\em Euler'sche Gleichung}
\begin{align}
e^{i\pi}+1 = 0.
\end{align}
\end{Bemerkung}

Jede komplexe Zahl kann somit geschrieben werden als
\begin{align*}
z = r \cdot e^{i\varphi}.
\end{align*}

\begin{center}
\psset{unit=1cm}
\begin{pspicture}(-1,-1)(10,6)
 \psline[linewidth=0.5pt,arrowsize=4pt]{->}(-1,0)(10,0)
 \psline[linewidth=0.5pt,arrowsize=4pt]{->}(0,-1)(0,6)
 
 \psline[linewidth=0.5pt,arrowsize=4pt,linestyle=dashed](0,4)(6,4)
 \psline[linewidth=0.5pt,arrowsize=4pt,linestyle=dashed](6,0)(6,4)
 
 \psline[linewidth=0.5pt,arrowsize=4pt]{->}(0,0)(6,4)
 
 \psarc[linewidth=.5pt](A){1}{0}{33}
 
 \rput(-0.2,-0.3){$0$}
 \rput(6,-0.3){$x$}
 \rput(-0.2,4){$y$}
 \rput(6.2,4.2){$z$}
 
 \rput(3.2,2.4){$r$}
 
 \rput(0.7,0.2){$\varphi$}
 
 \rput(9.7,-0.26){$\Re z$}
 \rput(-0.36,5.7){$\Im z$}
 
\end{pspicture}
\end{center}
\begin{Beispiel}
\begin{align*}
z^n &= (re^{i\varphi})^n = r^ne^{in\varphi}\\
(e^{i\varphi})^2  &= e^{i2\varphi} = \cos2\varphi + i\sin2\varphi \\
 &= (\cos\varphi + i\sin\varphi)^2  =\underbrace{\cos^2\varphi -
 \sin^2\varphi}_{\cos{2\varphi}} +
 i\,\underbrace{2\sin\varphi\cos\varphi}_{i\sin{2\varphi}} \\
 &= \cos(2\varphi) + i\sin(2\varphi)
\end{align*}
\end{Beispiel}

\begin{Bemerkung}
$\cos{(z)}$ und $\sin{(z)}$ sind nun ebenfalls für komplexe Zahlen definiert. 
Insbesondere gilt
\begin{align*}
\cos ix & = \frac{e^{i\cdot ix}+e^{-i\cdot ix}}{2} = \frac{e^{-x}+e^x}{2} =
\cosh x \\
\sin ix & = \frac{e^{-x}-e^x}{2i} = i\frac{e^x-e^{-x}}{2} = i\sinh x
\end{align*}
\end{Bemerkung}

\begin{Beispiel}
\begin{align*}
\cosh 2x = \cos 2ix = \cos^2 ix - \sin^2 ix = \cosh^2 x + \sinh^2 x
\end{align*}
\end{Beispiel}

\subsection {Logarithmus}
\begin{Definition}[Logarithmus Funktion]
Der Logarithmus ist definiert als Umkehrfunktion von $\exp(z)$:
\begin{align*}
z = e^w \Leftrightarrow w = \ln z
\end{align*} 
\end{Definition}

Schreiben wir $z = re^{i\varphi} = re^{i(\varphi + k\cdot2\pi)}$ für $k \in
\mathbb{Z}$, so gilt
\begin{align*}
\ln(z) = \ln\left( re^{i(\varphi + k\cdot2\pi)} \right) = \ln(r) +
\ln\left( e^{i(\varphi + k\cdot2\pi)} \right) = \ln(r) + i(\varphi +
k\cdot2\pi).
\end{align*}
Somit ist der Logarithmus nicht mehr eindeutig.
\par
Die Exponentialfunktion bildet den Bereich $x \in (-\infty, \infty)$ und $y \in
[0,2\pi)$ bereits auf die gesamte komplexe Ebene ab. Daher wissen wir nicht
mehr, von welchem $k$ wir gestartet sind. Der Logarithmus hat daher einen
Schnitt in der komplexen Ebene, in dem die Funktion nicht stetig ist. Die Lage
des Schnittes kann im Prinzip frei gewählt werden.
\par
Mit Hilfe des Logarithmus' lassen sich jetzt auch beliebige Wurzeln und Potenzen
definieren.
\begin{equation*}
f(z) = z^v = \exp(v \cdot \ln z)
\end{equation*}
Für $v = \frac{1}{2}$ erhalten wir die Winkelfunktion
\begin{equation*}
f(z) = \sqrt{z}.
\end{equation*}
Die Vieldeutigkeit des Logarithmus wird auf die Winkelfunktion vererbt
\begin{equation*}
\sqrt{z} = \sqrt{re^{i\varphi}} = \sqrt{r}\;e^{i\frac{\varphi}{2}+ki\pi} \qquad
k = 0,1
\end{equation*}

Die Vieldeutigkeit kann elegant interpretiert werden durch das Einführen von
{\em Riemannschen Blättern}. Man interpretiert die z-Ebene bestehend aus 2
Riemannschen Blättern: Das erste Blatt wird mittels $\sqrt{z}$ auf die obere
Halbebene in der w-Ebene abgebildet. Das zweite Riemannsche Blatt wird auf die
untere Halbebene abgebildet.
\begin{center}
\psset{unit=1cm}
\begin{pspicture}(-6,-4)(6,4)

 \psframe[fillstyle=solid,fillcolor=lightgray,linestyle=none](-5,-2)(-1,2)
 \psframe[fillstyle=solid,fillcolor=lightgray,linestyle=none](1,0)(5,2)
 
 \psline[linewidth=0.5pt,arrowsize=4pt]{->}(-5.5,0)(-0.5,0)
 \psline[linewidth=0.5pt,arrowsize=4pt]{->}(-3,-2.5)(-3,2.5)
 
 \psline[linewidth=0.5pt,arrowsize=4pt]{->}(0.5,0)(5.5,0)
 \psline[linewidth=0.5pt,arrowsize=4pt]{->}(3,-2.5)(3,2.5)
 
 \psarc{->}(0,0){3}{60}{120}
 \psarc{->}(0,0){3}{240}{300}
 
 \rput(0,3.4){$w^2$}
 \rput(0,-2.6){$\sqrt{w}$}
 
 \rput(-0.8,-0.26){$\Re z$}
 \rput(5.2,-0.26){$\Re z$}
 \rput(-3.4,2.3){$\Im z$}
 \rput(2.6,2.3){$\Im z$}
 
\end{pspicture}
\end{center}
Die beiden Riemannschen Blätter heißen Riemannsche Flächen. Die Wurzel Funktion
ist dann analytisch von der Riemannschen Fläche nach $\mathbb{C}$ (außer
$z=0$). Der Schnitt ist die Kreuzungslinie der beiden Blätter.

\begin{Bemerkung}
Für den $\ln(z)$ ist die Riemannsche Fläche eine Spirale.
\end{Bemerkung}
\begin{Beispiel}
Lösungen von $z^n = 1$ haben die Form $$z = e^{i\frac{2\pi}{n}}$$
\end{Beispiel}
\newpage
\section{Gewöhnliche Differentialgleichungen}
Differentialgleichungen treten in der Physik sehr häufig auf

\begin{Beispiel}[Radioaktiver Zerfall]
Die Anzahl Atome, die in einem Zeitintervall $\Delta t$ zerfallen, ist
proportional zur Anzahl Teilchen und der Rate $\Gamma$.
\begin{align}
&\Delta N = -N(t)\;\Gamma \cdot\Delta t\\
\underset{\Delta t \to 0}{\Rightarrow} & \frac{dN(t)}{dt} = -\Gamma
N(t)\nonumber
\end{align}
Die DG hat die Lösung $N(t) = N_0 e^{-\Gamma t}$ mit $N_0$ als Atomzahl zur
Zeit $t = 0$.
\end{Beispiel}

\begin{Beispiel}[Harmonischer Oszillator]
\begin{align*}
& m\ddot{x}(t) = F = -m\omega^2x\\
\Rightarrow & \ddot{x}(t) + \omega^2 x = 0
\end{align*}
Die DG hat die Lösung $x(t) = Ae^{i\omega t} + Be^{-i\omega t}$. Da $x(t)$
reell ist, muss gelten $A^* = B = \frac{x_0}{2}e^{i\varphi}$ und die
allgemeinste Lösung hat die Form
\begin{equation*}
x(t) = x_0 \cos(\omega t + \varphi)
\end{equation*}
mit $x_0$ und $\varphi$ bestimmt durch die Anfangs-/Randbedingungen.
\begin{center}
\psset{unit=1cm}
\begin{pspicture}(-6,-1)(6,6)
 \psline[linewidth=0.5pt,arrowsize=4pt]{->}(-6,0)(6,0)
 \psline[linewidth=0.5pt,arrowsize=4pt]{->}(0,-1)(0,6)
 
 \psplot[linewidth=1.2pt,algebraic=true]{-3}{3}{(1/2)*x^2}
 \psdot[linewidth=0.1cm,dotsize=0.5cm](2,2.8)
 \psline{->}(1.75,2.4)(1.4,1.7)

 \rput(5.7,-0.26){$x$}
 \rput[l](0.2,5.7){$v(x) = \frac{1}{2}x^2$}
 \end{pspicture}
\end{center}

\end{Beispiel}


\subsection{Lineare Differentialgleichungen 1. Ordnung}
Lineare DG 1. Ordnung haben die Form
\begin{equation}
y' = a(x)y + b(x),
\end{equation}
und benötigen im Allgmeinen eine Anfangsbedingung
\begin{equation}
y(0) = y_0,
\end{equation}
um eine spezielle Lösung zu bestimmen.

\subsubsection{Der triviale Fall}
Ist $a(x) = 0$, so ist die Lösung bestimmt durch das
Integral von $b(x)$.
\begin{equation}
y(x) = \int \limits_{0}^{y} \du b(u) + y_0
\end{equation}

\begin{Beispiel}[Geschwindigkeit eines Teilchens mit zeitabhängiger Kraft]
\begin{equation*}
\dot{v} = \frac{F(t)}{m} \Rightarrow v = \int \limits_{0}^{t} \frac{F(t)}{m} dt
+ v_0
\end{equation*}
\end{Beispiel}

\subsubsection{Der homogene Fall}
Eine homogene Differentialgleichung hat die Form
\begin{equation}
y' = a(x)y.
\end{equation}

\begin{Bemerkung}
 Eine Eigenschaft von linearen homogenen Differentialgleichungen ist, dass für die Lösungen $y_1, y_2$ auch folgende Funktionen Lösungen sind:
\begin{itemize}
 \item $\lambda y_1(x)$
 \item $y_1(x)+y_2(x)$
\end{itemize}
\end{Bemerkung}

Die Lösung der Differentialgleichung finden wir mittels Division von $y$:
\begin{align*}
 & \frac{y'}{y} = \frac{d}{dx} \ln{y} = a(x)\\
\Rightarrow & \ln{y(x)} = \int \limits_{0}^{x} \du a(u) + c \equiv A(x)+c\\
\Rightarrow & y(x) = y_0 e^{\int_0^{x}\du a(u)} = y_0 e^{A(x)}
\end{align*}

\subsubsection{Der inhomogene Fall}
Zusätzlich zur homogenen Gleichung haben wir noch einen Treiber $b(x)$.
\begin{align}
 y' - a(x)y = b(x)
\end{align}
Für zwei Lösungen $y_1(x)$ und $y_2(x)$ gilt, dass
\begin{align*}
 y_1(x)-y_2(x)
\end{align*}
eine Lösung der homogenen Gleichung ist.
\par
\begin{info}
\begin{align*}
&\left[y_1(x) - y_2(x)\right]' - a(x)\left[y_1(x) - y_2(x)\right] \\
= & \underbrace{\left[y_1'(x) - a(x)y_1(x)\right]}_{b(x)} -
\underbrace{\left[y_2'(x) - a(x)y_2(x)\right]}_{b(x)} \\
 = & b(x) - b(x) = 0
\end{align*}
\end{info}

\par
Eine Lösung erhält man mittels Multiplikation der DG mit $e^{-A(x)}$
\begin{align}
\Rightarrow & e^{-A(x)}y' - \underbrace{a(x)e^{-A(x)}}_{(e^{-A(x)})'}y =
b(x)e^{-A(x)}\nonumber\\
\Rightarrow & \frac{d}{dx}\left(ye^{-A(x)}\right) = b(x)e^{-A(x)}\nonumber\\
\Rightarrow & y(x) = e^{A(x)} \underbrace{\int
\limits_{0}^{x} \du b(u)e^{-A(u)}}_{\text{partikuläre Lösung}} +
\underbrace{y_0e^{A(x)}}_{\text{homogene Lösung}}
\label{TrickDgLsg}
\end{align}
\begin{Beispiel}[Beschleunigter Massenpunkt mit linearem Luftwiderstand]
\begin{align*}
m\dot{v} = \underbrace{F}_{\text{Beschleunigung}} - \underbrace{R\cdot
v}_{\text{Luftwiderstand}}\\
v =
\underbrace{\frac{F}{R}}_{v_\infty}(1-e^{-t\frac{R}{m}})
+ v_0 e^{-t\frac{R}{m}}
\end{align*}
\begin{center}
\psset{unit=1cm}
\begin{pspicture}(-1,-1)(10,6)
 \psline[linewidth=0.5pt,arrowsize=4pt]{->}(-1,0)(10,0)
 \psline[linewidth=0.5pt,arrowsize=4pt]{->}(0,-1)(0,6)
 
 \psline[linewidth=0.5pt,arrowsize=4pt,linestyle=dashed](0,4)(10,4)
 \psline[linewidth=0.5pt,arrowsize=4pt,linestyle=dashed](2,0)(2,3.46)
 
 \psplot[linewidth=1.2pt,algebraic=true]{0}{10}{4-4*(2.72)^(-x)}
 
 
 \rput(-0.3,-0.3){$0$}
 \rput(2,-0.3){$m/R$}
 
 \rput[r](-0.2,4){$v_\infty$}
 
 \rput(9.7,-0.26){$t$}
 \rput[r](-0.2,5.7){$v(t)$}
 
\end{pspicture}
\end{center}
\end{Beispiel}

\subsection{Nichtlineare Differentialgleichung 1. Ordnung}
Die allgemeine Form lautet
\begin{align}
 \frac{dy}{dx} =  F(x,y),
\end{align}
und ist im Allgemeinen nicht geschlossen lösbar, außer in Spezialfällen, von
denen wir einige untersuchen wollen.

\subsubsection{Separierbare Differentialgleichungen}
Falls $F(x,y)$ separierbar ist, d.h.
\begin{align*}
 F(x,y) = f(x) g(y),
\end{align*}
so können wir die Differentialgleichung umschreiben auf
\begin{align*}
 & \frac{y'}{g(y)} = f(x)\\
  \overset{\text{integrieren mit }x}{\Rightarrow} &
\int \dx \frac{y'}{g(y)} = \underbrace{\int \dy \frac{1}{g(y)}}_{G(y)} = \underbrace{\int \dx f(x)}_{H(x)+c}
\end{align*}
Falls $G(y)$ invertierbar ist, erhalten wir
\begin{align*}
 y(x) = G^{-1}(H(x)+c)
\end{align*}

\begin{Beispiel}
Kettengleichung: Freihängendes Seil/Kette
\begin{align*}
y'' &= \alpha \sqrt{1+(y')^2}
\end{align*}
\begin{center}
%\psset{unit=1cm}
\begin{pspicture}(-6,-1)(6,6)
 \psline[linewidth=0.5pt,arrowsize=4pt]{->}(-6,0)(6,0)
 \psline[linewidth=0.5pt,arrowsize=4pt]{->}(0,-1)(0,6)
 
 \psplot[linewidth=1.2pt,algebraic=true]{-2.2}{0}{-0.2*x^3+2}
 \psplot[linewidth=1.2pt,algebraic=true]{0}{2.2}{0.2*x^3+2}
 \psline[linewidth=0.5pt,algebraic=true]{<->}(1,1.2)(2.5,4.8)
 \psline[linewidth=0.5pt,arrowsize=4pt,linestyle=dashed]{->}(2,3.6)(3,3.6)
 \psline[linewidth=0.5pt,arrowsize=4pt,linestyle=dashed]{->}(2,2.6)(2,4.6)
 \psline[linewidth=0.5pt,arrowsize=4pt,linestyle=dashed]{->}(2,2.6)(2,4.6)
 \psline[linewidth=0.5pt,arrowsize=4pt,linestyle=dashed](1,2.6)(2,2.6)
 \psline[linewidth=0.5pt,arrowsize=4pt,linestyle=dashed](1,2.6)(1,1.2)
 
 \rput(5.7,-0.26){$x$}
 
 \rput[r](1.9,4.6){$F_g$}
 \rput[l](3.1,3.8){$F_\perp$}
 \rput[l](1,2.9){$F_\perp$}
 \rput[r](0.9,1.4){$F_0$}
 \rput[l](2.05,2.8){$\dy$}
 \rput[lt](1.5,2.5){$\dx$}
 
 \rput[l](0.2,5.7){$v(x) = \frac{1}{2}x^100$}
 \end{pspicture}
\end{center}
Setze $u=y'$
\begin{align*}
& \frac{u'}{\sqrt{1+u^2}} = \alpha\\
\Rightarrow & \int du \frac{1}{\sqrt{1+u^2}} = \int \dz
\frac{\cosh(z)}{\sqrt{1+\sinh^2(z)}} = \int \dz \frac{\cosh(z)}{\cosh(z)} \\
 & = z = 
\arcsinh{u} = \int \dx \alpha = \alpha x + c_1\\ \Rightarrow & y' = u =
\sinh(\alpha x + c_1)\\ \Rightarrow & y = \frac{1}{\alpha} \cosh(\alpha x + c_1)
+
\tilde{c}
\end{align*}
\end{Beispiel}

\par{\bf Bernoulli-Gleichung}
\begin{equation*}
y' = a(x)y + b(x)y^{\nu}
\end{equation*}
Mit einer geschickten Substitution kann diese Gleichung auf die Form einer
inhomogenen Differentialgleichung gebracht werden.
\begin{align*}
\frac{y'}{y^{\nu}} = a(x)y^{1-\nu}+b(x)\\
\frac{1}{1-\nu}\left(y^{1-\nu}\right)' = \frac{v'}{1-\nu}
\end{align*}
Setze $y^{1-\nu} = v$ und man erhält
\begin{equation*}
v' = (1-\nu)a(x)v+(1-\nu)b(x)
\end{equation*}
$\Rightarrow$ Löse mit dem Trick von Gleichung \ref{TrickDgLsg}


\subsection{Lineare Differentialgleichungen höherer Ordnung}
Die allgemeine Form ist
\begin{align}
 \sum \limits_{i=0}^{n} a_i(x)y^{(i)}(x) = b(x).
\end{align}
Die Lösung der Differentialgleichung ist eindeutig bestimmt durch das Anfangswertproblem
\begin{align}
 y(0) = y_0, y^{(1)}(0) = y_0^{(1)}, ..., y^{(n-1)}(0) = y_0^{(n-1)}
\end{align}

\begin{Beispiel}[Differentialgleichung 2. Ordnung]
\begin{align}
 \underbrace{L}_{\text{Induktion}} \ddot{I} + \underbrace{R}_{\text{Widerstand}}\dot{I} + \frac{1}{\underbrace{C}_{\text{Kapazität}}}I = \underbrace{\frac{dV}{dt}}_{\text{ext. Drive}} \; \text{ : elektrischer Schwingkreis}
\end{align}
ist durch die Vorgabe von $I(0)$ und $\dot{I}(0)$ bestimmt.
\end{Beispiel}

Allgemeiner Lösungsansatz
\begin{itemize}
  \item Finde $n$ unabhängige Lösungen $y_1(x), ..., y_n(x)$ für die homogene
  Gleichung mit $b(x) = 0$.
  \begin{align*}
  \Rightarrow y_c(x) = C_1\,y_1(x)+ C_2\,y_2(x) + ... + C_n\,y_n(x)
  \end{align*}
  Lösung der homogenen Gleichung.
  \item Finde eine partikuläre Lösung $y_p(x)$ der inhomogenen Gleichung mit
  $b(x) \neq 0$.
  \item Die allgemeine Lösung hat die Form
  \begin{align}
  y(x) = y_c(x) + y_p(x)
  \end{align}
\end{itemize}

\subsubsection{Konstante Koeffizienten}
{\bf Homogene Gleichung}
\begin{align}
 a_n \frac{d^ny}{dx^n} + a_{n-1} \frac{d^{n-1}y}{dx^{n-1}} + ... + a_1\frac{dy}{dx} + a_0y = 0
\end{align}

Mit dem Ansatz $y = e^{\lambda x}$ mit $\lambda \in \Co$ können wir die Differentialgleichung in eine algebraische Gleichung umwandeln:
\begin{align*}
 a_n\lambda^n + a_{n-1}\lambda^{n-1} + \ldots + a_1\lambda + a_0 = 0
\end{align*}

Die Gleichung hat genau $n$ Lösungen $\lambda_1 \ldots \lambda_n$. Es ist jetzt nötig, 3-Fälle zu unterscheiden:
\begin{enumerate}
  \item Alle $\lambda_i$ sind reell und verschieden. Somit sind $y_i(x) =
  e^{\lambda ix}$ linear unabhängig und die Lösung hat die Form
  \begin{align}
  y_c(x) = C_1e^{\lambda_1 x}+ C_2e^{\lambda_2 x}+\ldots+ C_ne^{\lambda_n x}
  \end{align}
  \item Einige $\lambda_i$ sind komplex. Falls $a_i$ reell sind, so ist
  $\overline{\lambda_i}$ ebenfalls eine Lösung. So können wir schreiben
  $(\lambda_i = \alpha+i\beta)$
  \begin{align}
  y_i(x) &= C_i e^{(\alpha + i\beta)x} + \overline{C_i} e^{(\alpha - i\beta)x}\\
  &= 2A e^{\alpha x}\cos{(\beta x + \varphi)}
  \end{align}
  \item Einige Lösungen sind mehrfache Nullstellen $\lambda_1 = \lambda_2 =
  \ldots = \lambda_k \equiv \lambda$. Dann sieht man, dass die Funktionen
  \begin{align*}
  e^{\lambda x}, xe^{\lambda x}, x^2e^{\lambda x}, \ldots, x^{k-1}e^{\lambda x}
  \end{align*}
  ebenfalls Lösungen sind. Wir erhalten somit wieder $n$ unabhängige Lösungen.
  \begin{align}
  y_c = (C_1+C_2x+\ldots+C_kx^{k-1})e^{\lambda x} +
  C_{k+1}e^{\lambda_{k+1}x}+\ldots+C_ne^{\lambda_nx}
 \end{align}
\end{enumerate}

\begin{Beispiel}
\begin{align*}
&y''(x) - 2y' +y = 0\quad \text{ Ansatz } y = e^{\lambda x} \\
\Rightarrow & (\lambda^2 -2\lambda +1)e^{\lambda x} = 0\\
\Rightarrow & 0 = \lambda^2 -2\lambda +1 = (\lambda -1)^2\\
\Rightarrow & e^x, xe^x \text{ sind Lösungen}
\end{align*}
\par
\begin{info}
$(xe^x)'' - 2(xe^x)' + xe^x = xe^x+2e^x-2(xe^x+e^x)+xe^x = 0$
\end{info}
\par
Die Lösung ist somit $y_c(x) = (C_1+xC_2)e^x$.
\end{Beispiel}

{\bf Inhomogene Gleichung}
\begin{align}
 a_n\frac{d^ny}{dx^n} + \ldots + a_1\frac{dy}{dx} + a_0 = b(x)
\end{align}

Es gibt keine allgemeine Methode, die eine partikulär Lösung liefert. Für
spezielle $b(x)$ helfen aber folgende Ansätze.
\begin{enumerate}
  \item $b(x) = Ae^{\Gamma x}$ mit $\Gamma$ reell oder komplex\\
  Ansatz: $y_p(x) = Be^{\Gamma x}$
  \item $b(x) = A_1\sin(\Gamma x) + A_2\cos(\Gamma x)$\\
  Ansatz: $y_p(x) = B_1\sin(\Gamma x) + B_2\cos(\Gamma x)$
  \item $b(x) = A_0 + A_1x+\ldots+A_nx^n$\\
  Ansatz: $y_p(x) = B_0+B_1x+\ldots+B_nx^n$
  \item Falls $b(x)$ eine Summe oder ein Produkt von obigen Formen ist, so ist der Ansatz ebenfalls eine Summe oder Produkt der entsprechenden Ansätze.
\end{enumerate}

\begin{Beispiel}
 \begin{align*}
  & y'' + y = \cos{2x}\\
&\text{Ansatz: } y_p(x) = B_1\cos{2x} + B_2\sin{2x}\\
\Rightarrow & -B_1 4\cos{2x} - B_2 4\sin{2x} + B_1\cos{2x} + B_2\sin{2x} = \cos{2x}
 \end{align*}
\begin{align*}
 \Rightarrow & -4B_1 +B_1 = 1 & \Rightarrow  & B_1 = -\frac{1}{3}\\
&-4B_2 +B_2 = 0 & & B_2 = 0\\
\end{align*}
\begin{align*}
 \Rightarrow & y_p(x) = -\frac{1}{3}\cos{2x} & \text{ : partikulär Lösung}\\
 \Rightarrow & y_c(x) = C_1\cos{x}+C_2\sin{x} & \text{ : homogene Lösung}
\end{align*}
\begin{align*}
 \Rightarrow & \text{Vollständige Lösung:}\\
  & y(x) = y_p(x)+y_c(x) = -\frac{1}{3}\cos{2x}+C_1\cos{x}+C_2\sin{x}
\end{align*}
\end{Beispiel}

\begin{Bemerkung}
 Die allgemeine Lösung folgt aus der Summe der partikulär Lösung $y_p$ und der
 Lösung der homogenen Gleichung $y_c$.
\begin{align}
 y(x) = y_p(x) + y_c(x)
\end{align}
\end{Bemerkung}

\subsection{Green'sche Funktion}
Betrachte die inhomogene Differentialgleichung
\begin{align}
\frac{d^2}{dx^2}y + a\frac{d}{dx}y + by = f(x),
\end{align}
für einen allgemeinen Treiber $f(x)$. Die Lösung des homogenen Problems kennen
wir.

\par{\bf Frage} Gibt es einen speziellen Treiber $h(x,z)$, sodass wir mit einer
Lösung von
\begin{align}
\frac{d^2}{dx^2}G(x,z) + a\frac{d}{dx}G(x,z) + bG(x,z) = h(x,z)
\end{align}
eine Lösung zur obigen Gleichung finden für einen beliebigen Treiber $f(x)$:
\begin{align}
y(x) = \int \limits_{-\infty}^{\infty} \dz G(x,z)f(z)
\end{align}

\begin{Bemerkung}
Der Ansatz $y(x) = \int_{-\infty}^{\infty} \dz G(x,z)f(z)$ folgt aus der
Linearität der Differentialgleichung:
\begin{itemize}
  \item Ist $y(x)$ Lösung zum Treiber $f(x)$\\
  $\Rightarrow cy(x)$ ist Lösung zum Treiber $cf(x)$
  \item Ist $y_1(x)$ Lösung zu $f(x)$ und $y_2(x)$ Lösung zu $g(x)$\\
  $\Rightarrow y_1(x)+ y_2(x)$ ist Lösung zum Treiber $f(x) + g(x)$
\end{itemize}
\end{Bemerkung}

Einsetzen von $y(x) = \int \limits_{-\infty}^{\infty}\dz G(x,z)f(z)$ in die
Differentialgleichung liefert
\begin{align}
f(x) &= \frac{d^2}{dx^2}y(x) + a\frac{d}{dx}y(x) + by(x) \\
&= \int
\limits_{-\infty}^{\infty}\dz f(z)\left[
\underbrace{\frac{d^2}{dx^2}G(x,z)+
a\frac{d}{dx}G(x,z) + bG(x,z)}_{h(x,z)} \right]\\
&= \int \limits_{-\infty}^{\infty} \dz f(z) h(x,z)
\end{align}
d.h., der spezielle Treiber $h(x,z)$ muss die Eigenschaft haben
\begin{align}
\int \limits_{-\infty}^{\infty}f(z)h(x,z) = f(x)
\end{align}

Die Funktion, die diese Eigenschaft besitzt wird in der Physik als {\em
Dirac-Delta Funktion} bezeichnet, mit der Notation $h(x,z)\equiv \delta(x-z)$
und ist mathematisch gesehen eine {\em Distribution} oder uneigentliche
Funktion. Als nächstes untersuchen wir diese Funktion im Detail.

\subsubsection{Dirac-$\delta$-Funktion}

Die wichtigste Eigenschaft der $\delta$-Funktion ist
\begin{align}
\int \limits_{-\infty}^{\infty} \dx \delta(x-x_0) f(x) = f(x_0),
\end{align}
für alle Funktionen $f(x)$.
\begin{info}
$f(x)$ soll beliebig oft differenzierbar sein.
\end{info}

\par
Wir können die $\delta$-Funktion beliebig genau approximieren mit der Funktion
\begin{align}
h_{\varepsilon}(x) = \frac{1}{\sqrt{\varepsilon\pi}}e^{-\frac{x^2}{\varepsilon}}
\end{align}
\begin{center}
\psset{unit=1cm}
\begin{pspicture}(-1,-1)(10,6)
 \psaxes[labels=none,ticks=none]{->}(0,0)(-0.5,-0.5)(8,5)[\textbf{$x$},-90][\textbf{$y$},0]
 \psplot[linewidth=1.2pt,algebraic=true,plotpoints=3600]{0}{8}{8/(3.41)^0.5*(2.72)^(-(x-3)^2)}

 \psxTick(3){$x_0$}
 \psxTick(2){$x_0-\frac{\varepsilon}{2}$}
 \psxTick(4){$x_0+\frac{\varepsilon}{2}$}
 
 \rput[l](3.6,4.2){$h_\varepsilon(x-x_0)$}
\end{pspicture}
\end{center}
Für kleine $\varepsilon$ gilt somit
\begin{align*}
&\int \limits_{-\infty}^{\infty} \dx h_{\varepsilon}(x-x_0)f(x) \\
= &\int
\limits_{-\infty}^{\infty} \dx h_{\varepsilon}(x-x_0)\left[
f(x_0) + f'(x_0)(x-x_0) + \frac{f''(x_0)}{2}(x-x_0)^2 + \sigma(x-xo) \right]
\end{align*}
Es gilt
\begin{itemize}
  \item
\begin{align*}
&\int \limits_{-\infty}^{\infty} \dx h_{\varepsilon}(x-x_0)f(x_0) = f(x_0) \int
\limits_{-\infty}^{\infty}\dx
\frac{1}{\sqrt{\varepsilon\pi}}e^{\frac{-(x-x_0)^2}{\varepsilon}}\\
&\overset{z = \frac{x-x_0}{\sqrt{\varepsilon}}}{=} f(x_0)
\underbrace{\int \limits_{-\infty}^{\infty}\dz \frac{1}{\sqrt{\pi}}
e^{-z^2}}_{1} = f(x_0)
\end{align*}

\item
\begin{align*}
&\int \limits_{-\infty}^{\infty}h_{\varepsilon}(x-x_0)f'(x_0)(x-x_0) \\
&\overset{z=x-x_0}{=} f'(x_0)\int \limits_{-\infty}^{\infty}\dz
\frac{1}{\sqrt{\varepsilon\pi}}
\underbrace{e^{-\frac{z^2}{\epsilon}}}_{\text{punktsymmetrisch}} z = 0
\end{align*}

\item
\begin{align*}
&\int \limits_{-\infty}^{\infty}
h_{\varepsilon}(x-x_0)\frac{1}{2}f''(x_0)(x-x_0)\\
&\overset{z=\frac{x-x_0}{\sqrt{\varepsilon}}}{=} \frac{1}{2} f''(x_0) \int
\limits_{-\infty}^{\infty}\dz \varepsilon \frac{1}{\sqrt{\pi}} z^2e^{-z^2}\\
&= \frac{1}{4}f''(x_0)\cdot\varepsilon\quad 
\underset{\varepsilon\to0}{\rightarrow} 0
\end{align*}
\end{itemize}
Somit folgt
\begin{align}
\int \limits_{-\infty}^{\infty}\dx h_{\varepsilon}(x-x_0)f(x) = f(x_0) +
\frac{1}{4} f''(x_0)\varepsilon + \sigma(\varepsilon^2) \underset{\varepsilon\to0}{\rightarrow} f(x_0)
\end{align}

Die Frage ist somit, ob man das Integral mit dem Limes vertauschen kann
\begin{align*}
\int \limits_{-\infty}^{\infty}\dx \delta(x-x_0) f(x) &= f(x_0) =
\lim_{\varepsilon\to0} \int \limits_{-\infty}^{\infty}\dx h_{\varepsilon}(x-x_0)
f(x)\\
&''='' \int \limits_{-\infty}^{\infty} \dx \lim_{\varepsilon\to0}
h_{\varepsilon}(x-x_0)f(x)
\end{align*}

\begin{align*}
\Rightarrow \delta(x-x_0) &''='' \lim_{\varepsilon\to0} h_{\varepsilon}(x-x_0)
=
\lim_{\varepsilon\to0}
\frac{1}{\sqrt{\pi\varepsilon}} e^{\frac{-(x-x_0)^2}{\varepsilon}}\\
& = \begin{cases}0 & x\neq x_0\\\infty & x = x_0\end{cases}
\end{align*}

Wir sehen, dass die $\delta$-Funktion keine eigentliche Funktion ist sondern,
dass sie erst Sinn ergibt, indem man über sie integriert. Trotzdem ist es
möglich, mit ihr als abstraktes Objekt zu rechnen.
\par
Eigenschaften der $\delta$-Funktion
\begin{itemize}
  \item $\int_{-\infty}^{\infty}\dx\delta(x) = 1$
  \item $\int_{-\eta}^{\eta}\dx f(x)\delta(x) = f(0)$ für beliebige $\eta > 0$
  \par
  \begin{info}
\begin{align*}
&\int \limits_{-\eta}^{\eta}\dx f(x)\delta(x) = \lim_{\varepsilon\to0} \int
\limits_{-\eta}^{\eta}\dx h_{\varepsilon}(x)f(x) \\
&= \lim_{\varepsilon\to0}\left[\underbrace{\int \limits_{-\infty}^{\infty}\dx
h_{\varepsilon}(x)f(x)}_{f(0)} - \underbrace{\int \limits_{-\infty}^{-\eta}\dx
h_{\varepsilon}(x)f(x)}_{=0} - \underbrace{\int \limits_{\eta}^{\infty}\dx
h_{\varepsilon}(x)f(x)}_{=0}\right]\\
&= f(0)
\end{align*}
\end{info}
\item $\delta(x) = \delta(-x)$\\
\begin{info}
$\int_{-\infty}^{\infty} \dx \delta(x)f(x) = f(0) = \int_{-\infty}^{\infty}
\dx \delta(x)f(-x) = \int_{-\infty}^{\infty} \dx \delta(-x)f(x)$
\end{info}
\item $\delta(x)\cdot x = 0$\\
\begin{info} $\int_{-\infty}^{\infty} \dx f(x)\,x\,\delta(x) = 
\int_{-\infty}^{\infty} \dx g(x)\,\delta(x) = g(0) = f(0)\cdot 0 = 0$ 
\end{info}
\end{itemize}

\begin{Beispiel}
\par
\begin{itemize}
  \item $\int_{-\infty}^{\infty} \dx f(x)[\delta(x-1)+\delta(x+1)] = f(1)+f(-1)$
  \item $\int_{a}^{b} \dx f(x) \delta(x-c) = \begin{cases}f(c) & a
< c < b\\0 & c < a \text{ oder } c > b\end{cases}$
\item $\int_{-1}^{2}\dx x^2 \delta(x-1) = 1$
\item $\int_{-1}^{2}\dx x^2\delta(x+2) = 0$
\item $\int_{-\infty}^{\infty}\dx e^{-x} \delta(x-y^2) = e^{-y^2}$
\end{itemize}
\end{Beispiel}

Alternative Funktionsfolgen, die gegen die $\delta$-Funktion konvergieren
\begin{itemize}
  \item $\frac{1}{\pi}\sin{\frac{Nx}{x}} \qquad \text{ für } N \to \infty$
  \item $\frac{1}{\pi}\frac{\varepsilon}{\varepsilon^2 + x^2} \qquad \text{ für
  } \varepsilon \to 0$
\end{itemize}
\begin{center}
\begin{pspicture}(-3,-1)(3,6)
 \psset{xunit=.25,yunit=2} 
 
 \psaxes[labels=none,ticks=none]{->}(0,0)(-11,-0.25)(11,2)[\textbf{$x$},-90][\textbf{$y$},0]
 \psplot[linewidth=1.2pt,algebraic=true,plotpoints=3600]{-10}{10}{sin(x)/x} 
 \psyTick(1){$N$}
 \rput[l](2,1){$\frac{1}{\pi}\sin\left(\frac{Nx}{x}\right)$}
\end{pspicture}
\begin{pspicture}(-3,-1)(3,6)
 \psset{xunit=.25,yunit=2} 

 \psaxes[labels=none,ticks=none]{->}(0,0)(-11,-0.25)(11,2)[\textbf{$x$},-90][\textbf{$y$},0]
 \psplot[linewidth=1.2pt,algebraic=true,plotpoints=3600]{-10}{10}{1/(x^2+1)}
 \psyTick(1){$\frac{1}{\varepsilon}$}
 \rput[l](2,1){$\frac{1}{\pi}\frac{\varepsilon}{\varepsilon^2+x^2}$}
\end{pspicture}
\end{center}

\begin{Bemerkung}
Mathematisch ist die $\delta$-Funktion eine Distribution.
\begin{itemize}
  \item Mit S bezeichnen wir den {\em Schwartz-Raum}. Für $f
  \in S$ gilt, $f \in C^{\infty}$ und $x^{\alpha}f(x)
  \underset{x\to\pm\infty}{\to} 0\quad \alpha \in \N_0$
  \item Distributionen sind stetige lineare Abbildungen vom Schwartz-Raum in
  die Menge der reellen Zahlen.
  \begin{align*}
  &T: f\in S \rightarrow \R\\
  &\text{mit } T(\alpha f + \beta g) = \alpha T(f) + \beta T(g).
  \end{align*}
\end{itemize}
\end{Bemerkung}

\begin{Beispiel}
\begin{itemize}
  \item Für eine Funktion $h(x)$ erhalten wir die Distribution
  $$H(f) \equiv \int \dx h(x)f(x)$$
  \item Die Abbildung
  $$f(x) \mapsto f(x_0)$$
  ist gerade die $\delta$-Funktion.
  \item Weitere Distributionen sind auch
  $$f(x) \mapsto f'(x_0)$$
  oft als $\delta'(x-x_0)$ bezeichnet.
\end{itemize}
\end{Beispiel}

\subsubsection{Green'sche Funktion für den Oszillator}
Die Differentialgleichung für die Green'sche Funktion lautet
\begin{align}
\frac{d^2}{dx^2}G(x,z) + a\frac{d}{dx}G(x,z) + bG(x,z) = \delta(x-z)
\end{align}
Integrieren wir diese Gleichung um eine kleine Umgebung
$\int_{z-\varepsilon}^{z+\varepsilon} \dx$ erhalten wir
\begin{align*}
&\int \limits_{z-\varepsilon}^{z+\varepsilon} \dx \frac{d^2}{dx^2}G(x,z) + a\int
\limits_{z-\varepsilon}^{z+\varepsilon} \frac{d}{dx}G(x,z) + b \int
\limits_{z-\varepsilon}^{z+\varepsilon} \dx G(x,z)  = 1 \\
\Leftrightarrow &
\left[\frac{d}{dx} G(x,z)\right]_{z-\varepsilon}^{z+\varepsilon}
+ \left[G(x,z)\right]_{z-\varepsilon}^{z+\varepsilon}
+ G(z,z)\varepsilon
\end{align*}
Die Funktion $G(x,z)$ ist aber stetig bei $x = z$ und nur die Ableitung hat
einen Sprung.
\begin{center}
\begin{pspicture}(-1,-1)(5.5,5)
 \psaxes[labels=none,ticks=none]{->}(0,0)(-0.5,-0.5)(5,4)[$x$,-90][\textbf{
 $G(x,z)$},0]
 \psplot[linewidth=1.2pt,algebraic=true,plotpoints=3600]{0}{2.5}{0.1*x^2+2}
 \psplot[linewidth=1.2pt,algebraic=true,plotpoints=3600]{2.5}{5}{0.1*(x-5)^2+2}
 
 \psxTick(2.5){$z$}
\end{pspicture}
\end{center}
Somit gilt für $\varepsilon \to 0$
\begin{align}
  \left[\frac{d}{dx} G(x,z)\right]_{x-\varepsilon}^{x+\varepsilon} =
  G'(z+\varepsilon,z) - G'(z-\varepsilon,z) = 1
\end{align}
Da aber $\delta(x-z) = 0$ für $x\neq z$ gilt
\begin{align}
\frac{d^2}{dx^2} G(x,z) + a \frac{d}{dx} G(x,z) + bG(x,z) = 0 \quad x\neq z
\end{align}
und $G(x,z)$ kann konstruiert werden aus der Lösung der homogenen Gleichung mit
obiger Bedingung für $x=z$.

\begin{Beispiel}
Green'sche Funktion zur Differentialgleichung für $x\in[0,\frac{\pi}{2}]$ mit
\begin{align*}
\frac{d^2}{dx^2} G(x,z) + G(x,z) = \delta(x-z) \quad G(0,z) = 0 \quad
G(\frac{\pi}{2}, z) = 0
\end{align*}
\begin{align*}
\text{Ansatz: } &G(x,z) = A(z)\sin{x} & x < z\\
 & G(x,z) = B(z)\cos{x} & x > z
\end{align*}
Somit folgt:
\begin{align*}
\text{ Stetigkeit: } & A(z)\sin{z} - B(z)\cos{z} = 0\\
\text{Springen in Ableitung: } & -A(z)\cos{z} + B(z)\sin{z} = -1
\end{align*}
\begin{align*}
&A(z) = -\cos{z}\\
&B(z) = -\sin{z}\\
\end{align*}
Die Green'sche Funktion hat die Form
\begin{align*}
G(x,z) = \begin{cases} -\sin{z}\cos{x} & x\ge z\\ -\cos{z}\sin{x} & x<z\end{cases}
\end{align*}
\begin{center}
\begin{pspicture}(-1,-2.5)(5.5,3)
 \psaxes[labels=none,ticks=none]{->}(0,0)(-0.5,-2)(5,2)[$x$,-90][\textbf{
 $\frac{\partial}{\partial x}G(x,z)$},0]
 \psplot[linewidth=1.2pt,algebraic=true,plotpoints=3600]{0}{2.5}{0.1*x^2+1}
 \psplot[linewidth=1.2pt,algebraic=true,plotpoints=3600]{2.5}{5}{-0.1*(x-5)^2-1}
 \psline[linewidth=0.5pt,linestyle=dashed,linecolor=lightgray](2.5,1.625)(2.5,-1.625)
 \psxTick(2.5){$z$}
\end{pspicture}
\begin{pspicture}(-1,-2.5)(5.5,3)
 \psaxes[labels=none,ticks=none]{->}(0,0)(-0.5,-2)(5,2)[$x$,-90][$y$,0]
 \psplot[linewidth=1.2pt,algebraic=true,plotpoints=3600]{0}{2}{-(1-(2.71)^(-x))}
 \psplot[linewidth=1.2pt,algebraic=true,plotpoints=3600]{2}{4}{-(1-(2.71)^(x-4))}

 \psxTick(2){$z$}
 \psxTick(4){$\pi/2$}
\end{pspicture}
\end{center}
\end{Beispiel}

\begin{Beispiel}
Finde eine partikulär Lösung zur Differentialgleichung auf dem Intervall
$[0, \frac{\pi}{2}]$
\begin{align*}
y''(x) + y'(x) = \frac{1}{\sin{x}}
\end{align*}
\begin{align*}
\Rightarrow y(x) & = \int \limits_{0}^{\frac{\pi}{2}} \dz \frac{1}{\sin{z}}
G(x,z)\\
& = -\cos{x}\int \limits_{0}^{x} \dz \frac{\sin{z}}{\sin{z}} - \sin{x} \int
\limits_{x}^{\frac{\pi}{2}}\dz \frac{\cos{z}}{\sin{z}}\\
& = -x\,\cos{x} + \sin{x}\,\ln{\sin{x}}
\end{align*}
\end{Beispiel}
\newpage
\section{Differenzial- und Integralrechnung mit mehreren Variablen}
Die Verallgemeinerung der Ableitung für Funktionen mit mehreren Variablen wird mittels der {\em partiellen Ableitung} erreicht.
\begin{itemize}
  \item Funktion $f(x,y)$
  \item partielle Ableitung
  \begin{align*}
&\frac{\partial f}{\partial x} = \lim \limits_{\Delta x\to0} \frac{f(x+\Delta x, y) - f(x,y)}{\Delta x}\\
&\frac{\partial f}{\partial y} = \lim \limits_{\Delta y\to0} \frac{f(x, y+\Delta y) - f(x,y)}{\Delta y}
\end{align*}
\end{itemize}

Die partielle Ableitung entspricht der normalen Ableitung wobei die weiteren Variablen fixiert werden.

\begin{Bemerkung}
Äquivalente Schreibweisen
  \begin{align*}
\frac{\partial f}{\partial x} \equiv \partial_x f \equiv f_x
\end{align*}
\end{Bemerkung}

Manchmal gibt man die Variable, die fixiert werden soll, noch explizit an
\begin{align*}
\left(\frac{\partial f}{\partial x}\right)_y \qquad : \text{ partielle
Ableitung nach $x$ mit $y$ fixiert}
\end{align*}

\begin{Beispiel}
\begin{align*}
& f(x,y) = x^3y - e^{xy}\\
\Rightarrow\; & \partial_x f = 3x^2y - ye^{xy}\\
&\partial_y f = x^3 - xe^{xy}\\
&\partial^2_x f = \frac{\partial}{\partial x}\left( \frac{\partial}{\partial x}f\right) = 6xy - y^2e^{xy}\\
&\partial^2_y f = \frac{\partial}{\partial y}\left( \frac{\partial}{\partial y}f\right) =-x^2e^{xy}\\
&\partial_x\partial_y f = \frac{\partial}{\partial x}\left(\frac{\partial}{\partial y}f\right) = 3x^2 - e^{xy} - xye^{xy}\\
&\partial_y\partial_x f = \frac{\partial}{\partial y}\left(\frac{\partial}{\partial x}f\right) = 3x^2 - e^{xy} - xye^{xy}\\
\end{align*}
\end{Beispiel}

\begin{Bemerkung}
\par\par
\begin{itemize}
\item Es gilt allgemein, dass die Reihenfolge der partiellen Ableitungen
  keine Rolle spielt.
  \begin{align}
\frac{\partial^2}{\partial x\partial y} f(x,y) = \frac{\partial^2}{\partial y \partial x} f(x,y)
\end{align}
\item Die Verallgemeinerung für Funktionen mit mehr als zwei Variablen folgt
analog.
\end{itemize}
\end{Bemerkung}

\begin{Beispiel} $f(x,y,z) = xyz$
\begin{align*}
\Rightarrow\;\partial_x f &= yz\\
\partial_y f &= xz\\
\partial_z f &= xy
\end{align*}
\end{Beispiel}


\par{\bf Kettenregel}\\
Die Variablen $x,y$ der Funktion $f(x,y)$ hängen von
einem Parameter $t$ ab $\Rightarrow x(t), y(t)$.\\ Die Ableitung der Funktion $f(x(t), y(t))$ nach $t$ ist somit gegeben durch
\begin{align}
\frac{\partial f}{\partial t} = \frac{\partial f}{\partial x}\frac{\partial x}{\partial t} + \frac{\partial f}{\partial y}\frac{\partial y}{\partial t}
\end{align}
\begin{info}
\begin{align*}
\Delta f(x,y) & = \frac{\partial f}{\partial x}\Delta x + \frac{\partial
f}{\partial y}\Delta y +\sigma(\Delta x, \Delta y)\\ &=  \frac{\partial
f}{\partial x}
\frac{\partial x}{\partial t}\Delta t +  \frac{\partial f}{\partial y}
\frac{\partial y}{\partial t}\Delta t\\ \Rightarrow \frac{df}{dt} & \cong \lim \limits_{\Delta t \to 0} \frac{\Delta f}{\Delta t} =  \frac{\partial f}{\partial x} \frac{\partial x}{\partial t} +  \frac{\partial f}{\partial y} \frac{\partial y}{\partial t}
\end{align*}
\end{info}

\begin{Beispiel}
\begin{itemize}
  \item $f(x,y) = xe^{-y}$ mit $x = 1+t, y= t^3$\\
  \begin{align*}
\frac{d}{dt} f(x(t), y(t)) &=\underbrace{e^{-y(t)}}_{\frac{\partial
f}{\partial x}} \cdot\underbrace{1}_{\frac{dx}{dt}} -  \underbrace{
x(t)e^{-y(t)}}_{\frac{\partial f}{\partial y}} \underbrace{3t^2}_{\frac{dy}{dt}}\\ &= \left[1-3t^2(1+t)\right]e^{-t^3}
\end{align*}
\item $f(x,y) = \int_{0}^{y} \dx e^{-zx^2}$
\begin{align*}
&\frac{\partial}{\partial y}f = e^{-zy^2}\\
&\frac{\partial}{\partial z}f = \int \limits_{0}^{y} \dx
\frac{\partial}{\partial z} e^{-zx^2} = \int \limits_{0}^{y} \dx
-x^2\;e^{-zx^2} \\
&y=t^2\qquad z=t\\
&\frac{d}{dt}\left[ \int \limits_{0}^{t^2} \dx e^{-tx^2} \right] =
e^{-zy^2}\cdot2t - \left[\int \limits_{0}^{y} \dx
x^2\;e^{-zx^2}\right]\cdot1
\end{align*}
\end{itemize}
\end{Beispiel}

\subsection{Totales Differential}
Wir betrachten die Funktion $f(x,y)$ auf $\R^2$. Das totale Differential ist eine lineare Abbildung, die jedem Vektor $\vec v = (v_x, v_y)$ eine reelle Zahl zuordnet.

\begin{align}
\left[df\right](\vec v) = \frac{d}{dt} f(x+v_xt, y+v_yt) = \frac{\partial f}{\partial x}v_x + \frac{\partial f}{\partial y} v_y
\end{align}

Insbesondere haben wir die speziellen Differentiale
\begin{align*}
&[dx](\vec v) = v_x\\
&[dy](\vec v) = v_y
\end{align*}
und somit können wir schreiben
\begin{align}
\left[df\right](\vec v) =  \frac{\partial f}{\partial x}dx + \frac{\partial f}{\partial y} dy
\end{align}

\begin{Bemerkung}
\par
\begin{itemize}
  \item Das totale Differential wird in der Physik oft als infinitesimale Änderung von $f$ unter infinitesimalem $dx$ und $dy$ interpretiert.
  \item Es ist jedoch besser es zu interpretieren, dass für $\vec v = (\Delta x, \Delta y)$ gilt
  \begin{align*}
\Delta f \cong [df](\vec v) = \frac{\partial f}{\partial x}\Delta x +
\frac{\partial f}{\partial y}\Delta y
\end{align*}
\end{itemize}
\end{Bemerkung}
Ein Differential $A(x,y)dx+B(x,y)dy$ heißt exakt, wenn die Funktion $f(x,y)$ existiert mit
\begin{align*}
&A(x,y) = \frac{\partial f}{\partial x}\\
&B(x,y) = \frac{\partial f}{\partial y}
\end{align*}
Eine notwendige Bedingung, dass ein Differential exakt ist (und auf
topologisch sehr vielen Gebieten auch hinreichend), ist
\begin{align}
\frac{\partial A(x,y)}{\partial y} = \frac{\partial B(x,y)}{\partial x}
\end{align}
\begin{Beispiel}
\begin{itemize}
  \item $y\dx + x\dy \Rightarrow f(x,y) = xy+c$
  \item $y\dx + 3x\dy \Rightarrow$ inexakt
  \item Anwendung finden wir vor allem in der Thermodynamik
  \begin{align*}
&dF = - \underbrace{S}_{\text{Entropie}}\underbrace{dT}_{\text{Änderung der Temperatur}} - \underbrace{p}_{\text{Druck}}dV : \text{ Freie Energie}\\
&\Rightarrow \left(\frac{\partial F}{\partial T} \right)_V = -S \qquad \left(\frac{\partial F}{\partial V} \right)_T = -p\\
&\Rightarrow \left(\frac{\partial S}{\partial V} \right)_T =
\left(\frac{\partial p}{\partial T} \right)_V \qquad :\text{ Maxwell Relation}\\
\end{align*}
\end{itemize}
\end{Beispiel}

\subsection{Variablen Transformation}
Wir betrachten die Funktion $f(x_1, \ldots, x_n)$. Die Variablen $x_i$ sind aber ebenfalls Funktionen von $n$-anderen Variablen $u_i$:
\begin{align*}
x_i = x_i(u_1,\ldots,u_n)
\end{align*}
aus der Kettenregel folgt somit
\begin{align}
\frac{\partial f}{\partial u_i} = \sum \limits_{j=1}^{n} \frac{\partial f}{\partial x_j} \frac{\partial x_j}{\partial u_i} = \sum \limits_{j=1}^{n}\left( \frac{\partial x_j}{\partial u_i} \right)\frac{\partial f}{\partial x_j}
\end{align}

\begin{Beispiel}[Polarkoordinaten]
\begin{align*}
&x = r\cdot\cos\varphi\\
&y = r\cdot\sin\varphi\\
\end{align*}
\begin{align*}
\Rightarrow\;r &= \sqrt{x^2 - y^2}\\
\varphi &= \arctan{\frac{y}{x}}\\
\end{align*}
\begin{align*}
\Rightarrow \frac{\partial r}{\partial x} &= \frac{x}{\sqrt{x^2+y^2}} = \cos{\varphi} \qquad \frac{\partial r}{\partial y}  =\sin{\varphi}\\
\frac{\partial \varphi}{\partial x} &= \frac{-(\frac{y}{x^2})}{1+(\frac{y}{x})^2} = -\frac{\sin\varphi}{r}\\
\frac{\partial \varphi}{\partial y} &= \frac{\frac{1}{x}}{1+(\frac{y}{x})^2} = \frac{\cos\varphi}{r}
\end{align*}
\begin{center}
\begin{pspicture}(-1,-1)(5,5)
 \psline[linecolor=framecolor](-1,-1)(-1,5)(5,5)(5,-1)(-1,-1)
 \psaxes[labels=none,ticks=none]{->}(0,0)(-0.5,-0.5)(4.5,4.5)[$x$,-90][$y$,0]
 \psline[linewidth=1.2pt,algebraic=true](0,0)(3,3)
 \psarc{->}(0,0){1}{0}{45}
 
 \rput(2,2.4){$r$}
 \rput(0.6,0.2){$\varphi$}
\end{pspicture}
\end{center}
Somit gilt
\begin{align*}
&\partial_x = \cos(\varphi)\partial_r - \frac{\sin\varphi}{r}\partial_\varphi\\
&\partial_y = \sin(\varphi)\partial_r - \frac{\cos\varphi}{r}\partial_\varphi\\
\end{align*}
\end{Beispiel}

\subsection{Mehrdimensionale Integrale}
Das bestimmte Integral
\begin{align*}
\int \limits_{a}^{b} \dx f(x)
\end{align*}
kann aufgefasst werden als Integral über die eindimensionale Region $a\le x \le
b$ der Funktion $f(x)$.
\begin{center}
\begin{pspicture}(0,-1)(5,1)
 \psline[linecolor=framecolor](0,-1)(0,5)(5,5)(5,-1)(0,-1)
 \psaxes[labels=none,yAxis=false,linewidth=1pt,ticks=none]{->}(0,0)(0,0)(4.5,0)[$x$,-90][,0]
 
 \psxTick(1){$a$}
 \psxTick(4){$b$}
\end{pspicture}
\end{center}
Mehrdimensionale Integrale erweitern dies nun auf höherdimensionale Regionen:
\par
Das bestimmte Integral von $f(x,y)$ auf der Fläche $A$ ist definiert als
\begin{align}
I = \int \limits_{A}\dx\dy f(x,y).
\end{align}
\begin{center}
\begin{pspicture}(-3,-3)(4,3)
 \psline[linecolor=framecolor](-3,-3)(-3,3)(4,3)(4,-3)(-3,-3)
 \psaxes[labels=none,ticks=none]{->}%
 (0,0)(-2.5,-2.5)(3.5,2.5)[$x$,-90][$y$,0]
 
 \psccurve[fillstyle=solid,fillcolor=lightgray]%
(-2,-1.5)(-1.5,0)(-2,1.5)%
(0,1.25)(3,1.5)(2.5,0)%
(3,-1)
\rput(2,2){$A$}
\end{pspicture}
\end{center}
Die formale Definition folgt analog zum Riemann'schen Integral über den
Grenzwert.
\begin{align}
S = \sum \limits_{p=1}^{N} f(x_p, y_p)\Delta A_p
\end{align}
\begin{center}
\begin{pspicture}(-3,-3)(4,3)
 \psline[linecolor=framecolor](-3,-3)(-3,3)(4,3)(4,-3)(-3,-3)
 
 \psaxes[labels=none,ticks=none]{->}%
 (0,0)(-2.5,-2.5)(3.5,2.5)[$x$,-90][$y$,0]
 
 \psccurve[fillstyle=solid,fillcolor=lightgray]%
(-2,-1.5)(-1.5,0)(-2,1.5)%
(0,1.25)(3,1.5)(2.5,0)%
(3,-1)
\rput(2,2){$A$}

\psline(-1.2,-2.05)(-1.2,1.59)
\psline(-0.5,-2.25)(-0.5,1.35)
\psline(-1.56,-0.3)(2.63,-0.3)
\psline(-2.05,-1.4)(2.7,-1.4)

\rput[l](-0.4,0){$(x_p,y_p)$}
\rput[r](-0.6,-0.8){$A_p$}
\end{pspicture}
\end{center}

Unterteile die Fläche $A$ in $N$ Unterflächen $\Delta A_p \quad p = 1\ldots N$
und wähle einen beliebigen Punkt $(x_p,y_p)$ in $A_p$. Falls die Summe $S$ gegen einen eindeutigen Wert konvergiert für $\Delta A_p \to 0$, existiert das Integral mit
\begin{align}
I = \int \limits_{A} \dx\dy f(x,y) = \lim \limits_{\Delta A_p \to 0} S
\end{align}
Der bequemen Weg das Integral zu berechnen ist zuerst das Integral über einen horizontalen Streifen zu berechnen
\begin{align*}
\left[\int \limits_{x_1(y)}^{x_2(y)} \dx f(x,y) \right]\dy
\end{align*}
und im zweiten Schritt das Integral über $y$ zu berechnen
\begin{align*}
I = \int \limits_{c}^{d} \dy \left(\int \limits_{x_1(y)}^{x_2(y)} \dx f(x,y) \right)
\end{align*}
\begin{center}
\begin{pspicture}(-4,-3)(4,3)
 \psline[linecolor=framecolor](-4,-3)(-4,3)(4,3)(4,-3)(-4,-3)
 
 \psaxes[labels=none,ticks=none]{->}%
 (0,0)(-3.5,-2.5)(3.5,2.5)[$x$,-90][$y$,0]
 
 \psccurve[fillstyle=solid,fillcolor=lightgray]%
 (-2,1.2)(0,1)(2,1.4)(2,-1.4)(0,-1.2)(-2,-1.2)

\rput(2,2){$A$}

\psline[linewidth=0.5pt](-2.42,-0.2)(2.52,-0.2)
\psline[linewidth=0.5pt](-2.25,-0.8)(2.4,-0.8)

\psline[linewidth=0.5pt,linestyle=dashed](0,1.53)(1.7,1.53)
\psline[linewidth=0.5pt,linestyle=dashed](0,-1.56)(1.7,-1.56)

\psline[linewidth=0.5pt,linestyle=dashed](-2.38,-0.5)(-2.38,-1.4)
\psline[linewidth=0.5pt,linestyle=dashed](2.48,-0.5)(2.48,-1.4)

\psxTick(2.55){}
\psxTick(-2.45){}
\rput[r](-2.6,0.2){$a$}
\rput[l](2.6,0.2){$b$}

\rput[r](-0.15,-1.6){$c$}
\rput[r](-0.15,1.6){$d$}

\rput(-2.38,-1.7){$x_1(y)$}
\rput(2.48,-1.7){$x_2(y)$}

\rput(2.8,-0.5){$\dy$}
\end{pspicture}
\end{center}

\begin{Bemerkung}
Als Alternative kann auch zuerst das Integral über $y$ berechnet werden und im
zweiten Schritt über $x$.
\begin{center}
\begin{pspicture}(-4,-3)(4,3)
 \psline[linecolor=framecolor](-4,-3)(-4,3)(4,3)(4,-3)(-4,-3)
 
 \psaxes[labels=none,ticks=none]{->}%
 (0,0)(-3,-2.5)(3.5,2.5)[$x$,-90][$y$,0]
 
 % Kartoffel
 \psccurve[fillstyle=solid,fillcolor=lightgray]%
 (-2,1.2)(0,1)(2,1.4)(2,-1.4)(0,-1.2)(-2,-1.2)

% Flächenrahmen
\psline[linewidth=0.5pt](-2.1,-1.1)(-2.1,1.1)
\psline[linewidth=0.5pt](-1.8,-1.35)(-1.8,1.33)

% Begrenzungsstriche y1,y2
\psline[linewidth=0.5pt,linestyle=dashed](-1.95,1.25)(-2.8,1.25)
\psline[linewidth=0.5pt,linestyle=dashed](-1.95,-1.25)(-2.8,-1.25)

% Begrenzungsstriche c,d
\psline[linewidth=0.5pt,linestyle=dashed](0,1.53)(1.7,1.53)
\psline[linewidth=0.5pt,linestyle=dashed](0,-1.56)(1.7,-1.56)

\psxTick(2.55){}
\psxTick(-2.45){}

\rput[r](-2.6,0.2){$a$}
\rput[l](2.6,0.2){$b$}

\rput[r](-0.15,-1.6){$c$}
\rput[r](-0.15,1.6){$d$}


\rput(2,2){$A$}
\rput[r](-2.9,1.25){$y_1(x)$}
\rput[r](-2.9,-1.25){$y_2(x)$}

\rput(-1.95,1.7){$\dy$}
\end{pspicture}
\end{center}
\end{Bemerkung}

\begin{Beispiel}
\begin{itemize}
  \item Die Fläche $A$ sei das Quadrat mit Seitenlänge 1 und $f(x,y) = xy^2$
  \begin{center}
\begin{pspicture}(-1,-1)(3,3)
 \psline[linecolor=framecolor](-1,-1)(-1,3)(3,3)(3,-1)(-1,-1)
 
 \psaxes[labels=none,ticks=none]{->}%
 (0,0)(-0.5,-0.5)(2.5,2.5)[$x$,-90][$y$,0]
 
% Quadratramen
\psline[linewidth=1.2pt](0,2)(2,2)
\psline[linewidth=1.2pt](2,0)(2,2)

\psxTick(2){$1$}
\psyTick(2){$1$}

\end{pspicture}
\end{center} 
  \begin{align*}
&\int \limits_{A} \dx\dy xy^2 = \int \limits_{0}^{1} \dy \int \limits_{0}^{1}
\dx xy^2 = \int \limits_{0}^{1} \dy y^2 \left(\int \limits_{0}^{1} \dx
x \right) \\
& \int \limits_{0}^{1} \dy y^2 \left[\frac{1}{2}x^2\right]_{0}^{1}
= \frac{1}{2}\int\limits_{0}^{1} \dy y^2 = \frac{1}{6}
\end{align*}
\item Die Fläche $A$ des Dreiecks beschränkt durch die Linien $x=0,y=0$ $x+y = 1$
\begin{center}
\begin{pspicture}(-1,-1)(3,3)
 \psline[linecolor=framecolor](-1,-1)(-1,3)(3,3)(3,-1)(-1,-1)
 
 \psaxes[labels=none,ticks=none]{->}%
 (0,0)(-0.5,-0.5)(2.5,2.5)[$x$,-90][$y$,0]
 
% Quadratramen
\psline[linewidth=1.2pt](0,2)(2,0)

\psxTick(2){$1$}
\psyTick(2){$1$}

\rput[l](1.2,1.2){$x+y=1$}

\end{pspicture}
\end{center}
\begin{align*}
\int \limits_{A} \dy\dy xy^2 &= \int \limits_{0}^{1} \dy y^2 \int \limits_{0}^{1-y} \dx x = \int \limits_{0}^{1}\dy y^2 \left(\left[ \frac{1}{2}x^2\right]_{0}^{1-y}\right) \\
&= \int \limits_{0}^{1}\dy \frac{1}{2} y^2 (1-y)^2 = \frac{1}{6} - \frac{1}{4} + \frac{1}{10} = \frac{10-15+6}{60}\\
&= \frac{1}{60}
\end{align*}
\end{itemize}
\end{Beispiel}

\subsection{Variablen Transformation}

Wir betrachten das Integral auf dem Bereich $A$
\begin{align*}
I = \int \limits_{A} \dx \dy f(x,y)
\end{align*}
wobei die Variablen $x(u,v), y(u,v)$ ebenfalls Funktionen von den Variablen $u,v$ sind.
\begin{align*}
\Rightarrow \; \hat{f}(u,v) = f(x(u,v),y(u,v))
\end{align*}
Im Folgenden wollen wir untersuchen, wie das Integral in den neuen Variablen aussieht.
\begin{center}
\begin{pspicture}(-1,-1)(6,6)
 \psline[linecolor=framecolor](-1,-1)(-1,6)(6,6)(6,-1)(-1,-1)
 
 \psaxes[labels=none,ticks=none]{->}%
 (0,0)(-0.5,-0.5)(5.5,5.5)[$x$,-90][$y$,0]
 
 % Kartoffel
 \psccurve[fillstyle=solid,fillcolor=lightgray]%
 (0.5,3.5)(2,4)(4,4.5)(4,3)(5,2)(2,1)(1.5,2)
 
 \psbezier[linecolor=black]%
(0.5,2)(1.5,3)%
(4,4.5)(4.5,4.5)
 \psbezier[linecolor=black]%
(0.5,1.5)(1.5,2.5)%
(4,4)(4.5,4)
 \psbezier[linecolor=black]%
(0.5,1)(1.5,2)%
(4,3.5)(4.5,3.5)

 \psbezier[linecolor=black]%
(1,4.4)(2,4.5)%
(3,1)(2.5,0.5)
 \psbezier[linecolor=black]%
(1.5,4.4)(2.5,4.5)%
(3.5,1)(3,0.5)
 \psbezier[linecolor=black]%
(2,4.4)(3,4.5)%
(4,1)(3.5,0.5)

\psdots(3.5,4.62)

\rput(3.5,5){$(x_0,y_0)$}

\rput(4.5,2){$A$}
\rput(0.5,2.6){$C$}

\rput(4.5,0.5){$u = \text{const}$}
\rput(5,3.2){$v = \text{const}$}
 %\pscurve[linewidth=0.5pt,linecolor=black](1,1)(2.5,3.5)(4,4)%

\end{pspicture}
\begin{pspicture}(-1,-1)(6,6)
 \psline[linecolor=framecolor](-1,-1)(-1,6)(6,6)(6,-1)(-1,-1)
 
 \psaxes[labels=none,ticks=none]{->}%
 (0,0)(-0.5,-0.5)(5.5,5.5)[$x$,-90][$y$,0]
 
 % Kartoffel
 \psccurve[fillstyle=solid,fillcolor=lightgray]%
 (1,0.5)(1,3)(3,2.5)(4.5,3)(4.5,1)(3,1)
 
 \psline(1,1)(2,1)(2,2)(1,2)(1,1)
 
 
\psdots(2,2.85)

\rput(2.4,3.2){$(u_0,v_0)$}

\rput(4.5,2){$A'$}
\rput(0.5,2.6){$C'$}

\rput[l](2.1,1.5){$\Delta v$}
\rput(1.5,0.75){$\Delta u$}
 %\pscurve[linewidth=0.5pt,linecolor=black](1,1)(2.5,3.5)(4,4)%

\end{pspicture}
\end{center}


\par
Zuerst müssen wir den Integrationsbereich $A$ in den neuen Integrationsbereich $A'$ umformen.\\
D.h., der Bereich $A$ ist beschränkt mit der geschlossenen Kontur $C$. Der Integrationsbereich $A'$ in den neuen Variablen ist beschränkt mit der Kontur $C'$:
\begin{itemize}
  \item $(u_0,v_0)\in C' \Rightarrow \left( x_0(u_0,v_0), y_0(u_0,v_0)\right)
  \in C$
  \item $(u,v) \in A' \Rightarrow (x(u,v),y(u,v)) \in A$
\end{itemize}
Die Integration in den $x,y$ Koordinaten ist der Limes von kleinen Volumenelementen.
\begin{center}
\begin{pspicture}(-0.5,-0.5)(5,5)
 \psline[linecolor=framecolor](-0.5,-0.5)(-0.5,5)(5,5)(5,-0.5)(-0.5,-0.5)
 \psaxes[labels=none,ticks=none]{->}%
 (0,0)(-0.5,-0.5)(4.5,4.5)[$x$,-90][$y$,0]
 
 % Kartoffel
 \psbezier%
 (1,1)(1.2,2.5)(1.8,3.6)(2,4)
 \psbezier%
 (3,1)(3.2,2.5)(3.8,3.6)(4,4)
  \psbezier%
 (0.5,1)(1.2,2)(3.6,1.3)(4,1.5)
 \psbezier%
 (0.8,2.8)(1.5,3.8)(3.9,3.1)(4.3,3.3)
 %\psline(1,1)(2,1)(2,2)(1,2)(1,1)
 
 \psline[linewidth=0.5pt]{->}(1.06,1.4)(1.25,2.8)
 \psline[linewidth=0.5pt]{->}(1.06,1.4)(2.5,2)
 \psline[linewidth=0.5pt,linestyle=dashed](2.5,2)(2.69,3.4)
 \psline[linewidth=0.5pt,linestyle=dashed](1.25,2.8)(2.69,3.4)
 
 \rput[r](1,2){$\vec{e}_v$}
 \rput[l](2.5,1.8){$\vec{e}_u$}
 
 \rput(2,2.5){$\mathrm{d}A_{uv}$}
 
 \psarc(1.06,1.4){0.5}{22}{83}
 
\end{pspicture}
\begin{pspicture}(-0.5,-0.5)(5,5)
 \psline[linecolor=framecolor](-0.5,-0.5)(-0.5,5)(5,5)(5,-0.5)(-0.5,-0.5)
 \psaxes[labels=none,ticks=none]{->}%
 (0,0)(-0.5,-0.5)(4.5,4.5)[$x$,-90][$y$,0]
 
 \psline(1.5,1.5)(1.5,3.5)(3.5,3.5)(3.5,1.5)(1.5,1.5)
 
 \rput[r](1.2,2.5){$\Delta v$}
 \rput(2.5,1){$\Delta u$}
 
\end{pspicture}
\end{center}
\par
Daher müssen wir die Fläche $dA_{uv}$ für ein kleines Quadrat $\Delta u \cdot \Delta v$ berechnen: $dA_{uv}$ ist ein Parallelogramm aufgespannt mit
\begin{align*}
&\vec{e_u} = \begin{pmatrix}\partial_u x(u,v)\\\partial_uy(uv,)\end{pmatrix} \qquad : \text{ Tangentenvektor an die Linie } v=const\\
&\vec{e_v} = \begin{pmatrix}\partial_v x(u,v)\\\partial_vy(uv,)\end{pmatrix} \qquad : \text{ Tangentenvektor an die Linie } u=const\\
\end{align*}
\begin{align*}
\Rightarrow\; dA_{uv} &= \Delta u \cdot \Delta v \cdot
|\vec{e_u}||\vec{e_v}||\sin{\varphi}|\\ 
&= \underbrace{\left|\frac{\partial x}{\partial u}\frac{\partial y}{\partial v}
- \frac{\partial x}{\partial v}\frac{\partial y}{\partial u}\right|}_{J
\text{ : Jacobian}}
\underbrace{\Delta u \cdot \Delta v}_{du\;dv}
\end{align*}
Somit folgt aus der Definition des Integrals
\begin{align}
I &= \int \limits_{A} \dx \dy f(x,y) = \lim \limits_{\Delta Ap\to 0} \sum \limits_{p=1}^{N} f(x_p,y_p)\Delta Ap\\ \nonumber
&= \lim \limits_{\Delta u\Delta v\to 0} \sum \limits_{p=1}^{N} f(x(u_p,v_p),
y(u_p,v_p))\left|\frac{\partial x}{\partial u}\frac{\partial
y}{\partial v}-\frac{\partial x}{\partial v}\frac{\partial y}{\partial
u}\right|\Delta u \cdot \Delta v\\ \nonumber &= \int \limits_{A'}\du \dv
f(x(u,v),y(u,v))\left|\frac{\partial x}{\partial u}\frac{\partial
y}{\partial v}-\frac{\partial x}{\partial v}\frac{\partial y}{\partial
u}\right|\\ \nonumber &=\int \limits_{A'} \du\dv \hat{f}(u,v)J(u,v)
\end{align}
mit dem {\em Jacobian}
\begin{align}
J(u,v) = \left|\frac{\partial x}{\partial u}(u,v)\frac{\partial y}{\partial
v}(u,v) - \frac{\partial x}{\partial v}(u,v)\frac{\partial y}{\partial
u}(u,v)\right|
\end{align}

\begin{Beispiel}
\begin{align*}
&\int\limits_{0}^{1}\dx\int\limits_{0}^{1}\dy \frac{1}{\sqrt{1-x^2}}\frac{1}{\sqrt{1-y^2}}\\
&\big[ J(\varphi,\vartheta) = \cos(\varphi)\cos(\vartheta) \big]\\
= & \int\limits_{0}^{\frac{\pi}{2}} d\varphi \int\limits_{0}^{\frac{\pi}{2}}
d\vartheta \cos{\varphi}\cos{\vartheta}\,
\frac{1}{\sqrt{1-\sin^2{\varphi}}\sqrt{1-\cos^2{\vartheta}}}\\ = & \left( \frac{\pi}{2} \right)^2
\end{align*}
\end{Beispiel}

\par{\bf Polar Koordinaten}
\begin{align*}
&x = r\cos{\varphi}\\
&y = r\sin{\varphi}\\
\end{align*}
\begin{center}
\psset{unit=1cm}
\begin{pspicture}(-1,-1)(5,5)
 \psline[linecolor=framecolor](-1,-1)(-1,5)(5,5)(5,-1)(-1,-1)
 
 \psaxes[labels=none,ticks=none]{->}%
 (0,0)(-0.5,-0.5)(4.5,4.5)[$x$,-90][$y$,0]
 
 \psline[linewidth=0.5pt,arrowsize=4pt]{->}(0,0)(3,3)
 
 \psarc[linewidth=.5pt](A){1}{0}{45}
 
 \rput(3.2,2.4){$r$}
 \rput(0.7,0.2){$\varphi$}
\end{pspicture}
\end{center}
\begin{align*}
\Rightarrow \; J(r,\varphi) &= \left|\frac{\partial x}{\partial r}
\frac{\partial y}{\partial \varphi} - \frac{\partial x}{\partial
\varphi}\frac{\partial y}{\partial r}\right| \\ &= |\cos(\varphi)r\cos(\varphi)
- \sin(\varphi)(-1)r\sin(\varphi)| = r
\end{align*}
\begin{align}
\Rightarrow \; \int\limits_{A}\dx\dy f(x,y) &= \int \limits_{A'}d\varphi\,dr\; r\,f(r\cos(\varphi), r\sin(\varphi)) \\ \nonumber
&= \int \limits_{A'} d\varphi\,dr\; r \hat{f}(r,\varphi)
\end{align}

\begin{Beispiel}
\begin{align*}
&f(x,y) = e^{-(x^2+y^2)}\\
&A:= \text{ Kreis mit Radius } R
\end{align*}
\begin{center}
\psset{unit=1cm}
\begin{pspicture}(-3,-3)(3,3)
 \psline[linecolor=framecolor](-3,-3)(-3,3)(3,3)(3,-3)(-3,-3)
 \pscircle[fillstyle=solid,fillcolor=lightgray](0,0){2}
 
 \psaxes[labels=none,ticks=none]{->}%
 (0,0)(-2.5,-2.5)(2.5,2.5)[$x$,-90][$y$,0]
 
 \psline[linewidth=0.5pt,arrowsize=4pt]{->}(0,0)(1.41,1.41)
 
 \rput(0.8,1.1){$R$}
\end{pspicture}
\end{center}
\begin{align*}
\Rightarrow &\hat{f}(r,\varphi) = e^{-r^2}\\
I &= \int\limits_{A}\dx\dy e^{-(x^2+y^2)} = \int \limits_{A'} d\varphi\,dr\;re^{-r^2}\\
&= \int \limits_{0}^{2\pi} d\varphi \int \limits_{0}^{R} dr\; re^{-r^2} = 2\pi \int \limits_{0}^{R} dr\;re^{-r^2}\\
&= 2\pi \left[\frac{1}{2}e^{-r^2}\right]_{0}^{R} = \pi\left(1-e^{-R^2}\right)
\end{align*}
\end{Beispiel}
Als Anwendung können wir jetzt das Integral
\begin{align*}
f = \int \limits_{-\infty}^{\infty} \dx e^{-x^2}
\end{align*}
berechnen:
\begin{align*}
F^2 &= \left(\int \limits_{-\infty}^{\infty}\dx e^{-x^2}\right)^2 = \left(\int \limits_{-\infty}^{\infty}\dx e^{-x^2}\right)\left(\int \limits_{-\infty}^{\infty}\dy e^{-y^2}\right)\\
&= \int \limits_{-\infty}^{\infty}\dx \int \limits_{-\infty}^{\infty}\dy e^{-(x^2+y^2)} = \int\limits_{0}^{2\pi}d\varphi\;\int\limits_{0}^{\infty}dr\;re^{-r^2} = \pi\\
\Rightarrow & F= \int \limits_{-\infty}^{\infty}\dx e^{-x^2} = \sqrt{\pi}
\end{align*}
Die Verallgemeinerung der Variablen Transformation in höheren Dimensionen ergibt:
\begin{align*}
f(x_1,\ldots,x_n)\qquad x_i(u_i,\ldots,u_j)
\end{align*}
\begin{align*}
I &= \int\limits_{A} dx_1\,\ldots,dx_n\qquad f(x_1,\ldots,x_n) \\
&= \int\limits_{A'}
du_1\,\ldots,du_n\qquad \hat{f}(u_1,\ldots,u_n)\ J(u_1,\ldots,u_n)
\end{align*}
Der Jacobian ist stets durch die Variablentransformationsmatrix gegeben
\begin{align*}
M = \begin{pmatrix} \frac{\partial x_1}{\partial u_1} & \ldots & \frac{\partial
x_n}{\partial u_1} \\ \vdots & \vdots & \vdots \\ \frac{\partial x_1}{\partial
u_n} & \ldots & \frac{\partial x_n}{\partial u_n} \end{pmatrix} \qquad n\times
n\text{ Einträge},
\end{align*}
wobei $J = \operatorname{det}M : \text{Determinante von $M$ (siehe Lineare
Algebra)}$.
\par
In der dritten Dimension lautet er
\begin{align*}
M = \begin{pmatrix} \frac{\partial x}{\partial u} & \frac{\partial y}{\partial u} & \frac{\partial z}{\partial u}\\
\frac{\partial x}{\partial v} & \frac{\partial y}{\partial v} & \frac{\partial z}{\partial v}\\
\frac{\partial x}{\partial w} & \frac{\partial y}{\partial w} & \frac{\partial z}{\partial w}\end{pmatrix}
\end{align*}
und
\begin{align*}
J\ =\; &\partial_ux(\partial_vy\partial_wz-\partial_wy\partial_vz) \\
&- \partial_uy(\partial_vx\partial_wz - \partial_wx\partial_vz) \\
&+\partial_uz(\partial_vx\partial_wy-\partial_wx\partial_vy)
\end{align*}

\par{\bf Kugelkoordinaten}
\begin{align*}
x &= r\cos(\varphi)\sin(\vartheta)\\
y &= r\sin(\varphi)\cos(\vartheta)\\
z &= r\cos(\vartheta)\\
\Rightarrow\;r &= \sqrt{x^2+y^2+z^2}
\end{align*}
\begin{center}
\psset{unit=1cm}
\begin{pspicture}(-3,-3)(5,5)
 \psline[linecolor=framecolor](-3,-3)(-3,5)(5,5)(5,-3)(-3,-3)
 \psline[arrowsize=4pt]{->}(0,0)(0,4.5)
 \psline[arrowsize=4pt]{->}(0,0)(4.5,0)
 \psline[arrowsize=4pt]{->}(0,0)(-2.5,-2.5)

 
 \psline[linewidth=0.5pt,arrowsize=4pt,linestyle=dashed](0,0)(2,-2)
 \psline[linewidth=0.5pt,arrowsize=4pt,linestyle=dashed](-2,-2)(2,-2)
 \psline[linewidth=0.5pt,arrowsize=4pt,linestyle=dashed](2,-2)(4,0)
 \psline[linewidth=0.5pt,arrowsize=4pt,linestyle=dashed](2,-2)(2,2)
 \psline[linewidth=0.5pt,arrowsize=4pt,linestyle=dashed](0,4)(2,2)
 \psline[linewidth=0.5pt,arrowsize=4pt]{->}(0,0)(2,2)
 
 \rput(2.2,2.2){$\vec r$}
 
 
 \rput(-2.7,-2.5){$x$}
 \rput(4.5,0.3){$y$}
 \rput(-0.3,4.5){$z$}
 
 \psarc(0,0){1.5}{45}{90}
 \psarc(0,0){1}{225}{315}
 
 \rput(0,-0.7){$\varphi$}
 \rput(0.3,0.7){$\vartheta$}
  
\end{pspicture}
\end{center}
Die Fläche $r = const$ beschreibt eine Kugel mit Zentrum $(0,0,0)$ und Radius
$r$.
\begin{itemize}
  \item $r\in[0,\infty)$
  \item $\varphi\in[0,2\pi)$
  \item $\vartheta\in[0,\pi)$
\end{itemize}

\begin{align*}
I = \int\limits_{A}\dx\dy\dz f(x,y,z) = \int\limits_{A'}d\varphi\,d\vartheta\,\sin(\vartheta)\,dr\;r^2\hat{f}(r,\varphi,\vartheta)
\end{align*}

\begin{Beispiel}
Volumen der Kugel: $A = \{(x,y,z) \;|\; x^2 + y^2 + z^2 \le R^2\}$
\begin{align*}
\int \limits_{A}\dx\dy\dz 1 &= \underbrace{\int \limits_{0}^{2\pi}
d\varphi}_{2\pi} \underbrace{\int \limits_{0}^{\pi} d\vartheta\;
\sin{\vartheta}}_{2} \underbrace{\int \limits_{0}^{R} dr\;
R^2}_{\frac{1}{3}\pi R^3} \\
&= \frac{4\pi}{3}R^3
\end{align*}
\end{Beispiel}

\newpage
\section{Vektoranalysis}
Wir sind interessiert an Funktionen von $\R^m\to\R^n$.
\begin{itemize}
 \item $f(x)$: reelle Funktion
 \item $f(\vec{r})$: skalares Feld $\phi(\vec{r}), E(\vec{r})$
 \item $\vec{f}(x)$: Raumkurve, $\vec{x}(t), \vec{v}(t)$
 \item $\vec{f}(\vec{r})$: Vektorfeld, $\vec{F}(\vec{r}), \vec{E}(\vec{r})$
\end{itemize}
Ableiten eines Vektors:
\begin{align*}
 \vec{x}(t) &= \begin{pmatrix}x_1(t)\\x_2(t)\\ \vdots\\x_n(t)\end{pmatrix} =
 \sum \limits_{i=1}^{n} x_i(t) \vec{e}_i\\ \Rightarrow \frac{d}{dt}\vec{x}(t) &= \begin{pmatrix} \dot{x}_1(t) \\  \dot{x}_2(t) \\ \vdots \\ \dot{x}_n(t) \end{pmatrix} = \sum \limits_{i=1}^{n} \dot{x}_i(t) \vec{e}_i
\end{align*}
\begin{align*}
 \vec{A}(\vec{r}) &= \begin{pmatrix}A_1(\vec{r})\\A_2(\vec{r})\\
 \vdots\\A_n(\vec{r})\end{pmatrix} = \sum \limits_{i=1}^{n}
 A_i(\vec{r})\vec{e}_i\\ \Rightarrow  \frac{\partial}{\partial x}\vec{A}(\vec{r}) &= \begin{pmatrix}\partial_x A_1(\vec{r})\\ \partial_x A_2(\vec{r})\\ \partial_x A_3(\vec{r}) \end{pmatrix} = \sum \limits_{i=1}^{n} \frac{\partial}{\partial x} A_i(\vec{r}) \vec{e_i}
\end{align*}

\begin{Bemerkung}
 Mittels Kettenregel ermitteln wir das Verhalten von verschiedenen Punkten
 \begin{itemize}
  \item
  $\frac{d}{du}(\phi \vec{a}) = \phi \frac{d\vec{a}}{du} + \left(\frac{d\phi}{du}\right)\vec{a}$
  \item
  $\frac{d}{du}(\vec{a}\cdot\vec{b}) = \vec{a} \frac{d\vec{b}}{du} + \left(\frac{d\vec{a}}{du}\right)\vec{b}$
  \item
  $\frac{d}{du}(\vec{a}\times\vec{b}) = 
   \left(\frac{d\vec{a}}{du}\right)\times\vec{b} +  \vec{a} \times \left(
   \frac{d\vec{b}}{du}\right)$
 \end{itemize}
\end{Bemerkung}
\begin{Beispiel}{\bf  Kugelkoordinaten}
\begin{align*}
 \vec{r}(r,\varphi,\vartheta) = \begin{pmatrix}r\cos(\varphi)\sin(\vartheta)\\r\sin(\varphi)\sin(\vartheta)\\r\cos(\vartheta)\end{pmatrix}
\end{align*}
\begin{align*}
 \Rightarrow \vec{e}_r &= \frac{d\vec{r}}{dr} = \begin{pmatrix}\cos(\varphi)\sin(\vartheta)\\\sin(\varphi)\sin(\vartheta)\\\cos(\vartheta)\end{pmatrix}\\
 \vec{e}_\varphi &= \frac{d\vec{r}}{d\varphi} = \begin{pmatrix}-r\sin(\varphi)\sin(\vartheta)\\r\cos(\varphi)\sin(\vartheta)\\0\end{pmatrix}\\
 \vec{e}_\vartheta &= \frac{d\vec{r}}{d\vartheta} = \begin{pmatrix}r\cos(\varphi)\cos(\vartheta)\\r\sin(\varphi)\cos(\vartheta)\\-r\sin(\vartheta)\end{pmatrix}
\end{align*}

\begin{align*}
\Rightarrow &|\vec{e}_\varphi| = |\vec{e}_\vartheta| = r\\
&|\vec{e}_r| = 1\\
&\vec{e}_r\cdot\vec{e}_\varphi = \vec{e}_r\cdot\vec{e}_\vartheta =
\vec{e}_\varphi\cdot\vec{e}_\vartheta = 0
\end{align*}
\end{Beispiel}

\subsection{Linien- und Oberflächenintegrale}

Ein Linienintegral ist gegeben durch Ausdrücke von der Form
\begin{align*}
I = \int \limits_{C} \dvecr\vec{A}(x,y,z),
\end{align*}
wobei $C$ eine Kurve im Raum darstellt. Die Kurve ist parametrisiert durch
einen Parameter $t$.
\begin{align*}
&[t_0,t_1] \to \R^3\qquad t \mapsto \vec{r}(t) \in C
\end{align*}
mit $\vec{r}(t_0) = \vec{r}_A,\; \vec{r}(t_1) = \vec{r}_B$. Die Berechnung des
Linienintegrals erfolgt dann mittels
\begin{align*}
\dvecr&= \frac{d}{dr}\vec{r}(t) \cdot \dt\\
\Rightarrow I &= \int \limits_{t_0}^{t_1} \dt \left[ \frac{d}{dt}\vec{r}(t)
\right]\cdot \vec{A}(\vec{r}(t))
\end{align*}

\begin{Beispiel}
Die Arbeit entlang eines Weges $\vec{r}(t)$ im Kraftfeld $\vec{F}(\vec{r})$
\begin{align*}
W = \int \limits_{C} \dvecr \vec{F}(\vec{r})
\end{align*}
\end{Beispiel}

\begin{Bemerkung}
Das Integral ist unabhängig von der Parametrisierung der Kurve $C$:
\begin{align*}
t = f(s) \Rightarrow \vec{r}(f(s)) \equiv \vec{\hat{r}}(s)
\end{align*}
\begin{align*}
\Rightarrow I &= \int \limits_{C} \dvecr \vec{A}(\vec{r}) = \int
\limits_{t_0}^{t_1} \dt \left[ \frac{d\vec{r}}{dt}\right]
\vec{A}(\vec{r}(t))\\
& \overset{dt = f'(s)ds,t = f(s)}{=} \int \limits_{s_0}^{s_1} \ds \underbrace{f'
\left[ \frac{d\vec{r}}{dt} \right]}_{\frac{d\vec{r}}{dt}} \vec{A}(\vec{r}(t))
\end{align*}
\end{Bemerkung}

\begin{Beispiel}
\begin{align*}
\vec{F} &= \frac{1}{x^2+y^2}\begin{pmatrix}-y \\ x\end{pmatrix}\\
\vec{r}(\varphi) &= R\begin{pmatrix}\cos(\varphi) \\ \sin(\varphi)
\end{pmatrix}\\
\frac{d}{d\varphi}\vec{r} &= R \begin{pmatrix}-\sin(\varphi) \\ \cos(\varphi)
\end{pmatrix}
\end{align*}
\begin{align*}
\Rightarrow I &= \int \limits_{0}^{2\pi} \dphi R \begin{pmatrix}-\sin(\varphi) \\ \cos(\varphi)
\end{pmatrix} \begin{pmatrix}-\sin(\varphi) \\ \cos(\varphi)
\end{pmatrix} \frac{R}{(\cos^2(\varphi)+\sin^2(\varphi))R^2}\\
&= \int \limits_{0}^{2\pi} \dphi = 2\pi
\end{align*}
\end{Beispiel}

\begin{Bemerkung}{\bf Weitere Linienintegrale}
\begin{align*}
\int \limits_{C} \dvecr f(\vec{r}) = \int \limits_{t_0}^{t_1} \dt
f(\vec{r}(t)) \frac{d\vec{r}}{dt} \text{ : Vektorgröße}
\end{align*}
{\bf Bogenlänge}
\begin{align*}
&\int \limits_{C} \ds = \int \limits_{t_0}^{t_1} \dt
\left|\frac{d\vec{r}(t)}{dt}\right|\\ \Rightarrow  &\int \limits_{C} \ds
\phi(\vec{r})
\text{ : Skalare Größe}\\ &\int \limits_{C} \ds \vec{F}(\vec{r}) \text{ : Vektorgröße}
\end{align*}
\end{Bemerkung}

\begin{Beispiel}[Länge einer Geraden]
\begin{align*}
&\vec{r}(t) = \begin{pmatrix}t\\ \sin(\varphi) t\end{pmatrix}\qquad 0\le
t\le1\\ &\int \limits_{C} \ds = \int \limits_{0}^{1} \dt \sqrt{1+\sin(\varphi)} =
\sqrt{1+\sin^2(\varphi)}
\end{align*}
\end{Beispiel}
\begin{Beispiel}[Umfang eines Kreises]
\begin{align*}
 &\vec{r}(\varphi) =
R\begin{pmatrix}\cos(\varphi)\\ \sin(\varphi) \end{pmatrix}\\
 &L = \int \limits_{C} \ds = \int \limits_{0}^{2\pi} \dphi
 R\sqrt{\sin^2(\varphi) + \cos^2(\varphi)} = R\cdot2\pi
 \end{align*}
\end{Beispiel}

Eine Fläche im $\R^3$ können wir darstellen mit Hilfe von zwei Parametern.
\begin{align*}
\vec{r}(u,v) = \begin{pmatrix}x(u,v)\\y(u,v)\\z(u,v)\end{pmatrix}
\end{align*}

\begin{Beispiel}
Eine Ebene aufgespannt durch $\vec{a}$ und $\vec{b}$ durch den Punkt $\vec{r}_0$
hat die Form
\begin{align*}
\vec{r}(u,v) = \vec{r}_0 + u\vec{a} + v\vec{b}.
\end{align*}
Eine Kugel mit dem Radius $R$
\begin{align*}
\vec{r}(\varphi, \vartheta) =
\begin{pmatrix}\cos(\varphi)\sin(\vartheta)\\\sin(\varphi)\sin(\vartheta)\\\cos(\vartheta)\end{pmatrix}.
\end{align*}
\end{Beispiel}

Die Vektoren
\begin{align*}
&\vec{e}_u = \frac{\partial}{\partial u} \vec{r}(u,v)\\
&\vec{e}_v = \frac{\partial}{\partial v} \vec{r}(u,v)\\
\end{align*}
beschreiben die Tangentenvektoren an die Linien auf der Fläche mit $v=const$
und $u=const$.\\
Das Flächendifferential $dA$ in einem Punkt hat somit die Form
\begin{align*}
dA = |\vec{e}_u(u,v) \times \vec{e}_v(u,v)| \overset{du\ dv}{=}
\left|\frac{d\vec{r}}{du}(u,v) \times \frac{d\vec{r}}{dv}(u,v)\right|du\ dv
\end{align*}
Somit ist der Flächeninhalt einer Fläche im $\R^3$ gegeben durch
\begin{align*}
F_B = \int \int \limits_{B} \du \dv \left|\frac{d}{du}\vec{r}(u,v) \times
\frac{d}{dv}\vec{r}(u,v)\right|
\end{align*}
\begin{Beispiel}{Darstellung der Fläche}
\begin{align*}
\vec{r}(u,v) = \begin{pmatrix}1\\0\\0\end{pmatrix} + u
\begin{pmatrix}-1\\1\\0\end{pmatrix} + v \begin{pmatrix}-1\\0\\1\end{pmatrix}
\end{align*}
\begin{align*}
\Rightarrow & \vec{e}_u = \begin{pmatrix}-1\\1\\0\end{pmatrix}\\
& \vec{e}_v = \begin{pmatrix}-1\\0\\1\end{pmatrix}\\
\Rightarrow & \left|\frac{d\vec{r}}{du}(u,v) \times
\frac{d\vec{r}}{dv}(u,v)\right| =
\left|\begin{pmatrix}1\\1\\1\end{pmatrix}\right| =
\sqrt{3}
\end{align*}
\begin{align*}
F_B = \int \limits_{0}^{1} \dv \int \limits_{0}^{1-v} \du \sqrt{3} = \int
\limits_{0}^{1} \dv \sqrt{3}(1-v) = \frac{\sqrt{3}}{2}
\end{align*}
\end{Beispiel}
Bei einem Oberflächenintegral wird jetzt jeder Punkt auf der Oberfläche mit
einer Funktion gewichtet.
\begin{align*}
\int \limits_{S} \ds \phi(\vec{r}) = \int \limits_{B} \du \dv
\phi(\vec{r}(u,v)) \left|\frac{d\vec{r}}{du}(u,v) \times
\frac{d\vec{r}}{dv}(u,v)\right| \text{ : Skalares Integral}
\end{align*}
Es ist jedoch auch möglich vektorielle Oberflächenintegrale zu definieren.
Insbesondere ist das Oberflächenelement
\begin{align*}
d\vec{A} = \frac{d\vec{r}}{du} \times
\frac{d\vec{r}}{dv} du dv = \vec{n} dA
\end{align*}
eine Vektorgröße, wobei der Einheitsvektor $\vec{n}$ senkrecht auf der
Oberfläche steht.
\begin{align*}
\vec{n} = \frac{\frac{d\vec{r}}{du} \times
\frac{d\vec{r}}{dv}}{\left| \frac{d\vec{r}}{du} \times
\frac{d\vec{r}}{dv} \right|}
\end{align*}
Die Richtung von $\vec{n}$ hängt von der Orientierung der Oberfläche ab. Bei
geschlossenen Flächen, wie z.B. einer Kugel, wird die Orientierung normalerweise
so gewählt, dass $\vec{n}$ nach außen zeigt.

\begin{Beispiel}[Kugel]
\begin{align*}
&\vec{r}(\vartheta,\varphi) =
\begin{pmatrix}\cos(\varphi)\sin(\vartheta) \\
\sin(\varphi)\sin(\vartheta) \\ \cos(\vartheta)\end{pmatrix}\\
&\vec{e}_\varphi = \frac{\partial \vec{r}}{\partial \varphi} =
\begin{pmatrix}-\sin(\varphi)\sin(\vartheta) \\ \cos(\varphi)\sin(\vartheta) \\
0\end{pmatrix}\\
&\vec{e}_\vartheta = \frac{\partial \vec{r}}{\partial \vartheta} =
\begin{pmatrix}\cos(\varphi)\cos(\vartheta) \\ \sin(\varphi)\cos(\vartheta) \\
-\sin(\vartheta)\end{pmatrix}\\
\Rightarrow & \vec{e}_\vartheta \times \vec{e}_\varphi =
\begin{pmatrix}\cos(\varphi)\sin^2(\vartheta) \\ \sin(\varphi)\sin^2(\vartheta)
\\ \cos(\vartheta)\sin(\vartheta)\end{pmatrix} = \sin(\vartheta) \begin{pmatrix}\cos(\varphi)\sin(\vartheta) \\ \sin(\varphi)\sin(\vartheta)
\\ \sin(\vartheta)\end{pmatrix}\\
&\left| \vec{e}_\vartheta \times \vec{e}_\varphi \right| = |\sin(\vartheta)|\\
\Rightarrow & \vec{n} = \begin{pmatrix}\cos(\varphi)\sin(\vartheta) \\
\sin(\varphi)\sin(\vartheta) \\ \cos(\vartheta)\end{pmatrix} =
\vec{r}(\vartheta, \varphi)
\end{align*}
\end{Beispiel}

Wir finden jetzt die weiteren Oberflächenintegrale
\begin{align*}
&\int \limits_{S} \dvecs \vec{F}(\vec{r}) = \int \limits_{B} \du\dv
\vec{F}(u,v) \cdot \left( \frac{\partial \vec{r}}{\partial u} \times
\frac{\partial \vec{r}}{\partial v}\right) \text{ : skalare Größe}\\
&\int \limits_{S} \dvecs \phi(\vec{r}) = \int \limits_{B} \du\dv
\phi(u,v) \left( \frac{\partial \vec{r}}{\partial u} \times
\frac{\partial \vec{r}}{\partial v}\right) \text{ : Vektorgröße}\\
&\int \limits_{S} \dvecs \times \vec{F}(\vec{r}) = \int \limits_{B} \du\dv
 \left( \frac{\partial \vec{r}}{\partial u} \times
\frac{\partial \vec{r}}{\partial v}\right) \times \vec{F}(u,v) \text{ :
Vektorgröße}\\
\end{align*}

\begin{Beispiel}
\begin{itemize}
\item  Der Tragflügel eines Flugzeugs erzeugt durch seine Form und den Luftstrom
verschiedene Drücke auf der Unter- und Oberseite. Bezeichnen wir mit
$p(\vec{r})$ das erzeugte Druckfeld, so ist der Auftrieb des Flugzeugs gegeben
durch
\begin{align*}
\vec{F} = \int \limits_{S} \dvecs  p(\vec{r}) \text{ : wobei $S$ die
Oberfläche der Tragflächen ist.}
\end{align*}
\item Eine Flüssigkeit fließt mit einem Geschwindigkeitsfeld
$\vec{v}(\vec{r})$. D.h. jedem Punkt $\vec{r}$ ordnen wir die lokale
Geschwindigkeit der Flüssigkeit zu. Zudem hat sie eine Massendichte
$\rho(\vec{r})$.\\
Die lokale Masse, die pro Zeiteinheit durch die Oberfläche fließt, ist
gegeben durch
\begin{align*}
M = \int \limits_{S} \dvecs \cdot \vec{v}(\vec{r})\rho(\vec{r}).
\end{align*}
\end{itemize}
\end{Beispiel}

\subsection{Ableitungsoperatoren}
\begin{Definition}[Gradient]
Ein Skalarfeld $\phi(\vec{r})$ ordnet jedem Punkt $\vec{r}$ im Raum eine reelle
Zahl zu. Der Gradient des Skalarfeldes ist definiert als
\begin{align*}
\nabla \phi(\vec{r}) = \frac{\partial \phi}{\partial x}\vec{e}_x +
\frac{\partial \phi}{\partial y}\vec{e}_y + \frac{\partial \phi}{\partial
z}\vec{e}_z.
\end{align*}
Somit ist $\nabla \phi(\vec{r})$ ein Vektorfeld. In seinen Komponenten hat es
die Form (in der natürlichen Basis)
\begin{align*}
\nabla \phi(\vec{r}) = \begin{pmatrix}\partial_x\phi \\
\partial_y\phi \\ \partial_z\phi\end{pmatrix}.
\end{align*}
\end{Definition}

Das Verhalten des Skalarfeldes von einem Punkt $\vec{r}_0$ lässt sich mit dem
Gradienten sehr einfach beschreiben.
\begin{align*}
&\vec{r}(s) = \vec{r}_0 + s\vec{a}\\
\Rightarrow & \frac{\partial \phi(\vec{r}(s))}{ds} = \vec{a} \cdot \nabla
\phi(\vec{r}_0) \text{ : Richtungsableitung}
\end{align*}
Der Gradient zeigt somit in die Richtung der stärksten Zunahme des
Skalarfeldes.\\
Die Gleichung $\phi(\vec{r}) = const$ bestimmt eine Fläche im $\R^3$, wobei der
Gradient immer senkrecht auf dieser Fläche steht.\\

\begin{Definition}[Ableitungsoperator]
Der Name {\em Ableitungsoperator} kommt von der Eigenschaft, dass jedem
Skalarfeld ein Vektorfeld zugeordnet ist. Daher schreibt man auch oft
\begin{align*}
\nabla = \partial_x \vec{e}_x + \partial_y \vec{e}_y + \partial_z \vec{e}_z, 
\text{ : Nabla}
\end{align*}
wobei für jedes Skalarfeld gilt
\begin{align*}
\nabla \phi(\vec{r}) = (\partial_x \vec{e}_x + \partial_y \vec{e}_y +
\partial_z \vec{e}_z)\phi(\vec{r}) = \partial_x \phi \vec{e}_x + \partial_y \phi
\vec{e}_y + \partial_z \phi \vec{e}_z
\end{align*}
\end{Definition}


\begin{Beispiel}
  \begin{align*}
  &\phi(x,y,z) = xyz\qquad \nabla\phi =
  \begin{pmatrix}yz\\xz\\yx\end{pmatrix}\\
   &\phi(x,y,z) = MmG \frac{1}{\sqrt{x^2+y^2+z^2}} \text{ :
 Gravitationspotential}\\
 \Rightarrow & F_g = -\nabla \phi = MmG
 \frac{1}{\sqrt{x^2+y^2+z^2}^{\frac{3}{2}}}\begin{pmatrix}x\\y\\z\end{pmatrix}
 \text{ : Gravitationskraft}
 \end{align*}

\end{Beispiel}

\begin{Definition}[Divergenz]
Betrachte ein Vektorfeld $\vec{A}(\vec{r})$, das jedem Punkt $\vec{r}$ einen
Vektor zuordnet.\\
Die {\em Divergenz} von $\vec{A} =
\begin{pmatrix}A_x\\A_y\\A_z\end{pmatrix}$ ist definiert als
\begin{align*}
\div \vec{A} \equiv \nabla \cdot \vec{A} = \frac{\partial A_x}{\partial x} +
\frac{\partial A_y}{\partial y} + \frac{\partial A_z}{\partial z}.
\end{align*}
\end{Definition}
Die Divergenz eines Vektorfeldes beschreibt physikalisch Quellterme.

\begin{itemize}
  \item Die Divergenz eines $\vec{E}$-Feldes verschwindet, wenn keine Ladungen
  vorhanden sind: $\div \vec{E} = 0$.
  \item Für das Geschwindigkeitsfeld $\vec{v}(\vec{r})$ und die Massendichte
  $\rho(\vec{r})$ einer fließenden Flüssigkeit gilt $\div \left[
  \rho(\vec{r})\vec{v}(\vec{r}) \right] = 0$, wenn keine Quelle/Abfluss
  vorhanden ist.
\end{itemize}

\begin{Beispiel}
\begin{align*}
&\vec{A}(x,y,z) = \begin{pmatrix}x\\y\\z\end{pmatrix} \Rightarrow \div \vec{A}
= 3\\
&\vec{A}(x,y,z) = \begin{pmatrix}y\\z\\x\end{pmatrix} \Rightarrow \div \vec{A}
= 0\\
\end{align*}
\end{Beispiel}

\begin{Definition}[Laplace Operator]
Die Kombination von Gradient und Divergenz ergibt den {\em Laplace Operator}
für ein Skalarfeld $\phi(\vec{r})$.
\begin{align*}
\Delta \phi(\vec{r}) = \div \nabla \phi(\vec{r}) = \partial^2_x\phi +
\partial^2_y\phi + \partial^2_z\phi
\end{align*}
\end{Definition}

\begin{Beispiel}
Gravitationspotential $\phi(\vec{r}) = MmG \frac{1}{\sqrt{x^2+y^2+z^2}}$
\begin{align*}
\Rightarrow \Delta \phi(\vec{r}) = 0 \text{ für } \vec{r} \neq 0
\end{align*}
Da aber bei $\vec{r} = 0$ eine Quelle/Masse sitzt, die ein Gravitationsfeld
erzeugt, sollte also $\div \nabla \phi(\vec{r})$ bei $\vec{r} = 0$ nicht
verschwinden. In der Tat gilt:
\begin{align*}
\Delta \phi(\vec{r}) = 4\pi MmG
\underbrace{\delta(x)\delta(y)\delta(z)}_{\delta(\vec{r})},
\end{align*}
mit der bekannten $\delta$-Funktion.
\end{Beispiel}

\begin{Definition}[Rotation]
Die {\em Rotation} ist definiert für ein Vektorfeld $\vec{A}(\vec{r})$ mittels
\begin{align*}
\rot \vec{A}(\vec{r}) &= \nabla \times \vec{A}(\vec{r}) \\
&=
\vec{e}_x\left(\partial_y A_z - \partial_z A_y\right) +
\vec{e}_y\left(\partial_z A_x - \partial_x A_z\right) +
\vec{e}_z\left(\partial_x A_y - \partial_y A_x\right) \\
&=
\begin{pmatrix}\partial_y A_z - \partial_z A_y \\ \partial_z A_x -
\partial_x A_z \\ \partial_x A_y - \partial_y A_x\end{pmatrix}
\end{align*}
und ist somit wieder ein Vektorfeld.
\end{Definition}

Die physikalische Interpretation ist, dass die Rotation Wirbel/Drehungen
beschreibt:
\par
Betrachte das Kraftfeld $\vec{F}(x,y,z) =
\begin{pmatrix}0\\x\\0\end{pmatrix}$.\\ 
Ein Objekt das wir in einem solchen Feld platzieren, beginnt sich um seine
eigenen Achse zu drehen.
\begin{align*}
\rot \vec{F} = \begin{pmatrix}0\\0\\ \partial_x F_y\end{pmatrix} =  
\begin{pmatrix}0\\0\\1\end{pmatrix} \text{ : Drehachse des Objekts}
\end{align*}



\begin{Beispiel}
{\bf Maxwell Gleichung Elektrodynamik}
 \begin{align*}
  &\vec{E} : \text{ elektrisches Feld}, && \vec{B} : \text{ magnetisches Feld}\\
  &\rho : \text{ Ladungsdichte}, &&j : \text{ Ladungsstrom}\\
  &\div \vec{E} = 4\pi\rho(\vec{r}) && \rot \vec{E} + \frac{1}{L}\partial_t
  \vec{B} = 0\\
   &\div \vec{B} = 4\pi\rho(\vec{r}) && \rot \vec{B} + \frac{1}{L}\partial_t
   \vec{E} = j(\vec{r}
 \end{align*}
{\bf Quantenmechanik des Wasserstoffatoms}
\begin{align*}
 i\hbar\partial_t \psi(\vec{r},t) = \frac{\hbar^2}{2m}\Delta\psi(\vec{r},t) + \frac{l^2}{|\vec{r}|}\psi(\vec{r},t)
\end{align*}
{\bf Navier-Stokes}
\begin{align*}
 \rho\left(\frac{d\vec{v}}{dt} +\vec{v}\cdot\nabla\cdot\vec{v}\right) = -\nabla p + \mu \Delta\vec{v}
\end{align*}
\end{Beispiel}

\begin{Bemerkung}
Für ein Skalarfeld $\phi(\vec{r})$ gilt
\begin{align*}
\rot \nabla \phi(\vec{r}) = 0
\end{align*}
\begin{info}
\begin{align*}
\rot \begin{pmatrix}\partial_x \phi \\ \partial_y \phi \\
\partial_z \phi\end{pmatrix} = \begin{pmatrix}\partial_y\partial_z \phi -
\partial_z\partial_y \phi \\ \partial_z\partial_x \phi - \partial_x\partial_z
\phi \\ \partial_x\partial_y \phi - \partial_y\partial_x \phi\end{pmatrix} = \begin{pmatrix}0\\0\\0\end{pmatrix}
\end{align*}
\end{info}
\par
Für ein Vektorfeld $\vec{A}(\vec{r})$ gilt
\begin{align*}
\div \rot \vec{A}(\vec{r}) = 0
\end{align*}
\end{Bemerkung}

Zudem sind folgende Relationen einfach zu beweisen
\begin{align*}
&\nabla\cdot(\phi\vec{A}) = (\nabla\phi)\cdot\vec{A} + \nabla\cdot\vec{A}\\
&\nabla\times(\phi\vec{A}) = (\nabla\phi)\times\vec{A} +
\phi(\nabla\times\vec{A})\\
&\nabla\cdot(\vec{A}\times\vec{B}) = \vec{B}\cdot(\nabla\times\vec{A}) -
\vec{A}\cdot(\nabla\times\vec{B})\\
&\nabla\times(\vec{A}\times\vec{B}) = \vec{A}(\nabla\cdot\vec{B}) -
\vec{B}(\nabla\cdot\vec{A}) + (\vec{B}\cdot\nabla)\vec{A} -
(\vec{A}\cdot\nabla)\vec{B}\\
&\nabla\times(\nabla\times\vec{A}) = \nabla(\nabla\cdot\vec{A}) - \Delta\vec{A}
\end{align*}

\subsection{Gauß'scher Integralsatz}
Die Integralsätze stellen einen Zusammenhang her zwischen den
Ableitungsoperatoren und den Oberflächenintegralen. Der Gauß'sche Integralsatz
besagt
\begin{align*}
\int \limits_{V} \dvecr \div \vec{A}(\vec{r}) = \int \limits_{S=\partial V} \dvecs \cdot
\vec{A}(\vec{r}) = \int \limits_{S = \partial V} \ds
\vec{n}\cdot\vec{A}(\vec{r}),
\end{align*}
wobei $\partial V$ die Oberfläche des Volumens $V$ beschreibt und $\vec{n}$
senkrecht auf der Oberfläche steht und nach außen zeigt.
\par
\begin{info}
Für einen Quader mit dem Volumen $V$ gilt
\begin{align*}
&\int \limits_{V} \dV \div \vec{A} \\
=&\int
\limits_{0}^{a}\dx\int\limits_{0}^{b}\dy\int\limits_{0}^{c}\dz
\left[\partial_xA_x + \partial_yA_y + \partial_zA_z\right]\\
=&\int\limits_{0}^{b}\dy\int\limits_{0}^{c}\dz A_x(a,y,z) -
\int\limits_{0}^{b}\dy\int\limits_{0}^{c}\dz A_x(0,y,z) \\&+
\int\limits_{0}^{a}\dx\int\limits_{0}^{c}\dz A_y(x,b,z) -
\int\limits_{0}^{a}\dx\int\limits_{0}^{c}\dz A_y(x,0,z) \\&+
\int\limits_{0}^{a}\dx\int\limits_{0}^{b}\dy A_z(x,y,c) - 
\int\limits_{0}^{a}\dx\int\limits_{0}^{b}\dy A_z(x,y,0) \\=& \int
\limits_{\partial V = S} \dvecs \cdot\vec{A}
\end{align*}
Für ein allgemeines Volumen folgt der Satz durch Zerlegen des Volumens in
kleine Quader und mittels einem Grenzwert.
\end{info}

\begin{Bemerkung}
Falls das umschlossene Gebiet Löcher aufweist, so müssen diese Löcher bei der
Bestimmung des Randes berücksichtigt werden.
\end{Bemerkung}

\begin{Beispiel}
Betrachte das Vektorfeld
\begin{align*}
\vec{A} = \vec{r} = \begin{pmatrix}x\\y\\z\end{pmatrix},\quad \Rightarrow \div
\vec{A} = 3
\end{align*}
mit dem Volumen $V$ einer Kugel mit Radius $R$.
\begin{align*}
\Rightarrow & \int \limits_{V} \dvecr  \div \vec{A} = 3\frac{4\pi}{3} R^3
= 4\pi R^3\\
& \int \limits_{S = \partial V} \ds \vec{n}\cdot\vec{r} = \int \limits_{0}^{2\pi}\dphi \int
\limits_{0}^{\pi}\dtheta \sin(\theta) \cdot R^3 = 4\pi R^3
\end{align*}
{\em Kontinuitätsgleichung} Die Änderung der Teilchenzahl von einer Flüssigkeit
im Volumen $V$ hat die Form
\begin{align*}
\partial_t N_v(t) = \partial_t \int \limits_{V} \dvecr \varphi(\vec{r})t
= \int \limits_{V} \dvecr \partial_t \varphi(\vec{r}, t), \qquad
\varphi(\vec{r}, t) \text{ : Teilchendichte}
\end{align*}
Teilchenfluss aus dem Volumen: $\int \limits_{\partial V = S}\dvecr \cdot
\underbrace{\varphi(\vec{r},t)\vec{v}(\vec{r},t)}_{j(\vec{r},t) : \text{
Teilchenstrom}}$
\begin{align*}
\Rightarrow \partial_t N_v(t) &= \int \limits_{V} \dvecr
\partial_t\varphi(\vec{r},t) = -\int \limits_{\partial V = S}\dvecs
\vec{j}(\vec{r},t) \\
&= -\int \limits_{V}\dvecr \div j(\vec{r},t) : \text{
gilt für alle Volumen } V
\end{align*}
\begin{align*}
\Rightarrow \partial_t \varphi(\vec{r},t) + \div j(\vec{r},t) = 0
\end{align*}
\end{Beispiel}

\subsection{Integralsatz von Stokes}
Der {\em Satz von Stokes} verbindet Oberflächenintegrale mit Linienintegralen
entlang der Begrenzungslinie der Oberfläche. Für ein Vektorfeld
$\vec{A}(\vec{r})$ gilt
\begin{align*}
\int \limits_{S}\dvecs \cdot\left(\rot \vec{A}(\vec{r})\right) = \int
\limits_{C = \partial S} \dvecr \cdot \vec{A}(\vec{r}).
\end{align*}
Dabei bildet der Normalenvektor $\vec{n}$ auf der Oberfläche mit der
Umlaufrichtung der Linie $C$ eine rechtshändige Schraube.

\begin{Bemerkung}
Die Fläche soll orientierbar sein, der Satz ist nicht anwendbar auf ein
Möbiusband.
\end{Bemerkung}
\par
Für ein Vektorfeld mit $\rot \vec{A} = 0$ gilt somit, dass alle geschlossenen
Linienintegrale verschwinden.
\begin{align*}
\int \limits_{C} \dvecr \vec{A}(\vec{r}) = 0
\end{align*}
$\Rightarrow$ man kann jedoch nun zeigen, dass somit ein Skalarfeld $\phi$
existiert mit
\begin{align*}
\vec{A}(\vec{r}) = \nabla\phi(\vec{r}).
\end{align*}


\subsection{Anwendung: Coulomb Potential}
Eine homogen gefüllte Kugel mit Masse $M$ erzeugt ein Gravitationspotential
\begin{align*}
\phi(\vec{r}) = \frac{\alpha}{|r|} \qquad & : \alpha = MmG\\
& : |r| > R \text{ Radius der Kugel}\\
& : M = \frac{4\pi}{3}R^3\rho 
\end{align*}
Mittels dem Gradienten erzeugt dies das Gravitationsfeld
\begin{align*}
\vec{F}(\vec{r}) = -\nabla\phi(\vec{r}) = -\alpha \frac{\vec{r}}{|r|^3}.
\end{align*}
Weiter gilt, dass $\div \vec{F}(\vec{r}) = 0$ für $\vec{r}\neq0$. Es bleibt
somit die Frage, was bei $\vec{r} = 0$ passiert. Allerdings macht das
Resultat keinen Sinn für $|r| < R$, da sich innerhalb der Kugel das Potential
verändert. Der Satz von Gauß besagt wiederum für $S$ eine Kugel mit Radius $> R$
\begin{align*}
\int \limits_{S}\dvecs \cdot\vec{F}(\vec{r}) &= \int
\limits_{0}^{2\pi}\dphi \int \limits_{0}^{\pi}\dtheta \sin(\vartheta) \alpha \cdot
\frac{\vec{r}}{|r|}\cdot\frac{\vec{r}}{|r|^3}\cdot|r|^2 \\
&= \int
\limits_{0}^{2\pi}\dphi \int\limits_{0}^{\pi}\dtheta \sin(\vartheta) \alpha =
4\pi\alpha : \text{ unabhängig von } L\\
&\overset{\text{Satz von Gauß}}{=} \int \limits_{V}\dvecr
\div\vec{F}(\vec{r}),
\end{align*}
also muss die $\div \vec{F}$ innerhalb des
massiven Körpers gerade so sein, dass sich die Konstante $4\pi\alpha$ ergibt.
\par
Der Körper hat eine homogene Massendichte und daher lässt sich das Potential
schreiben
\begin{align*}
\phi(\vec{r}) = \int \limits_{V}\dvecr \frac{3\alpha}{4\pi
R^3}\frac{1}{|r-r'|}
\end{align*}
Für $|r| > R$ ergibt dies $\phi(\vec{r}) = \frac{\alpha}{|r|}$, aber innerhalb
des Körpers wird das Potential zu
\begin{align*}
\phi(\vec{r}) =
\begin{cases}\frac{3}{2}\frac{\alpha}{R}-\frac{\alpha}{2}\frac{|r|^2}{R^3} &
|r| < R
\\
\frac{\alpha}{|r|} & |r| > R\end{cases}
\end{align*}
und das Gravitationsfeld wird zu
\begin{align*}
\vec{F}(\vec{r}) = -\nabla\phi(\vec{r}) = 
\begin{cases}\alpha\frac{\vec{r}}{|R^3|} & |r| < R \\
\alpha\frac{\vec{r}}{|r|^3} & |r| > R\end{cases}
\end{align*}
und die Divergenz
\begin{align*}
\div \vec{F}(\vec{r}) = \begin{cases}\frac{3\alpha}{R^3} & |r| < R \\
0 & |r| > R\end{cases}
\end{align*}
Der Satz von Gauß ist somit erfüllt für den realen Fall einer homogenen Kugel.
\par
Um jetzt das Verhalten einer reinen Punktladung zu untersuchen, lassen wir den
Radius der Kugel gegen Null gehen, behalten die Masse aber konstant.
\par
Im Limes $R\to0$ konvergiert $\div \vec{F}(\vec{r})$ aber gegen eine
$3$-dimensionale $\delta$-Funktion
\begin{align*}
\div \vec{F}(\vec{r}) \underset{R\to0}{\rightarrow} 4\pi\delta{\vec{r}}
\end{align*}
\begin{info}
$\int \dvecr \div\vec{F}(\vec{r}) = 4\pi$ unabhängig von R und für $R\to0$
ist alles Gewicht in einer kleinen Kugel um den Ursprung konzentriert.
\end{info}
\par
Somit gilt für das Potential einer punktförmigen Masse
\begin{align*}
&\phi(\vec{r}) = \frac{\alpha}{|\vec{r}|}\qquad \vec{F}(\vec{r}) =
\alpha\frac{\vec{r}}{|r|^3}\\
&\Delta\phi(\vec{r}) = -4\pi\alpha\delta(\vec{r}) 
\end{align*}
\begin{Bemerkung}
In der Elektrostatik hat das Potential eines geladenen Punktteilchens
(Elektron/Proton) ebenfalls das Potential
$\phi(\vec{r})\sim\frac{1}{|\vec{r}|}$. Daher der Name {\em Coulomb Potential}.
\end{Bemerkung}

\newpage
\section{Krummlinige Koordinaten}
Die drei häufigsten Koordinatensysteme sind
\begin{align*}
&x,y,z : \text{ Kartesische Koordinaten}\\
&r,\varphi,z : \text{ Zylinder Koordinaten}\\
&r,\varphi,\vartheta : \text{ Kugel Koordinaten}\\
\end{align*}
Eine gemeinsame Eigenschaft dieser Systeme ist, dass sie orthogonal sind und,
dass der Laplace-Operator separiert. (siehe später)
\par
Die Koordinaten Transformation hat im Allgemeinen die Form
\begin{align*}
&x_1 = x_1(u_1,u_2,u_3)\\
&x_2 = x_2(u_1,u_2,u_3)\\
&x_3 = x_3(u_1,u_2,u_3)\\
&\vec{r}(u_1,u_2,u_3) = \begin{pmatrix}x_1\\x_2\\x_3\end{pmatrix}.
\end{align*}
Die Tangentenvektoren an die Koordinatenlinien sind definiert als
\begin{align*}
\vec{T}_i \equiv \vec{e}_{u_i} = \frac{1}{h_i}\frac{\partial\vec{r}}{\partial
u_i}\qquad \text{ mit } h_i = \left|\frac{\partial\vec{r}}{\partial u_i} \right|
\end{align*}
Wir sind interessiert an orthogonalen Koordinatensystemen mit
\begin{align*}
\vec{e}_{u_i} \cdot \vec{e}_{u_j} = \delta_{ij} = \begin{cases}0 &
i\neq j\\1 &i = j\end{cases}
\end{align*}
Somit bilden in jedem Punkt $\vec{r}$ die Vektoren $\vec{e}_{u_i}$ eien
Orthonormierte Basis. Die Vektoren $\vec{e}_{u_i}$ hängen somit explizit vom
Raumpunkt $\vec{r}$ ab.
\begin{Bemerkung}
Wir können auch schreiben
\begin{align*}
\vec{e}_{u_1} = \vec{e}_{u_2} \times \vec{e}_{u_3}
\end{align*}
\end{Bemerkung}
Ein Skalarfeld $\phi(\vec{r})$ können wir einfach in den neuen Koordinaten
ausdrücken
\begin{align*}
\phi(u_1,u_2,u_3) = \phi(\vec{r}(u_1,u_2,u_3)).
\end{align*}
Für ein Vektorfeld $\vec{A}(\vec{r})$ müssen wir zusätzlich die neue Basis
$\vec{e}_{u_i}$ berücksichtigen.
\begin{align*}
\vec{A}(\vec{r}) = \sum \limits_{i=1}^{3} \vec{e}_{u_i} A_{u_i}(u_1,u_2,u_3).
\end{align*}
$A_{u_i}$ erhalten wir mittels dem Skalarprodukt
\begin{align*}
A_{u_i} = \vec{A}\cdot\vec{e}_{u_i}.
\end{align*}
\begin{Bemerkung}
In den krummlinigen Koordinaten sind jetzt $A_{u_i}$ und $\vec{e}_{u_i}$ von
$u_1,u_2,u_3$ abhängig.
\begin{align*}
\Rightarrow \partial_{u_j} \vec{A} = \sum \limits_{i=1}^{3}
\left[\frac{\partial A_{u_i}}{\partial u_j} + A_{u_i} \frac{\partial
\vec{e}_{u_i}}{\partial u_j}\right]
\end{align*}
\end{Bemerkung}
\begin{Beispiel}
Geschwindigkeitsvektor $\vec{v}$ in Zylinder-Koordinaten
\begin{align*}
\vec{v}(t) = \frac{d}{dt}\vec{r}(t) = \frac{d}{dt}\left(r\vec{e}_r +
z\vec{e}_z\right) = \dot{r}\vec{e}_r + r\dot{\vec{e}}_r + \dot{z}\vec{e}_z +
z\dot{\vec{e}}_z
\end{align*}
\end{Beispiel}
Im Folgenden wollen wir untersuchen, wie sich die Ableitungsoperatoren
transformieren.
\begin{Definition}[Gradient]
Die Komponente von $\nabla\phi$ in $\vec{e}_{u_i}$ ist
\begin{align*}
(\nabla\phi)_{u_i} &= \nabla\phi \cdot \vec{e}_{u_i} =
\nabla\phi\cdot\frac{1}{h_{u_i}}\frac{\partial\vec{r}}{\partial u_i} =
\frac{1}{h_{u_i}} \sum \limits_{j=1}^{3}\frac{\partial x_j}{\partial u_i}
\frac{\partial \phi}{\partial x_j} \\
&= \frac{1}{h_{u_i}}
\partial_{u_i}\phi(\vec{r}(u_1,u_2,u_3)) =
\frac{1}{h_{u_i}}\frac{\partial\phi}{\partial u_i}\\
\Rightarrow  \nabla \phi &= \sum \limits_{i} \vec{e}_{u_i} \frac{1}{h_i}
\frac{\partial \phi}{\partial u_i}
\end{align*}
\end{Definition}

\begin{Definition}[Divergenz]
Für die Divergenz in krumlinigen Koordianten gilt
\begin{align*}
\begin{split}
\div \vec{A}(u_1,u_2,u_3) = \frac{1}{h_{u_1} h_{u_2}
h_{u_3}}\big[\frac{\partial}{\partial u_1} \left(h_{u_2}h_{u_3}A_{u_1}\right)
+ \frac{\partial}{\partial u_2}\left(h_{u_1} h_{u_3} A_{u_2}\right) \\ +
\frac{\partial}{\partial u_3} \left(h_{u_1} h_{u_2} A_{u_3}\right) \big]
\end{split}
\end{align*}
\end{Definition}

\begin{info}
Betrachte den Anteil $\vec{e}_{u_i} A_{u_1}$
\begin{align*}
\nabla\cdot(\vec{e}_{u_1} A_{u_1}) &=
\nabla\cdot((\vec{e}_{u_2} \times \vec{e}_{u_3})A_{u_1}) \\
&= \nabla\cdot(h_{u_2} h_{u_3} A_{u_1} (\nabla u_2 \times \nabla u_3))\\
&= \nabla(h_{u_2} h_{u_3} A_{u_1})\cdot(\nabla u_2 \times \nabla u_3) \\ &
\quad + h_{u_2} h_{u_3} A_{u_1} \nabla \cdot (\nabla u_2 \times \nabla u_3)\\
&= \frac{1}{h_{u_2} h_{u_3}}\vec{e}_{u_1}\cdot\nabla(h_{u_2} h_{u_3} A_{u_1})\\
&= \frac{1}{h_{u_1} h_{u_2} h_{u_3}} \frac{\partial}{\partial u_1} (h_{u_2}
h_{u_3} A_{u_1})
\end{align*}
und analog für die anderen Komponenten.
\end{info}

\begin{Definition}[Laplace Operator]
Als wichtige Anwendung erhalten wir den Laplace Operator in krummlinigen
Koordinaten
\begin{align*}
\nabla\phi(u_1,u_2,u_3) = \frac{1}{h_{u_1} h_{u_2}
h_{u_3}}\big[\frac{\partial}{\partial u_1}\left(\frac{h_{u_2}
h_{u_3}}{h_{u_1}} \frac{\partial \phi}{\partial u_1}\right) +
\frac{\partial}{\partial u_2}\left(\frac{h_{u_1}
h_{u_3}}{h_{u_2}} \frac{\partial \phi}{\partial u_2}\right) \\+
\frac{\partial}{\partial u_3}\left(\frac{h_{u_1}
h_{u_2}}{h_{u_3}} \frac{\partial \phi}{\partial u_3}\right)
 \big]
\end{align*}
\end{Definition}

\begin{Definition}[Rotaion]
Zur Vollständigkeit noch die Rotaion
\begin{align*}
\rot \vec{A}(u_1,u_2,u_3) &= \frac{1}{h_{u_2}
h_{u_3}}\vec{e}_{u_1}\left[\frac{\partial}{\partial u_2}\left(h_{u_3}
A_{u_3}\right) - \frac{\partial}{\partial u_3}\left(h_{u_2}
A_{u_2}\right)\right]\\
& + \frac{1}{h_{u_1}
h_{u_3}}\vec{e}_{u_2}\left[\frac{\partial}{\partial u_3}\left(h_{u_1}
A_{u_1}\right) - \frac{\partial}{\partial u_1}\left(h_{u_3}
A_{u_3}\right)\right]\\
& + \frac{1}{h_{u_1}
h_{u_2}}\vec{e}_{u_3}\left[\frac{\partial}{\partial u_1}\left(h_{u_2}
A_{u_2}\right) - \frac{\partial}{\partial u_2}\left(h_{u_1}
A_{u_1}\right)\right]
\end{align*}
\end{Definition}

\subsection{Zylinderkoordinaten}
Wir haben die Transformation
\begin{align*}
\vec{r} = \begin{pmatrix}x\\y\\z\end{pmatrix} = \begin{pmatrix}r\cos(\varphi)
\\ r\sin(\varphi) \\ z\end{pmatrix}
\end{align*}
mit $0\le r<\infty, 0\le\phi<2\pi, -\infty<z<\infty$
\begin{align*}
&\vec{e}_r = \begin{pmatrix}\cos(\varphi) \\ \sin(\varphi) \\ 0\end{pmatrix}
\qquad &h_r = 1\\
&\vec{e}_\varphi = \begin{pmatrix}-\sin(\varphi) \\ \cos(\varphi) \\
0\end{pmatrix} \qquad &h_\varphi = r\\
&\vec{e}_z = \begin{pmatrix}0 \\ 0 \\ 1\end{pmatrix}
\qquad &h_z = 1\\
\end{align*}
\begin{align*}
\nabla\phi &= \vec{e}_r \frac{\partial}{\partial r}\phi +
\frac{1}{r}\vec{e}_\varphi\frac{\partial}{\partial \varphi}\phi + \vec{e}_z
\frac{\partial}{\partial z}\phi\\
\Delta\phi &= \left[\frac{1}{r}\partial_r r \partial_r +
\frac{1}{r^2}\partial^2_\varphi + \partial^2_z \right]\phi \\
&= \partial^2_r\phi + \frac{1}{r}\partial_r\phi +
\frac{1}{r^2}\partial^2_\varphi\phi + \partial^2_z\phi\\
\int \ dr^3\ &= \int \dphi \int \dr r \int \dz
\end{align*}

\subsection{Kugelkoordinaten}
\begin{align*}
\vec{r} = \begin{pmatrix}x \\ y \\ z\end{pmatrix} =
\begin{pmatrix}r\cos(\varphi)\sin(\vartheta) \\ r\sin(\varphi)\sin(\vartheta) \\
r\cos(\vartheta)\end{pmatrix}
\end{align*}
mit $0 \le r < \infty$, $0 \le \vartheta \le \pi$, $0 \le \varphi < 2\pi$.
\begin{align*}
\vec{e}_r = \begin{pmatrix}\cos(\varphi)\sin(\vartheta) \\
\sin(\varphi)\sin(\vartheta) \\ \cos(\vartheta)\end{pmatrix},\qquad h_r = 1
\end{align*}
\begin{align*}
\vec{e}_\vartheta = \begin{pmatrix}\cos(\varphi)\cos(\vartheta) \\
\sin(\varphi)\cos(\vartheta) \\ -\sin(\vartheta)\end{pmatrix},\qquad h_\vartheta
= r
\end{align*}
\begin{align*}
 \vec{e}_\varphi = \begin{pmatrix}-\sin(\varphi) \\ \cos(\varphi) \\
 0\end{pmatrix},\qquad h_\varphi = r\sin(\vartheta)
\end{align*}
\begin{align*}
\nabla\phi = \left[ \vec{e}_r\partial_r\phi +
\frac{1}{r}\vec{e}_\vartheta\partial_\vartheta\phi +
\frac{1}{r\sin(\vartheta)}\vec{e}_\varphi\partial_\varphi\phi \right]
\end{align*}
\begin{align*}
\Delta\phi = \left[ \frac{1}{r^2}\partial_r(r^2\partial_r) +
\frac{1}{r^2\sin(\vartheta)}\partial_\vartheta
(\sin(\vartheta)\partial_\vartheta) +
\frac{1}{r^2\sin^2(\vartheta)}\partial^2_\varphi \right]\phi
\end{align*}
\begin{align*}
\int \dvecr = \int \dr r^2 \int \dtheta \sin(\vartheta) \int \dphi
\end{align*}
\newpage
\section{Fourierreihe und Fouriertransformation}
\subsection{Fourierreihe}
Im Folgenden sind wir an periodischen Funktionen $f(x)$ interessiert, mit
Periode $L$, d.h.
\begin{align*}
f(x + L) = f(x)
\end{align*}
Insbesondere dürfen die Funktionen auch komplexwertig sein, und sollen die {\em
Dirichlet Bedingung} erfüllen.
\begin{enumerate}
  \item $|f(x)|$ ist integrierbar
  \item $f(x)$ ist stückweise stetig
  \item $f(x)$ hat endlich viele Extrema
  \item $f(x) = \lim \limits_{\varepsilon\to0}
  \frac{1}{2}\left[f(x+\varepsilon) + f(x-\varepsilon)\right]$
\end{enumerate}
Ein spezielles Set von solchen Funktionen ist
\begin{align*}
f_n(x) = e^{i\ k_n\ x}\qquad k_n = \frac{2\pi n}{L},\quad n\in\Z.
\end{align*}
Insbesondere gilt
\begin{align*}
\frac{1}{L} \int \limits_{0}^{L} \dx f_n(x) f_m^*(x) = \frac{1}{L} \int
\limits_{0}^{L} \dx e^{i\ \frac{2\pi}{L}\ x(n-m)} = \delta_{n,m}.
\end{align*}
\begin{Definition}[Fourierkoeffizienten]
Die Fourierkoeffizienten einer Funktion $f(x)$ sind definiert als
\begin{align*}
\hat{f}(n) = \frac{1}{L} \int \limits_{0}^{L} e^{-i\ k_n\ x} f(x) \equiv c_n.
\end{align*}
\end{Definition}
\begin{Definition}[Fourierreihe]
Die Fourierreihe hat somit die Form
\begin{align*}
\sum \limits_{n = -\infty}^{\infty} \hat{f}(n) e^{i\ k_n\ x},
\end{align*}
und es gilt, dass diese Reihe konvergiert und die Funktion identisch zu $f(x)$
ist, d.h. wir können $f(x)$ darstellen als
\begin{align*}
&f(x) = \sum \limits_{n = -\infty}^{\infty} \hat{f}(n) e^{i\ k_n\ x}\quad k_n
= \frac{2\pi n}{L}\\
&\hat{f}(n) = \frac{1}{L} \int \limits_{0}^{L} \dx e^{-i\ k_n\ x} f(x).
\end{align*}
Somit können wir jede Funktion in {\em elementare Schwingungen} zerlegen. 
\end{Definition}
\begin{Bemerkung}
$e^{ikx} = \cos(kx) + i\sin(kx)$ und somit haben wir die Funktion in $\sin$ und
$\cos$ zerlegt.
\end{Bemerkung}
\begin{Bemerkung}
Alternative Form der Fourierreihe
\begin{align*}
f(x) &= \sum \limits_{n=-\infty}^{\infty} \hat{f}(n)e^{i k_n x} = \sum
\limits_{n = -\infty}^{\infty} \hat{f}(n) \cos(k_n x) + i\hat{f}(n) \sin(k_n
x)\\
&= \underbrace{\hat{f}(0)}_{\frac{a_0}{2}} + \sum \limits_{n=1}^{\infty}
\underbrace{\left(\hat{f}(n) + \hat{f}(-n) \right)}_{a_n} \cos(k_n x) + i\sum
\limits_{n=1}^{\infty} \underbrace{\left( \hat{f}(n) - \hat{f}(-n)
\right)}_{b_n} \sin(k_n x) \\
&= \frac{a_0}{2} + \sum\limits_{n=1}^{\infty} a_n
\cos(k_n x) + \sum \limits_{n=1}^{\infty} b_n \sin(k_n x)
\end{align*}
\begin{align*}
\text{mit } & a_n = \frac{2}{L} \int \limits_{0}^{L} \dx \cos(k_n x)f(x)\\
&b_n = \frac{2}{L} \int \limits_{0}^{L} \dx \sin(k_n x)f(x)
\end{align*}
\end{Bemerkung}
\begin{Bemerkung}
\par
\begin{itemize}
  \item Falls $f(x)$ reell ist, so gilt $\hat{f}(n) = \hat{f}^*(-n)$ $(a_n,
  b_n \in \R)$
  \item Falls $f(x)$ symmetrisch ist, d.h. $f(x) = f(-x)$, so gilt $\hat{f}(n)
  = \hat{f}(-n)$ $(b_n = 0)$
  \item Falls $f(x)$ symmetrisch und reell ist, so ist $\hat{f}(n)$ symmetrisch
  und reell. $(a_n \in \R, b_n = 0)$
\end{itemize}
\end{Bemerkung}
Die Idee für den Beweis für den obigen Satz hat die Form
\begin{align*}
F_N(x) &= \sum \limits_{n=-N}^{N} \hat{f}(n) e^{i k_n x} = \frac{1}{L} \int
\limits_{0}^{L} \dy \sum \limits_{n=-N}^{N} e^{i k_n (x-y)} f(y) = \int
\limits_{0}^{L} \dy D_N (x-y) f(x)
\end{align*}
mit
\begin{align*}
D_N(x-y) \equiv
\frac{1}{L}\frac{\sin\left(\frac{2\pi(N+\frac{1}{2})(x-y)}{L}\right)}{\sin\left(\frac{\pi(x-y)}{L}\right)}
= e^{-i N \frac{2\pi}{L}(x-y)}\frac{1-
\left(e^{i\frac{2\pi}{L}(x-y)}\right)^{2N+1}}{1-e^{i\frac{2\pi}{L}(x-y)}}
\end{align*}
Die Funktion $D_N(x)$ gleicht aber der Funktionenreihe, die gegen eine
$\delta$-Funktion konvergiert, d.h.
\begin{align*}
\int \limits_{0}^{L} \dy D(x-y) = 1.
\end{align*}
Formal gilt daher
\begin{align*}
D_N(x-y) \underset{N\to\infty}{\rightarrow} \sum \limits_{j=-\infty}^{\infty}
\delta\left(x-y + jL\right),
\end{align*}
und somit
\begin{align*}
F_N(x) \overset{N\to\infty}{\rightarrow} \int \limits_{0}^{L} \dy \sum
\limits_{j=-\infty}^{\infty} \delta\left(x-y + jL\right)f(y) = f(x)
\end{align*}
\begin{Bemerkung}
Die Relation
\begin{align*}
\sum \limits_{n=-\infty}^{\infty} e^{i k_n x} = \sum
\limits_{j=-\infty}^{\infty} \delta\left(x + jL\right)
\end{align*}
wird in der Physik oft verwendet.
\end{Bemerkung}
\begin{Bemerkung}
Für die Fourierkoeffizienten gilt die Gleichung
\begin{align*}
\frac{1}{L} \int \limits_{0}^{L} \left|f(x)\right|^2 = \sum
\limits_{n=-\infty}^{\infty} \left| \hat{f}(n) \right|^2 \text{ : Satz von
Parseval}
\end{align*}
\end{Bemerkung}
\end{document}