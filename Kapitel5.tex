\section{Gewöhnliche Differentialgleichungen}
Differentialgleichungen treten in der Physik sehr häufig auf

\begin{Beispiel}[Radioaktiver Zerfall]
Die Anzahl Atome, die in einem Zeitintervall $\Delta t$ zerfallen, ist
proportional zur Anzahl Teilchen und der Rate $\Gamma$.
\begin{align}
&\Delta N = -N(t)\;\Gamma \cdot\Delta t\\
\underset{\Delta t \to 0}{\Rightarrow} & \frac{dN(t)}{dt} = -\Gamma
N(t)\nonumber
\end{align}
Die DG hat die Lösung $N(t) = N_0 e^{-\Gamma t}$ mit $N_0$ als Atomzahl zur
Zeit $t = 0$.
\end{Beispiel}

\begin{Beispiel}[Harmonischer Oszillator]
\begin{align*}
& m\ddot{x}(t) = F = -m\omega^2x\\
\Rightarrow & \ddot{x}(t) + \omega^2 x = 0
\end{align*}
Die DG hat die Lösung $x(t) = Ae^{i\omega t} + Be^{-i\omega t}$. Da $x(t)$
reell ist, muss gelten $A^* = B = \frac{x_0}{2}e^{i\varphi}$ und die
allgemeinste Lösung hat die Form
\begin{equation*}
x(t) = x_0 \cos(\omega t + \varphi)
\end{equation*}
mit $x_0$ und $\varphi$ bestimmt durch die Anfangs-/Randbedingungen.
\begin{center}
\psset{unit=1cm}
\begin{pspicture}(-6,-1)(6,6)
 \psline[linewidth=0.5pt,arrowsize=4pt]{->}(-6,0)(6,0)
 \psline[linewidth=0.5pt,arrowsize=4pt]{->}(0,-1)(0,6)
 
 \psplot[linewidth=1.2pt,algebraic=true]{-3}{3}{(1/2)*x^2}
 \psdot[linewidth=0.1cm,dotsize=0.5cm](2,2.8)
 \psline{->}(1.75,2.4)(1.4,1.7)

 \rput(5.7,-0.26){$x$}
 \rput[l](0.2,5.7){$v(x) = \frac{1}{2}x^2$}
 \end{pspicture}
\end{center}

\end{Beispiel}


\subsection{Lineare Differentialgleichungen 1. Ordnung}
Lineare DG 1. Ordnung haben die Form
\begin{equation}
y' = a(x)y + b(x),
\end{equation}
und benötigen im Allgmeinen eine Anfangsbedingung
\begin{equation}
y(0) = y_0,
\end{equation}
um eine spezielle Lösung zu bestimmen.

\subsubsection{Der triviale Fall}
Ist $a(x) = 0$, so ist die Lösung bestimmt durch das
Integral von $b(x)$.
\begin{equation}
y(x) = \int \limits_{0}^{y} \du b(u) + y_0
\end{equation}

\begin{Beispiel}[Geschwindigkeit eines Teilchens mit zeitabhängiger Kraft]
\begin{equation*}
\dot{v} = \frac{F(t)}{m} \Rightarrow v = \int \limits_{0}^{t} \frac{F(t)}{m} dt
+ v_0
\end{equation*}
\end{Beispiel}

\subsubsection{Der homogene Fall}
Eine homogene Differentialgleichung hat die Form
\begin{equation}
y' = a(x)y.
\end{equation}

\begin{Bemerkung}
 Eine Eigenschaft von linearen homogenen Differentialgleichungen ist, dass für die Lösungen $y_1, y_2$ auch folgende Funktionen Lösungen sind:
\begin{itemize}
 \item $\lambda y_1(x)$
 \item $y_1(x)+y_2(x)$
\end{itemize}
\end{Bemerkung}

Die Lösung der Differentialgleichung finden wir mittels Division von $y$:
\begin{align*}
 & \frac{y'}{y} = \frac{d}{dx} \ln{y} = a(x)\\
\Rightarrow & \ln{y(x)} = \int \limits_{0}^{x} \du a(u) + c \equiv A(x)+c\\
\Rightarrow & y(x) = y_0 e^{\int_0^{x}\du a(u)} = y_0 e^{A(x)}
\end{align*}

\subsubsection{Der inhomogene Fall}
Zusätzlich zur homogenen Gleichung haben wir noch einen Treiber $b(x)$.
\begin{align}
 y' - a(x)y = b(x)
\end{align}
Für zwei Lösungen $y_1(x)$ und $y_2(x)$ gilt, dass
\begin{align*}
 y_1(x)-y_2(x)
\end{align*}
eine Lösung der homogenen Gleichung ist.
\par
\begin{info}
\begin{align*}
&\left[y_1(x) - y_2(x)\right]' - a(x)\left[y_1(x) - y_2(x)\right] \\
= & \underbrace{\left[y_1'(x) - a(x)y_1(x)\right]}_{b(x)} -
\underbrace{\left[y_2'(x) - a(x)y_2(x)\right]}_{b(x)} \\
 = & b(x) - b(x) = 0
\end{align*}
\end{info}

\par
Eine Lösung erhält man mittels Multiplikation der DG mit $e^{-A(x)}$
\begin{align}
\Rightarrow & e^{-A(x)}y' - \underbrace{a(x)e^{-A(x)}}_{(e^{-A(x)})'}y =
b(x)e^{-A(x)}\nonumber\\
\Rightarrow & \frac{d}{dx}\left(ye^{-A(x)}\right) = b(x)e^{-A(x)}\nonumber\\
\Rightarrow & y(x) = e^{A(x)} \underbrace{\int
\limits_{0}^{x} \du b(u)e^{-A(u)}}_{\text{partikuläre Lösung}} +
\underbrace{y_0e^{A(x)}}_{\text{homogene Lösung}}
\label{TrickDgLsg}
\end{align}
\begin{Beispiel}[Beschleunigter Massenpunkt mit linearem Luftwiderstand]
\begin{align*}
m\dot{v} = \underbrace{F}_{\text{Beschleunigung}} - \underbrace{R\cdot
v}_{\text{Luftwiderstand}}\\
v =
\underbrace{\frac{F}{R}}_{v_\infty}(1-e^{-t\frac{R}{m}})
+ v_0 e^{-t\frac{R}{m}}
\end{align*}
\begin{center}
\psset{unit=1cm}
\begin{pspicture}(-1,-1)(10,6)
 \psline[linewidth=0.5pt,arrowsize=4pt]{->}(-1,0)(10,0)
 \psline[linewidth=0.5pt,arrowsize=4pt]{->}(0,-1)(0,6)
 
 \psline[linewidth=0.5pt,arrowsize=4pt,linestyle=dashed](0,4)(10,4)
 \psline[linewidth=0.5pt,arrowsize=4pt,linestyle=dashed](2,0)(2,3.46)
 
 \psplot[linewidth=1.2pt,algebraic=true]{0}{10}{4-4*(2.72)^(-x)}
 
 
 \rput(-0.3,-0.3){$0$}
 \rput(2,-0.3){$m/R$}
 
 \rput[r](-0.2,4){$v_\infty$}
 
 \rput(9.7,-0.26){$t$}
 \rput[r](-0.2,5.7){$v(t)$}
 
\end{pspicture}
\end{center}
\end{Beispiel}

\subsection{Nichtlineare Differentialgleichung 1. Ordnung}
Die allgemeine Form lautet
\begin{align}
 \frac{dy}{dx} =  F(x,y),
\end{align}
und ist im Allgemeinen nicht geschlossen lösbar, außer in Spezialfällen, von
denen wir einige untersuchen wollen.

\subsubsection{Separierbare Differentialgleichungen}
Falls $F(x,y)$ separierbar ist, d.h.
\begin{align*}
 F(x,y) = f(x) g(y),
\end{align*}
so können wir die Differentialgleichung umschreiben auf
\begin{align*}
 & \frac{y'}{g(y)} = f(x)\\
  \overset{\text{integrieren mit }x}{\Rightarrow} &
\int \dx \frac{y'}{g(y)} = \underbrace{\int \dy \frac{1}{g(y)}}_{G(y)} = \underbrace{\int \dx f(x)}_{H(x)+c}
\end{align*}
Falls $G(y)$ invertierbar ist, erhalten wir
\begin{align*}
 y(x) = G^{-1}(H(x)+c)
\end{align*}

\begin{Beispiel}
Kettengleichung: Freihängendes Seil/Kette
\begin{align*}
y'' &= \alpha \sqrt{1+(y')^2}
\end{align*}
\begin{center}
%\psset{unit=1cm}
\begin{pspicture}(-6,-1)(6,6)
 \psline[linewidth=0.5pt,arrowsize=4pt]{->}(-6,0)(6,0)
 \psline[linewidth=0.5pt,arrowsize=4pt]{->}(0,-1)(0,6)
 
 \psplot[linewidth=1.2pt,algebraic=true]{-2.2}{0}{-0.2*x^3+2}
 \psplot[linewidth=1.2pt,algebraic=true]{0}{2.2}{0.2*x^3+2}
 \psline[linewidth=0.5pt,algebraic=true]{<->}(1,1.2)(2.5,4.8)
 \psline[linewidth=0.5pt,arrowsize=4pt,linestyle=dashed]{->}(2,3.6)(3,3.6)
 \psline[linewidth=0.5pt,arrowsize=4pt,linestyle=dashed]{->}(2,2.6)(2,4.6)
 \psline[linewidth=0.5pt,arrowsize=4pt,linestyle=dashed]{->}(2,2.6)(2,4.6)
 \psline[linewidth=0.5pt,arrowsize=4pt,linestyle=dashed](1,2.6)(2,2.6)
 \psline[linewidth=0.5pt,arrowsize=4pt,linestyle=dashed](1,2.6)(1,1.2)
 
 \rput(5.7,-0.26){$x$}
 
 \rput[r](1.9,4.6){$F_g$}
 \rput[l](3.1,3.8){$F_\perp$}
 \rput[l](1,2.9){$F_\perp$}
 \rput[r](0.9,1.4){$F_0$}
 \rput[l](2.05,2.8){$\dy$}
 \rput[lt](1.5,2.5){$\dx$}
 
 \rput[l](0.2,5.7){$v(x) = \frac{1}{2}x^100$}
 \end{pspicture}
\end{center}
Setze $u=y'$
\begin{align*}
& \frac{u'}{\sqrt{1+u^2}} = \alpha\\
\Rightarrow & \int du \frac{1}{\sqrt{1+u^2}} = \int \dz
\frac{\cosh(z)}{\sqrt{1+\sinh^2(z)}} = \int \dz \frac{\cosh(z)}{\cosh(z)} \\
 & = z = 
\arcsinh{u} = \int \dx \alpha = \alpha x + c_1\\ \Rightarrow & y' = u =
\sinh(\alpha x + c_1)\\ \Rightarrow & y = \frac{1}{\alpha} \cosh(\alpha x + c_1)
+
\tilde{c}
\end{align*}
\end{Beispiel}

\par{\bf Bernoulli-Gleichung}
\begin{equation*}
y' = a(x)y + b(x)y^{\nu}
\end{equation*}
Mit einer geschickten Substitution kann diese Gleichung auf die Form einer
inhomogenen Differentialgleichung gebracht werden.
\begin{align*}
\frac{y'}{y^{\nu}} = a(x)y^{1-\nu}+b(x)\\
\frac{1}{1-\nu}\left(y^{1-\nu}\right)' = \frac{v'}{1-\nu}
\end{align*}
Setze $y^{1-\nu} = v$ und man erhält
\begin{equation*}
v' = (1-\nu)a(x)v+(1-\nu)b(x)
\end{equation*}
$\Rightarrow$ Löse mit dem Trick von Gleichung \ref{TrickDgLsg}


\subsection{Lineare Differentialgleichungen höherer Ordnung}
Die allgemeine Form ist
\begin{align}
 \sum \limits_{i=0}^{n} a_i(x)y^{(i)}(x) = b(x).
\end{align}
Die Lösung der Differentialgleichung ist eindeutig bestimmt durch das Anfangswertproblem
\begin{align}
 y(0) = y_0, y^{(1)}(0) = y_0^{(1)}, ..., y^{(n-1)}(0) = y_0^{(n-1)}
\end{align}

\begin{Beispiel}[Differentialgleichung 2. Ordnung]
\begin{align}
 \underbrace{L}_{\text{Induktion}} \ddot{I} + \underbrace{R}_{\text{Widerstand}}\dot{I} + \frac{1}{\underbrace{C}_{\text{Kapazität}}}I = \underbrace{\frac{dV}{dt}}_{\text{ext. Drive}} \; \text{ : elektrischer Schwingkreis}
\end{align}
ist durch die Vorgabe von $I(0)$ und $\dot{I}(0)$ bestimmt.
\end{Beispiel}

Allgemeiner Lösungsansatz
\begin{itemize}
  \item Finde $n$ unabhängige Lösungen $y_1(x), ..., y_n(x)$ für die homogene
  Gleichung mit $b(x) = 0$.
  \begin{align*}
  \Rightarrow y_c(x) = C_1\,y_1(x)+ C_2\,y_2(x) + ... + C_n\,y_n(x)
  \end{align*}
  Lösung der homogenen Gleichung.
  \item Finde eine partikuläre Lösung $y_p(x)$ der inhomogenen Gleichung mit
  $b(x) \neq 0$.
  \item Die allgemeine Lösung hat die Form
  \begin{align}
  y(x) = y_c(x) + y_p(x)
  \end{align}
\end{itemize}

\subsubsection{Konstante Koeffizienten}
{\bf Homogene Gleichung}
\begin{align}
 a_n \frac{d^ny}{dx^n} + a_{n-1} \frac{d^{n-1}y}{dx^{n-1}} + ... + a_1\frac{dy}{dx} + a_0y = 0
\end{align}

Mit dem Ansatz $y = e^{\lambda x}$ mit $\lambda \in \Co$ können wir die Differentialgleichung in eine algebraische Gleichung umwandeln:
\begin{align*}
 a_n\lambda^n + a_{n-1}\lambda^{n-1} + \ldots + a_1\lambda + a_0 = 0
\end{align*}

Die Gleichung hat genau $n$ Lösungen $\lambda_1 \ldots \lambda_n$. Es ist jetzt nötig, 3-Fälle zu unterscheiden:
\begin{enumerate}
  \item Alle $\lambda_i$ sind reell und verschieden. Somit sind $y_i(x) =
  e^{\lambda ix}$ linear unabhängig und die Lösung hat die Form
  \begin{align}
  y_c(x) = C_1e^{\lambda_1 x}+ C_2e^{\lambda_2 x}+\ldots+ C_ne^{\lambda_n x}
  \end{align}
  \item Einige $\lambda_i$ sind komplex. Falls $a_i$ reell sind, so ist
  $\overline{\lambda_i}$ ebenfalls eine Lösung. So können wir schreiben
  $(\lambda_i = \alpha+i\beta)$
  \begin{align}
  y_i(x) &= C_i e^{(\alpha + i\beta)x} + \overline{C_i} e^{(\alpha - i\beta)x}\\
  &= 2A e^{\alpha x}\cos{(\beta x + \varphi)}
  \end{align}
  \item Einige Lösungen sind mehrfache Nullstellen $\lambda_1 = \lambda_2 =
  \ldots = \lambda_k \equiv \lambda$. Dann sieht man, dass die Funktionen
  \begin{align*}
  e^{\lambda x}, xe^{\lambda x}, x^2e^{\lambda x}, \ldots, x^{k-1}e^{\lambda x}
  \end{align*}
  ebenfalls Lösungen sind. Wir erhalten somit wieder $n$ unabhängige Lösungen.
  \begin{align}
  y_c = (C_1+C_2x+\ldots+C_kx^{k-1})e^{\lambda x} +
  C_{k+1}e^{\lambda_{k+1}x}+\ldots+C_ne^{\lambda_nx}
 \end{align}
\end{enumerate}

\begin{Beispiel}
\begin{align*}
&y''(x) - 2y' +y = 0\quad \text{ Ansatz } y = e^{\lambda x} \\
\Rightarrow & (\lambda^2 -2\lambda +1)e^{\lambda x} = 0\\
\Rightarrow & 0 = \lambda^2 -2\lambda +1 = (\lambda -1)^2\\
\Rightarrow & e^x, xe^x \text{ sind Lösungen}
\end{align*}
\par
\begin{info}
$(xe^x)'' - 2(xe^x)' + xe^x = xe^x+2e^x-2(xe^x+e^x)+xe^x = 0$
\end{info}
\par
Die Lösung ist somit $y_c(x) = (C_1+xC_2)e^x$.
\end{Beispiel}

{\bf Inhomogene Gleichung}
\begin{align}
 a_n\frac{d^ny}{dx^n} + \ldots + a_1\frac{dy}{dx} + a_0 = b(x)
\end{align}

Es gibt keine allgemeine Methode, die eine partikulär Lösung liefert. Für
spezielle $b(x)$ helfen aber folgende Ansätze.
\begin{enumerate}
  \item $b(x) = Ae^{\Gamma x}$ mit $\Gamma$ reell oder komplex\\
  Ansatz: $y_p(x) = Be^{\Gamma x}$
  \item $b(x) = A_1\sin(\Gamma x) + A_2\cos(\Gamma x)$\\
  Ansatz: $y_p(x) = B_1\sin(\Gamma x) + B_2\cos(\Gamma x)$
  \item $b(x) = A_0 + A_1x+\ldots+A_nx^n$\\
  Ansatz: $y_p(x) = B_0+B_1x+\ldots+B_nx^n$
  \item Falls $b(x)$ eine Summe oder ein Produkt von obigen Formen ist, so ist der Ansatz ebenfalls eine Summe oder Produkt der entsprechenden Ansätze.
\end{enumerate}

\begin{Beispiel}
 \begin{align*}
  & y'' + y = \cos{2x}\\
&\text{Ansatz: } y_p(x) = B_1\cos{2x} + B_2\sin{2x}\\
\Rightarrow & -B_1 4\cos{2x} - B_2 4\sin{2x} + B_1\cos{2x} + B_2\sin{2x} = \cos{2x}
 \end{align*}
\begin{align*}
 \Rightarrow & -4B_1 +B_1 = 1 & \Rightarrow  & B_1 = -\frac{1}{3}\\
&-4B_2 +B_2 = 0 & & B_2 = 0\\
\end{align*}
\begin{align*}
 \Rightarrow & y_p(x) = -\frac{1}{3}\cos{2x} & \text{ : partikulär Lösung}\\
 \Rightarrow & y_c(x) = C_1\cos{x}+C_2\sin{x} & \text{ : homogene Lösung}
\end{align*}
\begin{align*}
 \Rightarrow & \text{Vollständige Lösung:}\\
  & y(x) = y_p(x)+y_c(x) = -\frac{1}{3}\cos{2x}+C_1\cos{x}+C_2\sin{x}
\end{align*}
\end{Beispiel}

\begin{Bemerkung}
 Die allgemeine Lösung folgt aus der Summe der partikulär Lösung $y_p$ und der
 Lösung der homogenen Gleichung $y_c$.
\begin{align}
 y(x) = y_p(x) + y_c(x)
\end{align}
\end{Bemerkung}

\subsection{Green'sche Funktion}
Betrachte die inhomogene Differentialgleichung
\begin{align}
\frac{d^2}{dx^2}y + a\frac{d}{dx}y + by = f(x),
\end{align}
für einen allgemeinen Treiber $f(x)$. Die Lösung des homogenen Problems kennen
wir.

\par{\bf Frage} Gibt es einen speziellen Treiber $h(x,z)$, sodass wir mit einer
Lösung von
\begin{align}
\frac{d^2}{dx^2}G(x,z) + a\frac{d}{dx}G(x,z) + bG(x,z) = h(x,z)
\end{align}
eine Lösung zur obigen Gleichung finden für einen beliebigen Treiber $f(x)$:
\begin{align}
y(x) = \int \limits_{-\infty}^{\infty} \dz G(x,z)f(z)
\end{align}

\begin{Bemerkung}
Der Ansatz $y(x) = \int_{-\infty}^{\infty} \dz G(x,z)f(z)$ folgt aus der
Linearität der Differentialgleichung:
\begin{itemize}
  \item Ist $y(x)$ Lösung zum Treiber $f(x)$\\
  $\Rightarrow cy(x)$ ist Lösung zum Treiber $cf(x)$
  \item Ist $y_1(x)$ Lösung zu $f(x)$ und $y_2(x)$ Lösung zu $g(x)$\\
  $\Rightarrow y_1(x)+ y_2(x)$ ist Lösung zum Treiber $f(x) + g(x)$
\end{itemize}
\end{Bemerkung}

Einsetzen von $y(x) = \int \limits_{-\infty}^{\infty}\dz G(x,z)f(z)$ in die
Differentialgleichung liefert
\begin{align}
f(x) &= \frac{d^2}{dx^2}y(x) + a\frac{d}{dx}y(x) + by(x) \\
&= \int
\limits_{-\infty}^{\infty}\dz f(z)\left[
\underbrace{\frac{d^2}{dx^2}G(x,z)+
a\frac{d}{dx}G(x,z) + bG(x,z)}_{h(x,z)} \right]\\
&= \int \limits_{-\infty}^{\infty} \dz f(z) h(x,z)
\end{align}
d.h., der spezielle Treiber $h(x,z)$ muss die Eigenschaft haben
\begin{align}
\int \limits_{-\infty}^{\infty}f(z)h(x,z) = f(x)
\end{align}

Die Funktion, die diese Eigenschaft besitzt wird in der Physik als {\em
Dirac-Delta Funktion} bezeichnet, mit der Notation $h(x,z)\equiv \delta(x-z)$
und ist mathematisch gesehen eine {\em Distribution} oder uneigentliche
Funktion. Als nächstes untersuchen wir diese Funktion im Detail.

\subsubsection{Dirac-$\delta$-Funktion}

Die wichtigste Eigenschaft der $\delta$-Funktion ist
\begin{align}
\int \limits_{-\infty}^{\infty} \dx \delta(x-x_0) f(x) = f(x_0),
\end{align}
für alle Funktionen $f(x)$.
\begin{info}
$f(x)$ soll beliebig oft differenzierbar sein.
\end{info}

\par
Wir können die $\delta$-Funktion beliebig genau approximieren mit der Funktion
\begin{align}
h_{\varepsilon}(x) = \frac{1}{\sqrt{\varepsilon\pi}}e^{-\frac{x^2}{\varepsilon}}
\end{align}
\begin{center}
\psset{unit=1cm}
\begin{pspicture}(-1,-1)(10,6)
 \psaxes[labels=none,ticks=none]{->}(0,0)(-0.5,-0.5)(8,5)[\textbf{$x$},-90][\textbf{$y$},0]
 \psplot[linewidth=1.2pt,algebraic=true,plotpoints=3600]{0}{8}{8/(3.41)^0.5*(2.72)^(-(x-3)^2)}

 \psxTick(3){$x_0$}
 \psxTick(2){$x_0-\frac{\varepsilon}{2}$}
 \psxTick(4){$x_0+\frac{\varepsilon}{2}$}
 
 \rput[l](3.6,4.2){$h_\varepsilon(x-x_0)$}
\end{pspicture}
\end{center}
Für kleine $\varepsilon$ gilt somit
\begin{align*}
&\int \limits_{-\infty}^{\infty} \dx h_{\varepsilon}(x-x_0)f(x) \\
= &\int
\limits_{-\infty}^{\infty} \dx h_{\varepsilon}(x-x_0)\left[
f(x_0) + f'(x_0)(x-x_0) + \frac{f''(x_0)}{2}(x-x_0)^2 + \sigma(x-xo) \right]
\end{align*}
Es gilt
\begin{itemize}
  \item
\begin{align*}
&\int \limits_{-\infty}^{\infty} \dx h_{\varepsilon}(x-x_0)f(x_0) = f(x_0) \int
\limits_{-\infty}^{\infty}\dx
\frac{1}{\sqrt{\varepsilon\pi}}e^{\frac{-(x-x_0)^2}{\varepsilon}}\\
&\overset{z = \frac{x-x_0}{\sqrt{\varepsilon}}}{=} f(x_0)
\underbrace{\int \limits_{-\infty}^{\infty}\dz \frac{1}{\sqrt{\pi}}
e^{-z^2}}_{1} = f(x_0)
\end{align*}

\item
\begin{align*}
&\int \limits_{-\infty}^{\infty}h_{\varepsilon}(x-x_0)f'(x_0)(x-x_0) \\
&\overset{z=x-x_0}{=} f'(x_0)\int \limits_{-\infty}^{\infty}\dz
\frac{1}{\sqrt{\varepsilon\pi}}
\underbrace{e^{-\frac{z^2}{\epsilon}}}_{\text{punktsymmetrisch}} z = 0
\end{align*}

\item
\begin{align*}
&\int \limits_{-\infty}^{\infty}
h_{\varepsilon}(x-x_0)\frac{1}{2}f''(x_0)(x-x_0)\\
&\overset{z=\frac{x-x_0}{\sqrt{\varepsilon}}}{=} \frac{1}{2} f''(x_0) \int
\limits_{-\infty}^{\infty}\dz \varepsilon \frac{1}{\sqrt{\pi}} z^2e^{-z^2}\\
&= \frac{1}{4}f''(x_0)\cdot\varepsilon\quad 
\underset{\varepsilon\to0}{\rightarrow} 0
\end{align*}
\end{itemize}
Somit folgt
\begin{align}
\int \limits_{-\infty}^{\infty}\dx h_{\varepsilon}(x-x_0)f(x) = f(x_0) +
\frac{1}{4} f''(x_0)\varepsilon + \sigma(\varepsilon^2) \underset{\varepsilon\to0}{\rightarrow} f(x_0)
\end{align}

Die Frage ist somit, ob man das Integral mit dem Limes vertauschen kann
\begin{align*}
\int \limits_{-\infty}^{\infty}\dx \delta(x-x_0) f(x) &= f(x_0) =
\lim_{\varepsilon\to0} \int \limits_{-\infty}^{\infty}\dx h_{\varepsilon}(x-x_0)
f(x)\\
&''='' \int \limits_{-\infty}^{\infty} \dx \lim_{\varepsilon\to0}
h_{\varepsilon}(x-x_0)f(x)
\end{align*}

\begin{align*}
\Rightarrow \delta(x-x_0) &''='' \lim_{\varepsilon\to0} h_{\varepsilon}(x-x_0)
=
\lim_{\varepsilon\to0}
\frac{1}{\sqrt{\pi\varepsilon}} e^{\frac{-(x-x_0)^2}{\varepsilon}}\\
& = \begin{cases}0 & x\neq x_0\\\infty & x = x_0\end{cases}
\end{align*}

Wir sehen, dass die $\delta$-Funktion keine eigentliche Funktion ist sondern,
dass sie erst Sinn ergibt, indem man über sie integriert. Trotzdem ist es
möglich, mit ihr als abstraktes Objekt zu rechnen.
\par
Eigenschaften der $\delta$-Funktion
\begin{itemize}
  \item $\int_{-\infty}^{\infty}\dx\delta(x) = 1$
  \item $\int_{-\eta}^{\eta}\dx f(x)\delta(x) = f(0)$ für beliebige $\eta > 0$
  \par
  \begin{info}
\begin{align*}
&\int \limits_{-\eta}^{\eta}\dx f(x)\delta(x) = \lim_{\varepsilon\to0} \int
\limits_{-\eta}^{\eta}\dx h_{\varepsilon}(x)f(x) \\
&= \lim_{\varepsilon\to0}\left[\underbrace{\int \limits_{-\infty}^{\infty}\dx
h_{\varepsilon}(x)f(x)}_{f(0)} - \underbrace{\int \limits_{-\infty}^{-\eta}\dx
h_{\varepsilon}(x)f(x)}_{=0} - \underbrace{\int \limits_{\eta}^{\infty}\dx
h_{\varepsilon}(x)f(x)}_{=0}\right]\\
&= f(0)
\end{align*}
\end{info}
\item $\delta(x) = \delta(-x)$\\
\begin{info}
$\int_{-\infty}^{\infty} \dx \delta(x)f(x) = f(0) = \int_{-\infty}^{\infty}
\dx \delta(x)f(-x) = \int_{-\infty}^{\infty} \dx \delta(-x)f(x)$
\end{info}
\item $\delta(x)\cdot x = 0$\\
\begin{info} $\int_{-\infty}^{\infty} \dx f(x)\,x\,\delta(x) = 
\int_{-\infty}^{\infty} \dx g(x)\,\delta(x) = g(0) = f(0)\cdot 0 = 0$ 
\end{info}
\end{itemize}

\begin{Beispiel}
\par
\begin{itemize}
  \item $\int_{-\infty}^{\infty} \dx f(x)[\delta(x-1)+\delta(x+1)] = f(1)+f(-1)$
  \item $\int_{a}^{b} \dx f(x) \delta(x-c) = \begin{cases}f(c) & a
< c < b\\0 & c < a \text{ oder } c > b\end{cases}$
\item $\int_{-1}^{2}\dx x^2 \delta(x-1) = 1$
\item $\int_{-1}^{2}\dx x^2\delta(x+2) = 0$
\item $\int_{-\infty}^{\infty}\dx e^{-x} \delta(x-y^2) = e^{-y^2}$
\end{itemize}
\end{Beispiel}

Alternative Funktionsfolgen, die gegen die $\delta$-Funktion konvergieren
\begin{itemize}
  \item $\frac{1}{\pi}\sin{\frac{Nx}{x}} \qquad \text{ für } N \to \infty$
  \item $\frac{1}{\pi}\frac{\varepsilon}{\varepsilon^2 + x^2} \qquad \text{ für
  } \varepsilon \to 0$
\end{itemize}
\begin{center}
\begin{pspicture}(-3,-1)(3,6)
 \psset{xunit=.25,yunit=2} 
 
 \psaxes[labels=none,ticks=none]{->}(0,0)(-11,-0.25)(11,2)[\textbf{$x$},-90][\textbf{$y$},0]
 \psplot[linewidth=1.2pt,algebraic=true,plotpoints=3600]{-10}{10}{sin(x)/x} 
 \psyTick(1){$N$}
 \rput[l](2,1){$\frac{1}{\pi}\sin\left(\frac{Nx}{x}\right)$}
\end{pspicture}
\begin{pspicture}(-3,-1)(3,6)
 \psset{xunit=.25,yunit=2} 

 \psaxes[labels=none,ticks=none]{->}(0,0)(-11,-0.25)(11,2)[\textbf{$x$},-90][\textbf{$y$},0]
 \psplot[linewidth=1.2pt,algebraic=true,plotpoints=3600]{-10}{10}{1/(x^2+1)}
 \psyTick(1){$\frac{1}{\varepsilon}$}
 \rput[l](2,1){$\frac{1}{\pi}\frac{\varepsilon}{\varepsilon^2+x^2}$}
\end{pspicture}
\end{center}

\begin{Bemerkung}
Mathematisch ist die $\delta$-Funktion eine Distribution.
\begin{itemize}
  \item Mit S bezeichnen wir den {\em Schwartz-Raum}. Für $f
  \in S$ gilt, $f \in C^{\infty}$ und $x^{\alpha}f(x)
  \underset{x\to\pm\infty}{\to} 0\quad \alpha \in \N_0$
  \item Distributionen sind stetige lineare Abbildungen vom Schwartz-Raum in
  die Menge der reellen Zahlen.
  \begin{align*}
  &T: f\in S \rightarrow \R\\
  &\text{mit } T(\alpha f + \beta g) = \alpha T(f) + \beta T(g).
  \end{align*}
\end{itemize}
\end{Bemerkung}

\begin{Beispiel}
\begin{itemize}
  \item Für eine Funktion $h(x)$ erhalten wir die Distribution
  $$H(f) \equiv \int \dx h(x)f(x)$$
  \item Die Abbildung
  $$f(x) \mapsto f(x_0)$$
  ist gerade die $\delta$-Funktion.
  \item Weitere Distributionen sind auch
  $$f(x) \mapsto f'(x_0)$$
  oft als $\delta'(x-x_0)$ bezeichnet.
\end{itemize}
\end{Beispiel}

\subsubsection{Green'sche Funktion für den Oszillator}
Die Differentialgleichung für die Green'sche Funktion lautet
\begin{align}
\frac{d^2}{dx^2}G(x,z) + a\frac{d}{dx}G(x,z) + bG(x,z) = \delta(x-z)
\end{align}
Integrieren wir diese Gleichung um eine kleine Umgebung
$\int_{z-\varepsilon}^{z+\varepsilon} \dx$ erhalten wir
\begin{align*}
&\int \limits_{z-\varepsilon}^{z+\varepsilon} \dx \frac{d^2}{dx^2}G(x,z) + a\int
\limits_{z-\varepsilon}^{z+\varepsilon} \frac{d}{dx}G(x,z) + b \int
\limits_{z-\varepsilon}^{z+\varepsilon} \dx G(x,z)  = 1 \\
\Leftrightarrow &
\left[\frac{d}{dx} G(x,z)\right]_{z-\varepsilon}^{z+\varepsilon}
+ \left[G(x,z)\right]_{z-\varepsilon}^{z+\varepsilon}
+ G(z,z)\varepsilon
\end{align*}
Die Funktion $G(x,z)$ ist aber stetig bei $x = z$ und nur die Ableitung hat
einen Sprung.
\begin{center}
\begin{pspicture}(-1,-1)(5.5,5)
 \psaxes[labels=none,ticks=none]{->}(0,0)(-0.5,-0.5)(5,4)[$x$,-90][\textbf{
 $G(x,z)$},0]
 \psplot[linewidth=1.2pt,algebraic=true,plotpoints=3600]{0}{2.5}{0.1*x^2+2}
 \psplot[linewidth=1.2pt,algebraic=true,plotpoints=3600]{2.5}{5}{0.1*(x-5)^2+2}
 
 \psxTick(2.5){$z$}
\end{pspicture}
\end{center}
Somit gilt für $\varepsilon \to 0$
\begin{align}
  \left[\frac{d}{dx} G(x,z)\right]_{x-\varepsilon}^{x+\varepsilon} =
  G'(z+\varepsilon,z) - G'(z-\varepsilon,z) = 1
\end{align}
Da aber $\delta(x-z) = 0$ für $x\neq z$ gilt
\begin{align}
\frac{d^2}{dx^2} G(x,z) + a \frac{d}{dx} G(x,z) + bG(x,z) = 0 \quad x\neq z
\end{align}
und $G(x,z)$ kann konstruiert werden aus der Lösung der homogenen Gleichung mit
obiger Bedingung für $x=z$.

\begin{Beispiel}
Green'sche Funktion zur Differentialgleichung für $x\in[0,\frac{\pi}{2}]$ mit
\begin{align*}
\frac{d^2}{dx^2} G(x,z) + G(x,z) = \delta(x-z) \quad G(0,z) = 0 \quad
G(\frac{\pi}{2}, z) = 0
\end{align*}
\begin{align*}
\text{Ansatz: } &G(x,z) = A(z)\sin{x} & x < z\\
 & G(x,z) = B(z)\cos{x} & x > z
\end{align*}
Somit folgt:
\begin{align*}
\text{ Stetigkeit: } & A(z)\sin{z} - B(z)\cos{z} = 0\\
\text{Springen in Ableitung: } & -A(z)\cos{z} + B(z)\sin{z} = -1
\end{align*}
\begin{align*}
&A(z) = -\cos{z}\\
&B(z) = -\sin{z}\\
\end{align*}
Die Green'sche Funktion hat die Form
\begin{align*}
G(x,z) = \begin{cases} -\sin{z}\cos{x} & x\ge z\\ -\cos{z}\sin{x} & x<z\end{cases}
\end{align*}
\begin{center}
\begin{pspicture}(-1,-2.5)(5.5,3)
 \psaxes[labels=none,ticks=none]{->}(0,0)(-0.5,-2)(5,2)[$x$,-90][\textbf{
 $\frac{\partial}{\partial x}G(x,z)$},0]
 \psplot[linewidth=1.2pt,algebraic=true,plotpoints=3600]{0}{2.5}{0.1*x^2+1}
 \psplot[linewidth=1.2pt,algebraic=true,plotpoints=3600]{2.5}{5}{-0.1*(x-5)^2-1}
 \psline[linewidth=0.5pt,linestyle=dashed,linecolor=lightgray](2.5,1.625)(2.5,-1.625)
 \psxTick(2.5){$z$}
\end{pspicture}
\begin{pspicture}(-1,-2.5)(5.5,3)
 \psaxes[labels=none,ticks=none]{->}(0,0)(-0.5,-2)(5,2)[$x$,-90][$y$,0]
 \psplot[linewidth=1.2pt,algebraic=true,plotpoints=3600]{0}{2}{-(1-(2.71)^(-x))}
 \psplot[linewidth=1.2pt,algebraic=true,plotpoints=3600]{2}{4}{-(1-(2.71)^(x-4))}

 \psxTick(2){$z$}
 \psxTick(4){$\pi/2$}
\end{pspicture}
\end{center}
\end{Beispiel}

\begin{Beispiel}
Finde eine partikulär Lösung zur Differentialgleichung auf dem Intervall
$[0, \frac{\pi}{2}]$
\begin{align*}
y''(x) + y'(x) = \frac{1}{\sin{x}}
\end{align*}
\begin{align*}
\Rightarrow y(x) & = \int \limits_{0}^{\frac{\pi}{2}} \dz \frac{1}{\sin{z}}
G(x,z)\\
& = -\cos{x}\int \limits_{0}^{x} \dz \frac{\sin{z}}{\sin{z}} - \sin{x} \int
\limits_{x}^{\frac{\pi}{2}}\dz \frac{\cos{z}}{\sin{z}}\\
& = -x\,\cos{x} + \sin{x}\,\ln{\sin{x}}
\end{align*}
\end{Beispiel}