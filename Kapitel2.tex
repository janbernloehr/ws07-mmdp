\section{Integration}
Das Integral $I = \int \limits_{a}^{b} f(x) \dx \equiv \int \limits_{a}^{b} dx f(x)$ kann als Fläche unter der Kurve $f(x)$ verstanden werden.

Die formale Definition folgt ebenfalls aus einem Grenzwertprozess. Dazu wird
das Intervall $a \le x \le b$ in eine große Anzahl von kleinen Intervallen
aufgeteilt
\begin{equation*}
 a = \xi_0 < \xi_1 < \xi_3 < ... < \xi_n = b
\end{equation*}
und dann folgende Summe geformt
\begin{equation}
 S = \sum \limits_{i=0}^{n} f(x_i)(\xi_i-\xi_{i-1}).
\end{equation}

\begin{center}
\psset{unit=1cm}
\begin{pspicture}(-1,-1)(10,6)
 \psline[linewidth=0.5pt,arrowsize=4pt]{->}(-1,0)(10,0)
 \psline[linewidth=0.5pt,arrowsize=4pt]{->}(0,-1)(0,6)

\psplot[linewidth=1.2pt,algebraic=true]{0}{10}{sin(2/3.41*x)+4} 
 \pscustom[fillstyle=solid,fillcolor=lightgray]{
\psplot[linewidth=1.2pt,algebraic=true]{2}{8}{sin(2/3.41*x)+4}
\psline[linestyle=none](8,3)(8,0)
\psline[linestyle=none](2,0)(2,4.922)
}
 
 \rput(-0.36,-0.26){$0$}
 \rput(2,-0.26){$a$}
 \rput(8,-0.26){$b$}
 \rput(9.7,-0.26){$x$}
 \rput(-0.36,5.7){$y$}
 \rput(8.5,4){$f(x)$}
\end{pspicture}
\end{center}

Die Positionen $x_i$ sind beliebig im Intervall $\xi_{i+1} \le x_i \le \xi_i$.
Das (Riemann'sche) Integral erhält man nun im Limes, wenn man die Länge der
Intervalle $\xi_{i-1} \le x \le \xi_i$ gegen Null streben lässt.

\begin{center}
\psset{unit=1cm}
\begin{pspicture}(-1,-1)(10,6)
 \psline[linewidth=0.5pt,arrowsize=4pt]{->}(-1,0)(10,0)
 \psline[linewidth=0.5pt,arrowsize=4pt]{->}(0,-1)(0,6)

\psplot[linewidth=1.2pt,algebraic=true]{0}{10}{sin(2/3.41*x)+4} 

\psframe[fillcolor=lightgray](2,0)(3,5)
\psframe[fillcolor=lightgray](3,0)(4,4.89)
\psframe[fillcolor=lightgray](4,0)(5,4.48)
\psframe[fillcolor=lightgray](5,0)(6,3.92)
\psframe[fillcolor=lightgray](6,0)(7,3.38)
\psframe[fillcolor=lightgray](7,0)(8,3.05)

\psline[linewidth=0.5pt,arrows=-*,linecolor=gray,linestyle=dashed](2.5,-0.6)(2.5,5)
\psline[linewidth=0.5pt,arrows=-*,linecolor=gray,linestyle=dashed](3.5,-0.6)(3.5,4.89)
\psline[linewidth=0.5pt,arrows=-*,linecolor=gray,linestyle=dashed](4.5,-0.6)(4.5,4.48)
\psline[linewidth=0.5pt,arrows=-*,linecolor=gray,linestyle=dashed](5.5,-0.6)(5.5,3.92)
\psline[linewidth=0.5pt,arrows=-*,linecolor=gray,linestyle=dashed](6.5,-0.6)(6.5,3.38)
\psline[linewidth=0.5pt,arrows=-*,linecolor=gray,linestyle=dashed](7.5,-0.6)(7.5,3.05)

 
 \rput(-0.36,-0.26){$0$}
 \rput(1.8,-0.26){$a=\xi_1$}
 \rput(3,-0.26){$\xi_2$}
 \rput(4,-0.26){$\xi_3$}
 \rput(5,-0.26){$\xi_4$}
 \rput(6,-0.26){$\xi_5$}
 \rput(7,-0.26){$\xi_6$}
 \rput(8.2,-0.26){$b=\xi_7$}
 \rput(9.7,-0.26){$x$}
 
 \rput(2.5,-0.8){$x_1$}
 \rput(3.5,-0.8){$x_2$}
 \rput(4.5,-0.8){$x_3$}
 \rput(5.5,-0.8){$x_4$}
 \rput(6.5,-0.8){$x_5$}
 \rput(7.5,-0.8){$x_6$}
 
 \rput(-0.36,5.7){$y$}
 \rput(8.5,4){$f(x)$}
\end{pspicture}
\end{center}

Eine Funktion $f(x)$ heißt integrierbar, wenn dieser Grenzwert existiert. Desweiteren muss er eindeutig sein, d.h., unabhängig von der Wahl von $\xi_i$ und $x_i$.

\begin{Beispiel}
\begin{equation*}
\int \limits_{0}^{b} \dx x = \frac{1}{2}b^2
\end{equation*}
Unterteile das Intervall $0 \le x \le b$ mit $\Delta x = \frac{b}{n}$. Wähle
$x_i = \xi_i$, so gilt für die Summe
\begin{align*}
 S &= \sum \limits_{k=1}^{n} \Delta x \cdot k \cdot \Delta x = b^2 \frac{1}{n^2} \sum \limits_{k=1}^{n} k \\ &= b^2 \frac{1}{n^2} \frac{n(n+1)}{2} = \frac{b^2}{2} \frac{n+1}{n}
\end{align*}
Den Grenzwert $\Delta x \to c$ erhalten wir somit für $n \to \infty$ und das Integral wird zu
\begin{align*}
 I = \int \limits_{a}^{b} \dx x = \lim \limits_{n\to\infty} \frac{b^2}{2} \frac{n+1}{n} = \frac{b^2}{2}
\end{align*}

\end{Beispiel}

\begin{Bemerkung}
Jedes Integral kann geschrieben werden als
\begin{align}
\int \limits_{a}^{b} f(x) = \lim \limits_{n\to \infty} \sum \limits_{i = 1}^{n}
f(x_i)\Delta x,\quad \text{ mit } \Delta x = \frac{b-a}{n} f(x_i)\Delta x.
\end{align}
\end{Bemerkung}
Das Integral hat folgende Eigenschaften
\begin{itemize}
 \item $\int \limits_{a}^{b} \dx 0 = 0$
 \item $\int \limits_{a}^{a} \dx f(x) = 0$
 \item $\int \limits_{a}^{c} \dx f(x) = \int \limits_{a}^{b} \dx f(x) + \int \limits_{b}^{c} \dx f(x)$
 \item $\int \limits_{a}^{b} \dx [f(x) + g(x)] = \int \limits_{a}^{b} \dx f(x) = \int \limits_{a}^{b} \dx g(x)$
 \item $\int \limits_{a}^{\infty} \dx f(x) = \lim \limits_{b\to\infty} \int \limits_{a}^{b} \dx f(x)$
\end{itemize}
Falls $a < b$, so definiert man
\begin{align}
 \int \limits_{a}^{b} \dx f(x) = -\int \limits_{b}^{a} \dx f(x)
\end{align}


\subsection{Stammfunktion}
Ersetzt man die obere Integrationsgrenze $b$ durch $x$, so definiert das Integral eine neue Funktion
\begin{equation}
 F(x) = \int \limits_{a}^{x} \du f(u)
\end{equation}
Diese Funktion lässt sich jetzt differenzieren
\begin{align*}
 F(x+\Delta x) &= \int \limits_{a}^{x+\Delta x} \du f(u) = \int \limits_{a}^{x}
 \du f(u) + \int \limits_{x}^{x+\Delta x} \du f(u) \\ &= F(x) + \int
 \limits_{x}^{x+\Delta x} \du f(u)
\end{align*}
Somit folgt die Ableitung
\begin{align*}
 \frac{d}{dx}F(x) &= \lim \limits_{\Delta x \to 0} \frac{F(x + \Delta x) - F(x)}{\Delta x}\\
 &= \lim \limits_{\Delta x \to 0} \frac{1}{\Delta x} \int
\limits_{x}^{x+\Delta x} \du f(u)\\
 &= \lim \limits_{\Delta x \to 0} \frac{1}{\Delta x} \left[ \Delta x f(x) + \sigma(\Delta x^2)\right] = f(x)
\end{align*}

\begin{center}
\psset{unit=1cm}
\begin{pspicture}(-1,-1)(10,6)
 \psline[linewidth=0.5pt,arrowsize=4pt]{->}(-1,0)(10,0)
 \psline[linewidth=0.5pt,arrowsize=4pt]{->}(0,-1)(0,6)

\psplot[linewidth=1.2pt,algebraic=true]{0}{10}{sin(2/3.41*x)+4}
 \psline[linewidth=0.5pt](0,4.2)(5,4.2)
\psline[linewidth=0.5pt,arrows=-*,linecolor=gray,linestyle=dashed](4.5,0)(4.5,4.48)
\psline[linewidth=0.5pt,arrows=-*,linecolor=gray,linestyle=dashed](5.5,0)(5.5,3.92)
 
 \rput(-0.36,-0.26){$0$}
 \rput(5,-0.26){$\Delta x$}
 \rput(9.7,-0.26){$x$}
 \rput(-0.36,5.7){$y$}
 \rput(6.5,4.4){$I \approx \Delta x f(x)$}
\end{pspicture}
\end{center}


\begin{Definition}[Stammfunktion]
 Jede Funktion $F(x)$ mit
\begin{equation}
\frac{d}{dx} F(x) = f(x)
\end{equation}
heißt {\em Stammfunktion} von $f(x)$ oder {\em unbestimmtes Integral}.
\end{Definition}

Alle Stammfunktionen unterscheiden sich nur durch eine Konstante: $G(x) = F(x) + c$
ist ebenfalls eine Stammfunktion zu $f(x)$.\\
Man schreibt daher für das unbestimmte Integral
\begin{equation}
F(x) = \int \dx f(x)
\end{equation}
Das {\em bestimmte Integral} folgt somit aus der Stammfunktion
\begin{equation}
\int \limits_{a}^{b} \dx f(x) = F(b) - F(a) \equiv F |_{a}^{b}
\end{equation}
Es folgen sogleich eine Reihe elementarer Integrale
\begin{itemize}
  \item $\int \dx x^n = \frac{x^{n+1}}{n+1}$
  \item $\int \dx e^{ax} = \frac{1}{a}e^{ax}$
  \item $\int \dx \frac{a}{x} = a\cdot\ln{|x|} $
  \item $\int \dx \cos{bx} = \frac{1}{b}\sin{bx} $
\end{itemize}

\subsection{Integrationsregeln}

\par{\bf Substitution}
Es sei $y = g(x)$ so gilt
\begin{equation}
\int \dy f(y) = \int \dx g'(x)f(g(x))
\end{equation}

\begin{Bemerkung}
Schreibe $\frac{d}{dx}y = g'(x) \Rightarrow dy = g'(x)dx$
\begin{equation*}
\int \limits_{a}^{b} \dy f(y) = \int \limits_{g^{-1}(a)}^{g^{-1}(b)} \dx
g'(x)f(g(x))
\end{equation*}
\end{Bemerkung}

\par{\bf Partielle Integration}
\begin{equation}
\int \dx f'(x)g(x) = f(x)g(x) - \int \dx f(x)g'(x)
\end{equation}

\begin{Beispiel}
\begin{align*}
\int \dx \frac{1}{\sqrt{1-x^2}} &\stackrel{x = \sin{y}}= \int \dy \cos{y}
\frac{1}{\sqrt{1-\sin^2{y}}} = \int \dy \cos{y} \frac{1}{\cos{y}}\\
 &= y = \arcsin{x}
\end{align*}
\begin{align*}
\int \dx \ln{x} &= \int \dx \underset{f'(x)}{1} \cdot \ln{x} = x\ln{x} -
\int \dx x \cdot \frac{1}{x} \\
&= x\ln{x} - x
\end{align*}
\end{Beispiel}