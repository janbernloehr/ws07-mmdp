\section{Krummlinige Koordinaten}
Die drei häufigsten Koordinatensysteme sind
\begin{align*}
&x,y,z : \text{ Kartesische Koordinaten}\\
&r,\varphi,z : \text{ Zylinder Koordinaten}\\
&r,\varphi,\vartheta : \text{ Kugel Koordinaten}\\
\end{align*}
Eine gemeinsame Eigenschaft dieser Systeme ist, dass sie orthogonal sind und,
dass der Laplace-Operator separiert. (siehe später)
\par
Die Koordinaten Transformation hat im Allgemeinen die Form
\begin{align*}
&x_1 = x_1(u_1,u_2,u_3)\\
&x_2 = x_2(u_1,u_2,u_3)\\
&x_3 = x_3(u_1,u_2,u_3)\\
&\vec{r}(u_1,u_2,u_3) = \begin{pmatrix}x_1\\x_2\\x_3\end{pmatrix}.
\end{align*}
Die Tangentenvektoren an die Koordinatenlinien sind definiert als
\begin{align*}
\vec{T}_i \equiv \vec{e}_{u_i} = \frac{1}{h_i}\frac{\partial\vec{r}}{\partial
u_i}\qquad \text{ mit } h_i = \left|\frac{\partial\vec{r}}{\partial u_i} \right|
\end{align*}
Wir sind interessiert an orthogonalen Koordinatensystemen mit
\begin{align*}
\vec{e}_{u_i} \cdot \vec{e}_{u_j} = \delta_{ij} = \begin{cases}0 &
i\neq j\\1 &i = j\end{cases}
\end{align*}
Somit bilden in jedem Punkt $\vec{r}$ die Vektoren $\vec{e}_{u_i}$ eien
Orthonormierte Basis. Die Vektoren $\vec{e}_{u_i}$ hängen somit explizit vom
Raumpunkt $\vec{r}$ ab.
\begin{Bemerkung}
Wir können auch schreiben
\begin{align*}
\vec{e}_{u_1} = \vec{e}_{u_2} \times \vec{e}_{u_3}
\end{align*}
\end{Bemerkung}
Ein Skalarfeld $\phi(\vec{r})$ können wir einfach in den neuen Koordinaten
ausdrücken
\begin{align*}
\phi(u_1,u_2,u_3) = \phi(\vec{r}(u_1,u_2,u_3)).
\end{align*}
Für ein Vektorfeld $\vec{A}(\vec{r})$ müssen wir zusätzlich die neue Basis
$\vec{e}_{u_i}$ berücksichtigen.
\begin{align*}
\vec{A}(\vec{r}) = \sum \limits_{i=1}^{3} \vec{e}_{u_i} A_{u_i}(u_1,u_2,u_3).
\end{align*}
$A_{u_i}$ erhalten wir mittels dem Skalarprodukt
\begin{align*}
A_{u_i} = \vec{A}\cdot\vec{e}_{u_i}.
\end{align*}
\begin{Bemerkung}
In den krummlinigen Koordinaten sind jetzt $A_{u_i}$ und $\vec{e}_{u_i}$ von
$u_1,u_2,u_3$ abhängig.
\begin{align*}
\Rightarrow \partial_{u_j} \vec{A} = \sum \limits_{i=1}^{3}
\left[\frac{\partial A_{u_i}}{\partial u_j} + A_{u_i} \frac{\partial
\vec{e}_{u_i}}{\partial u_j}\right]
\end{align*}
\end{Bemerkung}
\begin{Beispiel}
Geschwindigkeitsvektor $\vec{v}$ in Zylinder-Koordinaten
\begin{align*}
\vec{v}(t) = \frac{d}{dt}\vec{r}(t) = \frac{d}{dt}\left(r\vec{e}_r +
z\vec{e}_z\right) = \dot{r}\vec{e}_r + r\dot{\vec{e}}_r + \dot{z}\vec{e}_z +
z\dot{\vec{e}}_z
\end{align*}
\end{Beispiel}
Im Folgenden wollen wir untersuchen, wie sich die Ableitungsoperatoren
transformieren.
\begin{Definition}[Gradient]
Die Komponente von $\nabla\phi$ in $\vec{e}_{u_i}$ ist
\begin{align*}
(\nabla\phi)_{u_i} &= \nabla\phi \cdot \vec{e}_{u_i} =
\nabla\phi\cdot\frac{1}{h_{u_i}}\frac{\partial\vec{r}}{\partial u_i} =
\frac{1}{h_{u_i}} \sum \limits_{j=1}^{3}\frac{\partial x_j}{\partial u_i}
\frac{\partial \phi}{\partial x_j} \\
&= \frac{1}{h_{u_i}}
\partial_{u_i}\phi(\vec{r}(u_1,u_2,u_3)) =
\frac{1}{h_{u_i}}\frac{\partial\phi}{\partial u_i}\\
\Rightarrow  \nabla \phi &= \sum \limits_{i} \vec{e}_{u_i} \frac{1}{h_i}
\frac{\partial \phi}{\partial u_i}
\end{align*}
\end{Definition}

\begin{Definition}[Divergenz]
Für die Divergenz in krumlinigen Koordianten gilt
\begin{align*}
\begin{split}
\div \vec{A}(u_1,u_2,u_3) = \frac{1}{h_{u_1} h_{u_2}
h_{u_3}}\big[\frac{\partial}{\partial u_1} \left(h_{u_2}h_{u_3}A_{u_1}\right)
+ \frac{\partial}{\partial u_2}\left(h_{u_1} h_{u_3} A_{u_2}\right) \\ +
\frac{\partial}{\partial u_3} \left(h_{u_1} h_{u_2} A_{u_3}\right) \big]
\end{split}
\end{align*}
\end{Definition}

\begin{info}
Betrachte den Anteil $\vec{e}_{u_i} A_{u_1}$
\begin{align*}
\nabla\cdot(\vec{e}_{u_1} A_{u_1}) &=
\nabla\cdot((\vec{e}_{u_2} \times \vec{e}_{u_3})A_{u_1}) \\
&= \nabla\cdot(h_{u_2} h_{u_3} A_{u_1} (\nabla u_2 \times \nabla u_3))\\
&= \nabla(h_{u_2} h_{u_3} A_{u_1})\cdot(\nabla u_2 \times \nabla u_3) \\ &
\quad + h_{u_2} h_{u_3} A_{u_1} \nabla \cdot (\nabla u_2 \times \nabla u_3)\\
&= \frac{1}{h_{u_2} h_{u_3}}\vec{e}_{u_1}\cdot\nabla(h_{u_2} h_{u_3} A_{u_1})\\
&= \frac{1}{h_{u_1} h_{u_2} h_{u_3}} \frac{\partial}{\partial u_1} (h_{u_2}
h_{u_3} A_{u_1})
\end{align*}
und analog für die anderen Komponenten.
\end{info}

\begin{Definition}[Laplace Operator]
Als wichtige Anwendung erhalten wir den Laplace Operator in krummlinigen
Koordinaten
\begin{align*}
\nabla\phi(u_1,u_2,u_3) = \frac{1}{h_{u_1} h_{u_2}
h_{u_3}}\big[\frac{\partial}{\partial u_1}\left(\frac{h_{u_2}
h_{u_3}}{h_{u_1}} \frac{\partial \phi}{\partial u_1}\right) +
\frac{\partial}{\partial u_2}\left(\frac{h_{u_1}
h_{u_3}}{h_{u_2}} \frac{\partial \phi}{\partial u_2}\right) \\+
\frac{\partial}{\partial u_3}\left(\frac{h_{u_1}
h_{u_2}}{h_{u_3}} \frac{\partial \phi}{\partial u_3}\right)
 \big]
\end{align*}
\end{Definition}

\begin{Definition}[Rotaion]
Zur Vollständigkeit noch die Rotaion
\begin{align*}
\rot \vec{A}(u_1,u_2,u_3) &= \frac{1}{h_{u_2}
h_{u_3}}\vec{e}_{u_1}\left[\frac{\partial}{\partial u_2}\left(h_{u_3}
A_{u_3}\right) - \frac{\partial}{\partial u_3}\left(h_{u_2}
A_{u_2}\right)\right]\\
& + \frac{1}{h_{u_1}
h_{u_3}}\vec{e}_{u_2}\left[\frac{\partial}{\partial u_3}\left(h_{u_1}
A_{u_1}\right) - \frac{\partial}{\partial u_1}\left(h_{u_3}
A_{u_3}\right)\right]\\
& + \frac{1}{h_{u_1}
h_{u_2}}\vec{e}_{u_3}\left[\frac{\partial}{\partial u_1}\left(h_{u_2}
A_{u_2}\right) - \frac{\partial}{\partial u_2}\left(h_{u_1}
A_{u_1}\right)\right]
\end{align*}
\end{Definition}

\subsection{Zylinderkoordinaten}
Wir haben die Transformation
\begin{align*}
\vec{r} = \begin{pmatrix}x\\y\\z\end{pmatrix} = \begin{pmatrix}r\cos(\varphi)
\\ r\sin(\varphi) \\ z\end{pmatrix}
\end{align*}
mit $0\le r<\infty, 0\le\phi<2\pi, -\infty<z<\infty$
\begin{align*}
&\vec{e}_r = \begin{pmatrix}\cos(\varphi) \\ \sin(\varphi) \\ 0\end{pmatrix}
\qquad &h_r = 1\\
&\vec{e}_\varphi = \begin{pmatrix}-\sin(\varphi) \\ \cos(\varphi) \\
0\end{pmatrix} \qquad &h_\varphi = r\\
&\vec{e}_z = \begin{pmatrix}0 \\ 0 \\ 1\end{pmatrix}
\qquad &h_z = 1\\
\end{align*}
\begin{align*}
\nabla\phi &= \vec{e}_r \frac{\partial}{\partial r}\phi +
\frac{1}{r}\vec{e}_\varphi\frac{\partial}{\partial \varphi}\phi + \vec{e}_z
\frac{\partial}{\partial z}\phi\\
\Delta\phi &= \left[\frac{1}{r}\partial_r r \partial_r +
\frac{1}{r^2}\partial^2_\varphi + \partial^2_z \right]\phi \\
&= \partial^2_r\phi + \frac{1}{r}\partial_r\phi +
\frac{1}{r^2}\partial^2_\varphi\phi + \partial^2_z\phi\\
\int \ dr^3\ &= \int \dphi \int \dr r \int \dz
\end{align*}

\subsection{Kugelkoordinaten}
\begin{align*}
\vec{r} = \begin{pmatrix}x \\ y \\ z\end{pmatrix} =
\begin{pmatrix}r\cos(\varphi)\sin(\vartheta) \\ r\sin(\varphi)\sin(\vartheta) \\
r\cos(\vartheta)\end{pmatrix}
\end{align*}
mit $0 \le r < \infty$, $0 \le \vartheta \le \pi$, $0 \le \varphi < 2\pi$.
\begin{align*}
\vec{e}_r = \begin{pmatrix}\cos(\varphi)\sin(\vartheta) \\
\sin(\varphi)\sin(\vartheta) \\ \cos(\vartheta)\end{pmatrix},\qquad h_r = 1
\end{align*}
\begin{align*}
\vec{e}_\vartheta = \begin{pmatrix}\cos(\varphi)\cos(\vartheta) \\
\sin(\varphi)\cos(\vartheta) \\ -\sin(\vartheta)\end{pmatrix},\qquad h_\vartheta
= r
\end{align*}
\begin{align*}
 \vec{e}_\varphi = \begin{pmatrix}-\sin(\varphi) \\ \cos(\varphi) \\
 0\end{pmatrix},\qquad h_\varphi = r\sin(\vartheta)
\end{align*}
\begin{align*}
\nabla\phi = \left[ \vec{e}_r\partial_r\phi +
\frac{1}{r}\vec{e}_\vartheta\partial_\vartheta\phi +
\frac{1}{r\sin(\vartheta)}\vec{e}_\varphi\partial_\varphi\phi \right]
\end{align*}
\begin{align*}
\Delta\phi = \left[ \frac{1}{r^2}\partial_r(r^2\partial_r) +
\frac{1}{r^2\sin(\vartheta)}\partial_\vartheta
(\sin(\vartheta)\partial_\vartheta) +
\frac{1}{r^2\sin^2(\vartheta)}\partial^2_\varphi \right]\phi
\end{align*}
\begin{align*}
\int \dvecr = \int \dr r^2 \int \dtheta \sin(\vartheta) \int \dphi
\end{align*}